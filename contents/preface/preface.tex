\textcolor{black}{
Puji syukur kehadirat Allah SWT atas limpahan rahmat, karunia, serta petunjuk-
Nya sehingga tugas akhir berupa penyusunan skripsi ini telah terselesaikan dengan baik
dan tepat waktu. Skripsi ini disusun sebagai salah satu syarat untuk memperoleh gelar
Sarjana di Program Studi Teknologi Informasi.}

\textcolor{black}{Dalam proses penyusunan tugas akhir yang penuh tantangan ini, penulis telah ba-
nyak mendapatkan arahan, bantuan, dukungan moral, mental, dan juga ilmu pengetahuan
dari berbagai pihak. Oleh karena itu, pada kesempatan ini dengan segala kerendahan hati,
penulis mengucapkan terima kasih yang sebesar-besarnya kepada}

\begin{enumerate}
	\item Bapak Prof. Ir. Hanung Adi Nugroho, S.T., M.E., Ph.D., IPM., SMIEEE., selaku
Ketua Departemen Teknik Elektro dan Teknologi Informasi FT UGM, atas dukung-
an dan arahan yang tak ternilai selama masa studi.
	
	\item Bapak Ir. Lesnanto Multa Putranto, S.T., M.Eng., Ph.D., IPM., SMIEEE., selaku
Sekretaris Departemen, terima kasih atas berbagai kemudahan administrasi dan fasilitas yang telah diberikan selama penulis menempuh studi.
		
	\item Dr. Indriana Hidayah, S.T., M.T., sebagai Dosen Pembimbing Utama, yang telah meluangkan waktu, mencurahkan tenaga dan pikiran, memberikan bimbingan intensif, arahan yang tajam, serta
motivasi berkelanjutan sejak awal hingga akhir penyelesaian skripsi ini.

	\item Syukron Abu Ishaq Alfarozi, S.T., Ph.D., sebagai Dosen Pembimbing
Pendamping, yang telah memberikan arahan berharga, masukan konstruktif, serta
dukungan moral selama proses penyusunan skripsi ini.

% 	\item Seluruh Dosen dan Staf Pengajar di Program Studi Teknologi Informasi,
% Fakultas Teknik, Universitas Gadjah Mada, yang telah memberikan ilmu pengetahuan, waktu dan tenaga,
% dalam membimbing penulis selama menempuh studi di Program Studi Teknologi Informasi.

	\item Ayah, Teguh Bharata Adji, S.T., M.T., M.Eng., Ph.D., yang selalu mengingatkan untuk 
	salat lima waktu tepat waktu, serta berkorban mencari nafkah demi pendidikan anak-anaknya, mendidik menjadi pribadi yang kuat \& tegar.
	Terima kasih atas doa, dukungan moral, motivasi, dan kasih sayang yang tiada henti selama ini.

	\item Ibu, apt. RR. Sabtanti Harimurti, S.Si., M.Sc., Ph.D., yang selalu memberikan kasih sayang, memberikan motivasi, serta doa yang tak pernah putus untuk kesuksesan anak-anaknya.
	Terima kasih atas segala pengorbanan Ibu selama ini.

	\item Kakak tercinta, Lutfi Aji S.T., yang selalu menjadi panutan, mengenalkan arti tentang hidup, mengajar dan mengajak bermain bola basket, serta memberikan dukungan moral dan motivasi selama ini.

	\item Teman-teman "Darurat Bang" TIF 2021, "GMM", "Anak Bunda", terima kasih atas kebersamaan, dukungan, motivasi, serta canda tawa yang telah kita bagi bersama selama ini
	dari bangku SMA hingga perkuliahan. 

	\item Organisasi KMTETI, BEM KM FT Andal, Tim Basket DTETI, DTETI 21, WebDev TC 2023, dan teman-teman ITS\textit{Research Group} atas pengalaman berharga, kebersamaan, suka dan duka, serta ilmu yang telah kita bagi bersama selama ini.



\end{enumerate}

\textcolor{black}{Akhir kata, penulis berharap semoga skripsi ini dapat memberikan manfaat yang
sebesar-besarnya bagi perkembangan ilmu pengetahuan, khususnya di bidang Teknologi
Informasi, dan dapat memberikan inspirasi serta motivasi bagi kita semua, \textit{aamiin}.}

% \noindent\textcolor{red}{Catatan: setiap nama yang dituliskan boleh disertai dengan alasan berterima kasih.}

%\begin{flushright}
%	\begin{tabular}{c}
%		Yogyakarta, 13 Maret 2017 \\
%		\vspace{1cm} \\
%		Canggih
%	\end{tabular}
%\end{flushright}