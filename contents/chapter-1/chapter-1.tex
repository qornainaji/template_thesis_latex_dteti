\chapter{Pendahuluan}

\section{Latar Belakang}

Dunia pendidikan tinggi saat ini menuntut mahasiswa untuk bisa belajar secara mandiri.
Namun banyak dari mereka belum siap menghadapi tuntutan tersebut. Hal ini menimbulkan 
kesenjangan yang serius dalam keterampilan Self-Regulated Learning (SRL), yang dapat 
menghambat keberhasilan belajar dan mengganggu pemerataan akses pendidikan \cite{liu2025srl, xiao2019relationships}. 
Tantangan ini justru diperburuk oleh teknologi digital yang menjadi ciri khas pembelajaran 
masa kini. Meskipun menawarkan fleksibilitas, teknologi tersebut juga menghadirkan distraksi 
yang signifikan dan menuntut disiplin diri yang lebih besar \cite{chitra2022srl}. Kegagalan institusi dalam 
memberikan dukungan yang memadai untuk pengembangan SRL bukan hanya masalah akademik, melainkan 
juga masalah sosial yang berdampak pada prospek karier jangka panjang mahasiswa dan efektivitas 
sistem pendidikan \cite{liu2025srl}.
% ------------------------------------------------

Untuk mengatasi kesenjangan SRL, Zimmerman mengemukakan kerangka kerja yang mencakup tahap 
perencanaan (textit{forethought}), pelaksanaan (textit{performance}), dan refleksi diri (textit{self-reflection}) 
di mana SRL merupakan gabungan strategi kognitif, motivasional, dan perilaku \cite{liu2025srl, akdeniz2022srl}. 
Individu dengan kemampuan regulasi diri yang efektif terbentuk melalui penetapan tujuan, perancangan strategi, 
pemantauan diri secara aktif, dan evaluasi diri yang bijaksana. Sejumlah besar penelitian 
mengonfirmasi hubungan kuat antara keterampilan ini dan pencapaian akademik \cite{liu2025srl, xiao2019relationships}. 
Namun, pengembangannya kerap terhambat oleh hambatan personal, kontekstual, dan sosial—mulai dari 
manajemen waktu yang buruk dan distraksi digital hingga rendahnya kesadaran metakognitif—sehingga 
intervensi terstruktur diperlukan untuk membimbing mahasiswa menjalani proses yang kompleks 
ini \cite{chitra2022srl, 10.3389/feduc.2024.1418297}.
% ------------------------------------------------

Metode tradisional yang paling efektif untuk mengembangkan keterampilan ini adalah umpan balik 
personal dari ahli manusia. Umpan balik berkualitas tinggi yang berfokus tidak hanya pada 
jumlah tugas yang diselesaikan, tetapi juga pada proses belajar dan regulasi diri mahasiswa 
merupakan katalis kuat untuk perbaikan [15, 16]\cite{10.3389/fpsyg.2022.1027266, Bock14032024}. Namun standar emas ini menghadapi krisis 
skalabilitas dengan meningkatnya jumlah mahasiswa dan kebutuhan dukungan yang beragam membuat 
model bimbingan satu-satu sulit diterapkan dalam skala besar. Hal ini diperkuat
dengan studi-studi yang dilakukan dan menunjukkan bahwa  tekanan anggaran mendorong ukuran kelas 
yang semakin besar, sehingga interaksi personal antara dosen dan mahasiswa sangat terbatas \cite{schaffer2017automating, pardo2019using}. 
Ini menciptakan kebutuhan mendesak akan solusi inovatif berbasis teknologi yang dapat melengkapi keahlian manusia.
% ------------------------------------------------

Salah satu teknologi tersebut adalah penggunaan Large Language Models (LLM). LLM dapat 
digunakan sebagai tutor virtual yang mampu memberi umpan balik yang cepat dan konsisten.
LLM telah digunakan di berbagai penelitian dalam konteks pendidikan dengan bermacam terobosan
Temuan terbaru memperlihatkan bahwa ketika LLM digunakan untuk penilaian dan umpan balik, 
\textit{artificial intelligence} (AI) cenderung murah hati dalam memberikan nilai sehingga nilainya
lebih tinggi daripada nilai yang diberikan ahli manusia. 
Sementara itu penilaian 
antar rekan dan dosen lebih rendah dan sesuai dengan performa mahasiswa. 
Umpan balik AI sering terstruktur, tetapi tetap 
memerlukan pengawasan ahli manusia agar relevan secara pedagogis sehingga pendekatan 
hibrida (penggabungan AI dan ahli manusia) direkomendasikan \cite{Usher2025GenAIAssessments}.
Pendekatan hibrida direkomendasikan karena dapat menggabungkan kecepatan dan skalabilitas AI dengan
penguasaan konteks, pemahaman empati dan nuansa pedagogis dari penilaian manusia \cite{Usher2025GenAIAssessments}.
Anjuran penulis terhadap pendekatan hibrida menegaskan bahwa respons LLM masih membutuhkan pengawasan pakar agar relevan.
Di sisi lain, upaya standarisasi evaluasi tutor-AI menunjukkan bahwa meskipun model terkini seperti GPT-4 
kuat dalam menjawab pertanyaan, mereka cenderung terlalu 
cepat memberikan jawaban kepada mahasiswa 
dan kurang menuntun proses. Fakta ini menunjukkan bahwa LLM belum 
ideal sebagai tutor dan belum dapat menggantikan 
ahli manusia dalam penilaian kualitas pedagogis \cite{Maurya2025UnifyingAITutorEvaluation}.
%-------------------------------------------------

% Oleh karena itu, skripsi ini berfokus pada peningkatan kualitas performa respons 
% LLM sebagai pemberi umpan balik agar mendekati mutu ahli manusia secara kuantitatif 
% dan kualitatif. Peningkatan dilakukan melalui kerangka evaluasi hibrida yang menggabungkan 
% acuan ahli manusia sebagai \textit{ground truth} dan penilaian kualitas yang menilai 
% dimensi-dimensi pedagogis relevan \cite{Usher2025GenAIAssessments, Maurya2025UnifyingAITutorEvaluation}. 
% Dalam melakukan peningkatan kualitas umpan balik LLM, 
% skripsi ini menggabungkan \textit{context engineering} dengan data \textit{learning analytics} 
% untuk mengekstrak proses pembelajaran mahasiswa. Pendekatan kombinasi ini diharapkan mampu meningkatkan performa LLM 
% dalam memberikan \textit{feedback}, motivation, dan \textit{appreciation} kepada mahasiswa.
% Penelitian terdahulu telah menunjukkan potensi umpan balik berbasis LLM dalam mendukung hasil 
% belajar dan aspek afektif siswa.[]


%-------------------------------------------------
Oleh karena itu, skripsi ini berfokus pada peningkatan kualitas performa respons 
LLM sebagai pemberi umpan balik agar mendekati mutu ahli manusia secara kuantitatif 
dan kualitatif. Peningkatan dilakukan melalui kerangka evaluasi hibrida yang menggabungkan 
acuan ahli manusia sebagai \textit{ground truth} dan penilaian kualitas yang menilai 
dimensi-dimensi pedagogis relevan \cite{Usher2025GenAIAssessments, Maurya2025UnifyingAITutorEvaluation}. 
% -------------------------------------------------
Untuk mendukung proses pengukuran tersebut, skripsi ini memanfaatkan jejak proses belajar 
yang terstruktur. Data pembelajaran mahasiswa yang dipresentasikan ke dalam bentuk
Papan Kanban digital dipakai sebatas alat operasional untuk menata dan mengekstraksi
data proses (perpindahan kartu, checklist, waktu pengerjaan) yang dapat dipetakan
ke fase SRL (perencanaan–monitoring–refleksi) \cite{strickroth2022kanban}. 
Penggunaan Papan Kanban
hanya sebagai salah satu instrumen pengaya data agar evaluasi dan perbaikan kualitas 
umpan balik LLM bisa berjalan lebih terarah. Fokus kebaruan terletak pada rangkaian 
evaluasi dan perbaikan \textit{prompt} LLM yang menargetkan tiga fungsi
pedagogis utama yakni feedback kognitif, dukungan motivasional, dan dukungan apresiatif dengan
tujuan praktis yaitu menyamai kualitas ahli manusia pada metrik kuantitatif sekaligus memenuhi
standar kualitas kualitatif \cite{10.3389/fpsyg.2022.1027266, Bock14032024, Usher2025GenAIAssessments, Maurya2025UnifyingAITutorEvaluation}. 
Dengan membandingkan keluaran LLM dan pakar ahli, studi ini menilai peningkatan performa LLM agar 
dapat menjadi pertimbangan dalam penggunaan sebagai agen pedagogis pelengkap mahasiswa 
yang masif dan personal.

%-------------------------------------------------

% Oleh karena itu, studi ini menangani kesenjangan riset yang jelas dan mendesak pada irisan learning analytics, psikologi pendidikan, dan kecerdasan buatan. 
% % Pertanyaan utamanya 
% % adalah: Dapatkah LLM memberikan umpan balik terhadap proses SRL mahasiswa—sebagaimana direpresentasikan melalui data papan 
% % Kanban dengan kualitas yang sebanding dengan umpan balik dari seorang ahli manusia? 
% Untuk menangani permasalahan ini, dilakukan penelitian dengan membandingkan umpan balik yang dihasilkan oleh AI dengan ground truth dari ahli manusia. Tesis ini akan mengevaluasi 
% kelayakan umpan balik dari LLM saat disandingkan dengan ahli manusia. 
% % Untuk mendukung penelitian ini, telah dikembangkan sebelumnya sebuah platform pembelajaran berbasis Kanban yaitu
% % GAMATUTOR.ID sebagai .... 
% Temuan dari penelitian ini berpotensi memvalidasi model baru untuk dukungan akademik yang skalabel dan 
% menjadi proof-of-concept penting bagi pengembangan generasi berikutnya dari agen pedagogis cerdas yang dirancang untuk menjadikan setiap mahasiswa menjadi pembelajar yang 
% lebih efektif, mandiri, dan sepanjang hayat.



%HAPUS YANG TIDAK PERLU
%-------------------------------------------------
% \noindent\textbf{Contoh latar belakang penelitian untuk teknik elektro:} \\
% \noindent\fbox{%
% 	\parbox{\textwidth}{%
				
% 		\hspace{1cm} "Peningkatan konsumsi energi listrik yang terus menerus menyebabkan ketersediaan 
% 		sumber energi yang semakin terbatas. Sumber energi terbarukan seperti solar dan angin 
% 		menjadi solusi yang menjanjikan untuk memenuhi kebutuhan energi. Namun, kapasitas 
% 		produksi dan efisiensi dari sumber energi terbarukan masih sangat tergantung pada 
% 		kondisi cuaca dan geografis. Oleh karena itu, perlu adanya sistem penyimpanan energi 
% 		yang efisien dan dapat memastikan ketersediaan energi listrik secara kontinu. Penelitian 
% 		ini akan mengevaluasi kemampuan superkapasitor dalam menyimpan dan mengirimkan 
% 		energi secara efisien dan memastikan ketersediaan energi listrik secara kontinu."
		
% 	}%
% }

% %-------------------------------------------------	
% \vspace{5mm}
% Latar belakang ini memperkenalkan masalah ketersediaan sumber energi dan 
% peningkatan konsumsi energi listrik yang terus menerus. Ini juga memperkenalkan solusi 
% yang menjanjikan dari sumber energi terbarukan dan menjelaskan mengapa perlu adanya 
% sistem penyimpanan energi yang efisien. Latar belakang ini memberikan dasar yang kuat 
% bagi perumusan masalah dan tujuan penelitian, memastikan bahwa hasil penelitian 
% memiliki relevansi dan signifikansi bagi bidang terkait.

% %HAPUS YANG TIDAK PERLU
% %-------------------------------------------------
% \noindent\textbf{Contoh latar belakang penelitian untuk teknik biomedis:} \\
% \noindent\fbox{%
% 	\parbox{\textwidth}{%
		
% 		\hspace{1cm} "Diagnosis dan pengobatan penyakit memerlukan integrasi informasi medis yang 
% 		akurat dan terkini. Alat diagnostik tradisional seperti CT scan dan MRI memiliki resolusi 
% 		yang tinggi dan dapat mengidentifikasi masalah pada tingkat sel, tetapi sering 
% 		memerlukan banyak waktu dan biaya. Alat deteksi dini seperti tes darah dan urin memiliki 
% 		biaya rendah dan mudah digunakan, tetapi sering kurang akurat dan tidak memberikan gambaran yang jelas tentang masalah medis. Oleh karena itu, penting untuk menemukan metode baru yang memadukan keunggulan dari kedua jenis alat tersebut. Penelitian ini akan mengevaluasi kemampuan nanopartikel dalam meningkatkan akurasi dan efisiensi diagnosis medis."
		
% 	}%
% }

% %-------------------------------------------------	
% \vspace{5mm}
% Latar belakang ini memperkenalkan masalah diagnostik dan pengobatan penyakit dan mempertimbangkan pentingnya integrasi informasi medis yang akurat dan terkini. Ini 
% juga memperkenalkan kelemahan dari alat diagnostik tradisional dan deteksi dini dan menjelaskan mengapa penting untuk menemukan metode baru yang memadukan 
% keunggulan dari kedua jenis alat tersebut. Latar belakang ini memberikan dasar yang kuat bagi perumusan masalah dan tujuan penelitian, memastikan bahwa hasil penelitian 
% memiliki relevansi dan signifikansi bagi bidang terkait.

% %HAPUS YANG TIDAK PERLU
% %-------------------------------------------------
% \noindent\textbf{Contoh latar belakang penelitian untuk teknologi informasi:} \\
% \noindent\fbox{%
% 	\parbox{\textwidth}{%
		
% \hspace{1cm} "Dengan perkembangan teknologi informasi yang sangat pesat dalam beberapa tahun terakhir, penyimpanan data menjadi masalah yang semakin penting. Semakin banyak data yang diterima setiap hari, semakin penting bagi organisasi untuk memastikan bahwa data 
% mereka aman dan terlindungi. Pada saat yang sama, organisasi juga membutuhkan akses cepat dan efisien ke data mereka untuk membuat keputusan yang tepat. Teknologi 
% enkripsi kuantum baru-baru ini muncul sebagai solusi potensial untuk memenuhi kebutuhan ini, dengan menawarkan tingkat keamanan yang jauh lebih tinggi dan proses 
% enkripsi yang lebih cepat dibandingkan dengan teknologi enkripsi konvensional. Oleh karena itu, penting untuk mengevaluasi efektivitas dan keamanan teknologi enkripsi 
% kuantum dalam sistem penyimpanan data cloud."
		
% 	}%
% }

% %-------------------------------------------------	
% \vspace{5mm}
% Latar belakang ini memperkenalkan masalah penyimpanan data dan mempertimbangkan pentingnya keamanan data. Ini juga memperkenalkan teknologi enkripsi kuantum sebagai solusi potensial dan menjelaskan mengapa evaluasi teknologi ini penting bagi bidang teknologi informasi. Latar belakang ini memberikan dasar yang kuat bagi perumusan masalah dan tujuan penelitian, memastikan bahwa hasil penelitian memiliki relevansi dan signifikansi bagi bidang terkait.

\section{Rumusan Masalah}

% Berdasarkan latar belakang yang telah dipaparkan, rumusan masalah yang diangkat
% dalam peneltian tersebut meliputi beberapa poin berikut:

% \begin{enumerate}
% 	\item Bagaimana kualitas umpan balik yang dihasilkan LLM terhadap proses SRL mahasiswa yang direpresentasikan melalui data papan Kanban dibandingkan dengan umpan balik ahli manusia?
% 	\item Bagaimana format prompt yang paling optimal untuk disusun agar mampu menghasilkan kualitas umpan balik yang mendekati ahli manusia.
% \end{enumerate}

% ========== versi 1 ==========
% Berdasarkan latar belakang yang telah dipaparkan, rumusan masalah yang diangkat
% dalam peneltian tersebut adalah belum ada penelitian yang membahas bagaimana cara meningkatkan
% performa LLM secara kuantitatif dan kualitatif dalam memberikan umpan balik terhadap proses SRL 
% mahasiswa yang menargetkan tiga fungsi pedagogis yaitu \textit{feedback}, \textit{motivation} 
% dan \textit{appreciation} bila dibandingkan dengan ahli manusia. Beberapa penelitian sudah 
% menerapkan pendekatan untuk meningkatkan performa LLM tetapi dengan fokus yang berbeda seperti
% pada TutorLLM dan LLM-KT \cite{li2024tutorllm, wang2025llmktaligninglargelanguage} yang memanfaatkan data \textit{learning analytics} dipadukan dengan
% teknik \textit{Knowledge Tracing} untuk memahami tahapan belajar mahasiswa, tetapi tidak
% mengukur dampaknya terhadap kualitas umpan balik yang dihasilkan. Oleh karena itu, penelitian ini
% bertujuan untuk mengisi celah tersebut dengan mengeksplorasi cara-cara untuk meningkatkan
% performa LLM dalam memberikan umpan balik yang lebih baik dan lebih relevan bagi mahasiswa.

% ========== versi 2 ==========
Berdasarkan latar belakang yang telah dipaparkan, penelitian ini mengidentifikasi 
Secara spesifik, penelitian ini berfokus pada pengembangan umpan balik yang 
menargetkan tiga fungsi pedagogis: \textit{feedback}, \textit{motivation}, dan 
\textit{appreciation}. Rumusan masalah yang diangkat dalam penelitian ini adalah 
sebagai berikut:
\begin{enumerate}
    \item Bagaimana strategi optimal untuk meningkatkan performa LLM dalam 
	menghasilkan umpan balik yang mencakup ketiga fungsi pedagogis tersebut?
	\item Bagaimana kualitas umpan balik yang dihasilkan LLM terhadap proses SRL 
	mahasiswa dibandingkan dengan umpan balik ahli manusia?
	\item Bagaimana pengaruh besar ukuran model LLM terhadap kualitas umpan balik yang dihasilkan?
\end{enumerate}

% \begin{enumerate}
% 	\item Belum adanya penelitian yang membahas peningkatan kualitas umpan balik LLM dalam .
% 	\item Bagaimana format prompt yang paling optimal untuk disusun agar mampu menghasilkan kualitas umpan balik yang mendekati ahli manusia.
% \end{enumerate}

%-------------------------------------------------
% \noindent\fbox{%
% 	\parbox{\textwidth}{%
% \noindent\textbf{Contoh} perumusan masalah untuk \textbf{Teknik Elektro}: \\		

% \hspace{1cm} \textbf{"Bagaimana memperbaiki efisiensi penghematan energi pada sistem pencahayaan rumah tangga melalui implementasi teknologi kontrol otomatis?"} \\
	
% \hspace{1cm} Perumusan masalah ini jelas dan spesifik dan menentukan fokus penelitian pada perbaikan efisiensi penghematan energi dalam sistem pencahayaan rumah tangga dengan menggunakan teknologi kontrol otomatis. Ini juga mempertimbangkan latar belakang tentang pentingnya penghematan energi dan memberikan solusi praktis melalui implementasi teknologi. Perumusan masalah ini memberikan dasar yang kuat untuk tujuan dan hipotesis penelitian dan memastikan bahwa hasil penelitian akan berguna bagi bidang teknik elektro.
		
% 	}%
% }

% %-------------------------------------------------
% \noindent\fbox{%
% 	\parbox{\textwidth}{%
% \noindent\textbf{Contoh} perumusan masalah untuk \textbf{Teknik Biomedis}: \\		

% \hspace{1cm} \textbf{"Bagaimana memperbaiki akurasi deteksi kanker payudara dengan menggunakan teknologi pemindaian ultrasonografi berbasis AI?"} \\

% \hspace{1cm} Perumusan masalah ini jelas dan spesifik dan menentukan fokus penelitian pada 
% perbaikan akurasi deteksi kanker payudara dengan menggunakan teknologi pemindaian ultrasonografi berbasis AI. Ini mempertimbangkan latar belakang tentang pentingnya deteksi dini kanker payudara dan memberikan solusi praktis melalui implementasi teknologi. Perumusan masalah ini memberikan dasar yang kuat untuk tujuan dan hipotesis penelitian dan memastikan bahwa hasil penelitian akan berguna bagi bidang teknik biomedis.
		
% 	}%
% }

% %-------------------------------------------------	
% \noindent\fbox{%
% 	\parbox{\textwidth}{%
% \noindent\textbf{Contoh} perumusan masalah untuk \textbf{Teknologi Informasi}: \\		

% \hspace{1cm} \textbf{"Bagaimana meningkatkan efisiensi dan keamanan sistem penyimpanan data cloud melalui implementasi teknologi enkripsi kuantum?"} \\

% \hspace{1cm} Perumusan masalah ini jelas dan spesifik dan menentukan fokus penelitian pada peningkatan efisiensi dan keamanan sistem penyimpanan data cloud dengan menggunakan teknologi enkripsi kuantum. Ini mempertimbangkan latar belakang tentang pentingnya keamanan data dan memberikan solusi praktis melalui implementasi teknologi. Perumusan masalah ini memberikan dasar yang kuat untuk tujuan dan hipotesis penelitian dan memastikan bahwa hasil penelitian akan berguna bagi bidang teknologi informasi.
		
% 	}%
% }

%-------------------------------------------------	

\section{Tujuan Penelitian}

Tujuan penelitian ini adalah sebagai berikut:
\begin{enumerate}
	\item Merancang skema \textit{prompt} serta memberikan metode \textit{learning analytics} 
	yang efektif untuk meningkatkan kualitas umpan balik LLM agar mendekati ahli manusia.
	\item Mengevaluasi secara empiris apakah adanya peningkatan kualitas umpan balik 
	LLM atas proses SRL mahasiswa baik secara kuantitatif maupun kualitatif dengan umpan 
	balik ahli manusia.
	\item Melihat pengaruh besar ukuran model LLM terhadap kualitas umpan balik yang dihasilkan.
\end{enumerate}
%Secara umum, tujuan penelitian pada skripsi bidang teknik adalah untuk:
%
%\begin{itemize}
%	\item Mengidentifikasi dan menganalisis masalah atau permasalahan dalam bidang \textit{engineering}.
%	\item Meningkatkan pemahaman dan wawasan tentang bidang \textit{engineering} melalui penerapan teori dan metodologi yang sesuai.
%	\item Mengembangkan solusi atau produk baru yang inovatif dan efektif dalam 
%	bidang \textit{engineering}.
%	\item Menunjukkan penerapan prinsip-prinsip keteknikan dalam solusi atau produk yang dikembangkan.
%	\item Menjelaskan implikasi dan rekomendasi dari hasil penelitian bagi bidang \textit{engineering} dan masyarakat.
%\end{itemize}
%
%
%\noindent Catatan: Tujuan penelitian dalam skripsi bidang teknik harus jelas, spesifik, terukur, dan dapat dicapai dalam jangka waktu yang ditentukan.

% \section{Hipotesis Penelitian}
% % Revisi : Gunakan H0 dan H1 dalam menyampaikan hipotesisnya.
% % Kalo ada 2 objective, brrti perlu bikin 2 H0 dan H1.
% Hipotesis penelitian ini adalah penerapan \textit{context engineering} 
% yang dikombinasikan dengan metode dari \textit{learning analytics} %(sinyal SRL dari papan Kanban) 
% dengan dukungan \textit{prompt} yang terstruktur akan secara signifikan meningkatkan 
% kualitas kemiripan dan kelayakan respons LLM terhadap umpan balik ahli manusia
% dalam tiga fungsi pedagogis yaitu \textit{feedback}, \textit{motivation}, dan 
% \textit{appreciation} dibandingkan dengan LLM yang menggunakan metode \textit{baseline}.

% \textcolor{red}{Tujuan perlu disinkronkan dengan rumusan masalah. Setiap rumusan masalah boleh memiliki lebih dari 1 tujuan. Berikut ini adalah contoh tujuan dari rumusan masalah "Bagaimanakan memperbaiki efisiensi penghematan energi pada sistem pencahayaan rumah tangga melalui implementasi teknologi kontrol otomatis?"
% \begin{enumerate}
% 	\item Memodelkan sistem pencahayaan rumah tangga;
% 	\item Mendesain sistem kontrol otomatis untuk sistem pencahayaan rumah tangga.
% \end{enumerate}
% }

% \noindent Catatan: Tujuan penelitian dalam skripsi bidang teknik harus jelas, spesifik, terukur, dan dapat dicapai dalam jangka waktu yang ditentukan.

% Berikut ini adalah contoh detail tujuan untuk masing-masing prodi di DTETI:

% \newpage
% \vspace{5mm}
% \textbf{Contoh Tujuan Penelitian Skripsi Teknik Elektro:}

% \begin{minipage}{0.92\textwidth}
% Berikut adalah beberapa contoh tujuan penelitian yang sesuai dengan rumusan masalah 
% “perbaikan efisiensi penghematan energi pada sistem pencahayaan rumah tangga 
% melalui implementasi teknologi kontrol otomatis”:
% \end{minipage}



% %-------------------------------------------------	
% \noindent\fbox{%
% 	\parbox{\textwidth}{%
% \begin{enumerate}
% \item Menganalisis tingkat efisiensi energi pada sistem pencahayaan rumah tangga 
% sebelum dan setelah implementasi teknologi kontrol otomatis.
% \item Mengukur pengurangan biaya listrik setelah implementasi teknologi kontrol 
% otomatis pada sistem pencahayaan rumah tangga.
% \item Menunjukkan bagaimana teknologi kontrol otomatis dapat memperbaiki 
% efisiensi penghematan energi pada sistem pencahayaan rumah tangga.
% \item Meningkatkan kenyamanan dan keamanan pengguna rumah tangga melalui 
% penggunaan teknologi kontrol otomatis pada sistem pencahayaan.
% \item Menjelaskan bagaimana implementasi teknologi kontrol otomatis 
% mempengaruhi kualitas cahaya dan faktor-faktor lain dalam sistem pencahayaan 
% rumah tangga.
% \item Membandingkan efisiensi energi dan biaya pada sistem pencahayaan rumah 
% tangga dengan teknologi kontrol otomatis dan sistem manual.
% \item Menunjukkan implikasi dan rekomendasi dari hasil penelitian bagi rumah tangga 
% dan lingkungan.
% \end{enumerate}
		
% 	}%
% }

% %-------------------------------------------------	

% \newpage
% \vspace{5mm}
% \textbf{Contoh Tujuan Penelitian Skripsi Teknik Biomedis:}

% \begin{minipage}{0.92\textwidth}
% Berikut adalah beberapa contoh tujuan penelitian untuk penelitian dengan tema "Bagaimana memperbaiki akurasi deteksi kanker payudara dengan menggunakan 
% teknologi pemindaian ultrasonografi berbasis AI":
% \end{minipage}

% %-------------------------------------------------	
% \noindent\fbox{%
% 	\parbox{\textwidth}{%
% \begin{enumerate}
% \item Mengidentifikasi faktor-faktor yang mempengaruhi akurasi deteksi kanker 
% payudara dengan menggunakan teknologi pemindaian ultrasonografi berbasis 
% AI.
% \item Menilai efektivitas teknologi pemindaian ultrasonografi berbasis AI dalam 
% meningkatkan akurasi deteksi kanker payudara.
% \item Menentukan metode pemindaian ultrasonografi berbasis AI yang paling efektif dalam meningkatkan akurasi deteksi kanker payudara.
% \item Menilai keamanan dan tolerabilitas teknologi pemindaian ultrasonografi berbasis AI dalam deteksi kanker payudara.
% \item Membandingkan akurasi deteksi kanker payudara dengan teknologi pemindaian 
% ultrasonografi berbasis AI dengan metode deteksi lainnya.
% \item Menyediakan bukti ilmiah untuk menunjukkan bahwa teknologi pemindaian 
% ultrasonografi berbasis AI dapat digunakan sebagai metode deteksi kanker 
% payudara yang lebih efektif dan akurat.
% \item Meningkatkan akurasi deteksi kanker payudara dengan menggunakan teknologi 
% pemindaian ultrasonografi berbasis AI.
% \end{enumerate}
		
% 	}%
% }

% %-------------------------------------------------	

% \newpage
% \vspace{5mm}
% \textbf{Contoh Tujuan Penelitian Skripsi Teknologi Informasi:}

% \begin{minipage}{0.92\textwidth}
% Berikut adalah beberapa contoh tujuan penelitian untuk penelitian dengan tema 
% "Bagaimana meningkatkan efisiensi dan keamanan sistem penyimpanan data cloud 
% melalui implementasi teknologi enkripsi kuantum?":
% \end{minipage}

% %-------------------------------------------------	
% \noindent\fbox{%
% 	\parbox{\textwidth}{%
% \begin{enumerate}
% \item Menilai efektivitas implementasi teknologi enkripsi kuantum dalam 
% meningkatkan keamanan data pada sistem penyimpanan cloud.
% \item Menentukan metode enkripsi kuantum yang paling efektif dalam meningkatkan 
% keamanan data pada sistem penyimpanan cloud.
% \item Membandingkan efisiensi enkripsi kuantum dengan metode enkripsi lainnya 
% dalam meningkatkan keamanan data pada sistem penyimpanan cloud.
% \item Menilai keamanan dan stabilitas sistem penyimpanan data cloud setelah 
% implementasi teknologi enkripsi kuantum.
% \item Menyediakan bukti ilmiah untuk menunjukkan bahwa implementasi teknologi 
% enkripsi kuantum dapat meningkatkan efisiensi dan keamanan sistem 
% penyimpanan data cloud.
% \item Mengidentifikasi potensi masalah dan hambatan dalam implementasi teknologi 
% enkripsi kuantum pada sistem penyimpanan data cloud.
% \end{enumerate}
		
% 	}%
% }

%-------------------------------------------------	


\section{Batasan Penelitian}
Penelitian ini berfokus untuk melihat peningkatan performa dari LLM ketika diberikan perlakuan 
berupa metode \textit{context engineering} yang memanfaatkan data \textit{learning analytics}. 
Upaya untuk menyederhanakan proses ini dilakukan dengan cara memberikan batasan penelitian yang 
jelas dan ditetapkan oleh peneliti dengan mempertimbangkan aspek-aspek yang relevan. Berikut
adalah batasan-batasan tersebut:

% % Keterbatasan penelitian itu adalah kelemahan yang belum dikerjakan. (Misal belum nyoba API yg lain. Data Kanban juga masuk keterbatasan karena data sintetis).
% Penelitian ini berfokus untuk melihat peningkatan performa dari LLM ketika diberikan perlakuan 
% berupa metode \textit{context engineering} yang memanfaatkan data \textit{learning analytics}. 
% Upaya untuk menyederhanakan proses ini dilakukan dengan cara memberikan batasan penelitian yang 
% jelas dan ditetapkan oleh peneliti dengan mempertimbangkan aspek-aspek yang relevan. 
% Batasan-batasan tersebut adalah sebagai berikut:

% Batasan penelitian dapat ditulis dengan format \textit{list}. Contohnya adalah sebagai berikut:

% \begin{enumerate}
% 	\item Data Pembelajaran Mahasiswa berbasis Papan Kanban yang digunakan merupakan data sintetis 
% 	yang dibuat hanya untuk keperluan pengambilan data dari para ahli manusia dan LLM. Data sintetis 
% 	digunakan karena waktu yang terbatas untuk mengambil data yang ril. Selain itu, fokus penelitian 
% 	ini adalah pada respons dari LLM dan
% 	ahli manusia sehingga penggunaan data sintetis dianggap valid untuk tujuan penelitian ini.
% 	\item Penelitian ini hanya menggunakan model yang sudah ada dan bersifat \textit{open-source}
% 	untuk menghemat waktu dan pengeluaran.
% 	\item Penelitian ini menganalisis kemampuan model LLM dalam bahasa Indonesia. Oleh karena itu,
% 	semua data yang digunakan untuk penelitian ini dalam bahasa Indonesia.
% 	\item Sumber data utama yang digunakan berasal respons LLM dan ahli manusia sebagai data yang akan
% 	dianalisis dan data pembelajaran berbasis Kanban sebagai data pendukung untuk pembuatan respons.
% \end{enumerate}

\subsection{Objek Penelitian}
\begin{enumerate}
  \item[(a)] Ruang lingkup mencakup analisis kemampuan model bahasa besar pada tugas 
  berbahasa Indonesia dengan konteks pembelajaran berbasis papan Kanban.
  \item[(b)] Data pembelajaran yang digunakan berupa data sintetis yang disusun untuk 
  memicu respons dari model dan dari pakar manusia. Pemakaian data sintetis dinilai m
  emadai karena keterbatasan waktu pengambilan data ril serta karena fokus utama 
  penelitian adalah perbandingan kualitas respons model dan respons pakar.
  \item[(c)] Model yang diteliti merupakan model yang tersedia secara \textit{open source} 
  dan diakses melalui Groq API. Dua model yang digunakan adalah 
  Llama 3.1 8B Instant 128k dan Llama 3.3 70B Versatile 128k.
\end{enumerate}
% Penelitian ini menganalisis kemampuan model bahasa besar pada tugas berbahasa Indonesia 
% dengan konteks pembelajaran berbasis papan Kanban. Data pembelajaran yang digunakan berupa 
% data sintetis yang disusun untuk memicu respons dari model dan dari pakar manusia. 
% Fokus utama adalah membandingkan kualitas respons model dan respons pakar manusia sehingga 
% pemakaian data sintetis dinilai memadai untuk tujuan penelitian ini. Model yang diteliti 
% merupakan model yang tersedia secara \textit{open source} dan diakses melalui Groq API. 
% Dua model yang digunakan adalah Llama 3.1 8B Instant 128k dan Llama 3.3 70B Versatile 128k.

\subsection{Metode Penelitian}
Penelitian menggunakan rancangan eksperimen komparatif tanpa pelatihan ulang pada model. 
Seluruh prosedur mengandalkan rekayasa konteks yang mencakup perancangan templat 
\textit{prompt} penetapan instruksi dan penyediaan konteks dari data Kanban berbahasa 
Indonesia. Langkah penelitian meliputi penyusunan skenario soal dan konteks berbasis 
Kanban pembuatan respons oleh kedua model melalui Groq API pengumpulan respons pakar 
manusia pada skenario yang sama serta evaluasi kuantitatif dan penilaian kualitatif oleh 
pakar. Reproduksibilitas dijaga melalui pencatatan parameter \textit{decoding} 
penetapan \textit{seed} serta pelaporan lingkungan komputasi.

\subsection{Waktu dan Tempat Penelitian}
Penelitian dilaksanakan pada Juni 2025 hingga Oktober 2025. Proses perancangan dan 
analisis dilakukan dengan memanfaatkan layanan Groq untuk inferensi model.

\subsection{Populasi dan Sampel}
Populasi penelitian mencakup seluruh respons atas skenario berbahasa Indonesia yang 
relevan dengan konteks pembelajaran berbasis Kanban. Sampel terdiri atas sekumpulan 
skenario tugas yang dipilih secara purposif beserta pasangan respons model dan respons 
pakar manusia. Jumlah \textit{rater} pakar jumlah skenario serta pembagian data 
ditetapkan agar memadai untuk analisis kuantitatif dan penilaian kualitatif.

\subsection{Variabel}
\begin{enumerate}
  \item[(a)] \textbf{Variabel bebas} mencakup jenis model Llama 3.1 8B Instant 128k 
  dan Llama 3.3 70B Versatile 128k serta strategi rekayasa konteks yang digunakan 
  pada \textit{prompt} dan penyajian \textit{learning analytics} berbasis Kanban
  .
  \item[(b)] \textbf{Variabel terikat} adalah kualitas keluaran yang diukur dengan 
  metrik kuantitatif seperti \textit{accuracy} \textit{precision} \textit{recall} 
  \textit{macro F1} serta ukuran kesesuaian teks yang relevan.
  \item[(c)] \textbf{Variabel kendali} meliputi penggunaan bahasa Indonesia pada 
  seluruh data karakteristik skenario Kanban, parameter seperti 
  \textit{temperature} dan \textit{top-$p$} \textit{random seed} serta prosedur 
  evaluasi yang seragam.
\end{enumerate}

% \subsection{Hipotesis}
% H1. Model Llama 3.3 70B Versatile 128k memberikan skor kuantitatif yang lebih 
% tinggi dibandingkan Llama 3.1 8B Instant 128k pada skenario yang sama

% H2. Rekayasa konteks yang lebih kaya melalui templat \textit{prompt} dan 
% penyajian konteks Kanban menghasilkan skor kuantitatif yang lebih baik dibandingkan 
% templat dasar

% H3. Peningkatan skor kuantitatif selaras dengan preferensi penilaian pakar 
% pada uji kualitatif

\subsection{Keterbatasan Penelitian}
\begin{enumerate}
  \item Data pembelajaran berbasis papan Kanban yang digunakan merupakan data 
  sintetis yang disusun untuk keperluan pengambilan data dari pakar manusia dan 
  model. Data sintetis dipilih karena keterbatasan waktu pengambilan data ril 
  serta karena fokus penelitian berada pada perbandingan respons.
  \item Penelitian hanya menggunakan model yang telah tersedia dan bersifat 
  \textit{open source} guna efisiensi waktu dan biaya.
  \item Model diakses melalui Groq API sebagai sarana inferensi.
  \item Model yang digunakan terbatas pada dua model yaitu Llama 3.1 8B Instant 128k 
  dan Llama 3.3 70B Versatile 128k.
  \item Tidak dilakukan pelatihan ulang pada model sehingga seluruh prosedur bersifat 
  rekayasa konteks.
  \item Seluruh kegiatan dilakukan pada Juni 2025 hingga Oktober 2025.
  \item Analisis difokuskan pada kemampuan berbahasa Indonesia sehingga seluruh data 
  yang digunakan berbahasa Indonesia.
  \item Sumber data utama berasal dari keluaran model dan respons pakar manusia yang 
  dianalisis sedangkan data pembelajaran Kanban berfungsi sebagai pendukung penyusunan 
  konteks.
  \item Luaran penelitian berfokus pada identifikasi metode optimisasi yang paling 
  baik berdasarkan pengukuran kuantitatif yang selanjutnya diuji kepada pakar manusia 
  untuk penilaian kualitatif.
\end{enumerate}

% \subsection{Objek Penelitian}
% \begin{enumerate}
%   \item[(a)] 
% \end{enumerate}
% \subsection{Metode Penelitian}
% \subsection{Waktu dan Tempat Penelitian}
% \subsection{Populasi dan Sampel}
% \subsection{Variabel}
% \subsection{Keterbatasan Penelitian}



% \noindent \textbf{Contoh penulisan batasan skripsi Teknik Elektro:}

% %-------------------------------------------------	
% \noindent\fbox{
% 	\parbox{\textwidth}{
% \begin{enumerate}
% \item Objek penelitian: Studi efisiensi dan kinerja sistem pemantauan suhu dan arus pada sistem pembangkit tenaga listrik.
% \item Metode penelitian: Penelitian eksperimental menggunakan analisis simulasi dan pengujian sistem pada skala laboratorium.
% \item Waktu dan tempat penelitian: Waktu penelitian adalah Maret-Agustus 2022 di 
% FasLab TTL.
% \item Populasi dan sampel: Populasi adalah sistem pembangkit tenaga listrik, dan sampel diambil sebanyak 3 sistem yang berbeda.
% \item Variabel: Variabel bebas adalah metode pemantauan suhu dan arus, dan variabel terikat adalah efisiensi dan kinerja
% \item Hipotesis: bahwa metode pemantauan suhu dan arus berpengaruh terhadap efisiensi dan kinerja sistem pembangkit tenaga listrik.
% \item Keterbatasan Penelitian: Keterbatasan penelitian adalah hanya melibatkan 
% pengujian sistem pada skala laboratorium dan membatasi analisis pada variabel 
% bebas dan terikat.
% \end{enumerate}

% }
% }
%
%%-------------------------------------------------	
%\newpage
%
%\noindent \textbf{Contoh penulisan batasan skripsi Teknik Biomedis:}
%
%%-------------------------------------------------	
%\noindent\fbox{%
%	\parbox{\textwidth}{%
%\begin{enumerate}
%\item Objek Penelitian: Studi efektivitas dan keamanan alat elektromedik seperti pacu jantung.
%\item Metode Penelitian: Penelitian deskriptif dan observasional menggunakan survei dan analisis data.
%\item Waktu dan Tempat Penelitian: Waktu penelitian adalah Januari-Juni 2022 di 
%Rumah Sakit Sarjito. 
%\item Populasi dan Sampel: Populasi adalah pasien yang menggunakan pacu jantung, dan sampel diambil dengan metode random sampling sebanyak 100 pasien. 
%\item Variabel: Variabel bebas nya adalah jenis alat pacu jantung, dan variabel terikatnya adalah efektivitas dan keamanan. 
%\item Hipotesis: bahwa jenis alat pacu jantung berpengaruh terhadap efektivitas dan keamanan. 
%\item Keterbatasan Penelitian: Keterbatasan penelitian adalah hanya melibatkan satu rumah sakit sebagai lokasi penelitian dan membatasi dalam analisis data hanya pada variabel bebas dan terikat.
%\end{enumerate}
%		
%	}%
%}
%
%%-------------------------------------------------	
%
%\newpage
%\noindent \textbf{Contoh penulisan batasan skripsi Teknologi Informasi:}
%
%%-------------------------------------------------	
%\noindent\fbox{%
%	\parbox{\textwidth}{%
%\begin{enumerate}
%\item Objek Penelitian: Analisis perbandingan efektivitas dan efisiensi antara sistem manajemen proyek tradisional dan sistem manajemen proyek berbasis teknologi informasi. 
%\item Metode Penelitian: Penelitian kualitatif dengan menggunakan wawancara dan 
%survei terhadap para pelaku proyek di berbagai perusahaan. 
%\item Waktu dan Tempat Penelitian: Waktu penelitian adalah Januari-Juni 2022 di 
%perusahaan-perusahaan di wilayah Bantul. 
%\item Populasi dan Sampel: Populasi nya adalah perusahaan yang melakukan proyek, dan sampel diambil sebanyak 10 perusahaan yang menerapkan sistem manajemen 
%proyek tradisional dan 10 perusahaan yang menggunakan sistem manajemen 
%proyek berbasis teknologi informasi. 
%\item Variabel: Variabel bebasnya adalah sistem manajemen proyek, dan variabel 
%terikatnya adalah efektivitas dan efisiensi. 
%\item Hipotesis: bahwa sistem manajemen proyek berbasis teknologi informasi lebih efektif dan efisien dibandingkan dengan sistem manajemen proyek tradisional.
%\item Keterbatasan Penelitian: Keterbatasan penelitian adalah penelitian hanya dilakukan pada perusahaan di wilayah Bantul dan hanya melibatkan wawancara dan survei sebagai metode pengumpulan data.
%\end{enumerate}
%		
%	}%
%}

\section{Manfaat Penelitian}

Penelitian ini menawarkan wawasan bagaimana penerapan LLM sebagai pendamping dalam proses SRL 
bagi mahasiswa bisa menjadi alternatif yang efektif dalam meningkatkan kualitas pembelajaran. 
Selain itu, penelitian ini juga dapat menjadi dasar untuk penelitian lebih lanjut untuk 
peningkatan performa respons LLM terutama pada bidang pedagogis sehingga seiring berjalannya 
waktu dapat memberikan hasil yang lebih optimal.
% \begin{enumerate}
% 	\item Enum 1
% 	\begin{itemize}
% 		\item Coba 1
% 		\item Coba 2
% 	\end{itemize}
% 	\item Enum 2
% \end{enumerate}



\section{Sistematika Penulisan}

Sistematika penulisan berisi pembahasan apa yang akan ditulis di setiap bab. 
Sistematika pada umumnya berupa paragraf yang setiap paragraf mencerminkan 
bahasan setiap Bab. Contoh:

\noindent Bab I membahas tentang pendahuluan yang berisi latar belakang, perumusan masalah 
dan tujuan penelitian. 

\noindent Bab II berisi tentang metodologi penelitian yang terdiri dari desain penelitian, sumber data, Teknik pengumpulan data dan Teknik analisis data.

\noindent Dan seterusnya.

