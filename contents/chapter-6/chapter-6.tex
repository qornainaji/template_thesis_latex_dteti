\chapter{Kesimpulan dan Saran}

\section{Kesimpulan}

Kesimpulan dapat diawali dengan apa yang dilakukan dengan tugas akhir ini lalu 
dilanjutkan dengan poin-poin yang menjawab tujuan penelitian, apakah tujuan sudah tercapai atau belum, tentunya berdasarkan data ataupun hasil dari Bab pembahasan sebelumnya. Dalam beberapa hal, kesimpulan dapat juga berisi tentang temuan/\textit{findings} yang Anda dapatkan setelah melakukan pengamatan dan atau analisis terhadap hasil penelitian. 

Kesimpulan menjawab seberapa jauh rumusan masalah tercapai berdasarkan hasil penelitian. Semua rumusan masalah harus disimpulkan berdasarkan data penelitian.

\section{Saran}

Saran berisi hal-hal yang bisa dilanjutkan dari penelitian atau skripsi ini, yang belum dilakukan karena batasan permasalahan. Saran bukan berisi saran kepada sistem atau pengguna, tetapi saran diberikan kepada aspek penelitian yang dapat dikembangkan dan ditambahkan di penelitian atau skripsi selanjutnya.
% Apa yang akan di kerjakan di masa depan, 
% jangan keluar dari scope penelitian yang sudah dilakukan.
% Membuat saran untuk penelitian selanjutnya,

\noindent \textcolor{red}{Catatan: Mahasiswa perlu melihat sinkronisasi antara rumusan masalah, tujuan, metode, hasil penelitian, dan kesimpulan.}
