\chapter{Kesimpulan dan Saran}

\section{Kesimpulan}
Penelitian ini telah dilakukan eksperimen untuk 
merancang, meningkatkan, dan mengevaluasi kualitas 
umpan balik pedagogis yang dihasilkan oleh 
LLM. Fokus utama penelitian adalah pada 
peningkatan tiga fungsi pedagogis yaitu \textit{feedback}, 
\textit{motivation}, dan \textit{appreciation} yang untuk 
mendukung proses SRL mahasiswa. Peningkatan ini dicapai 
melalui metode \textit{context engineering} yang 
diperkaya dengan \textit{learning analytics} (LA) 
yang diekstraksi dari data sintetis pembelajaran Mahasiswa
berbasis papan Kanban. Eksperimen dilakukan, 
tanpa melakukan pelatihan ulang (\textit{fine-tuning}) 
pada model. Empat variasi \textit{prompt} 
(\textit{Baseline}, Lis Materi, LA, dan ReAct + LA) diujikan 
pada dua model LLM dengan ukuran berbeda 
(Llama 3.1 8B dan Llama 3.3 70B). Evaluasi dilakukan 
secara kuantitatif menggunakan metrik BERTScore, 
BARTScore, dan \textit{LLM-as-a-Judge} terhadap 
data umpan balik ahli (\textit{ground truth}) , 
serta secara kualitatif melalui validasi pakar psikologi pendidikan. 
Berdasarkan hasil analisis dan pembahasan pada Bab 4, berikut adalah 
kesimpulan yang ditarik untuk menjawab tiga rumusan 
masalah penelitian:
\begin{enumerate} 
    \item \textbf{(Menjawab RQ 1)} Metode \textit{context engineering} 
    yang digabungkan dengan \textit{learning analytics} (LA) 
    terbukti \textbf{berhasil meningkatkan kualitas umpan balik LLM} 
    dibandingkan dengan \textit{Baseline}, namun efektivitasnya 
    sangat bergantung pada ukuran model. Pada model kecil (Llama 3.1 8B), 
    penambahan konteks yang kompleks dan metode ReAct dan 
    \textit{learning analytics}
    ("ReAct + LA") secara konsisten memberikan peningkatan performa 
    tertinggi di ketiga metrik evaluasi kuantitatif. Hal ini 
    menunjukkan bahwa model kecil sangat diuntungkan oleh panduan 
    konteks yang kaya dan terstruktur untuk menghasilkan umpan 
    balik yang lebih relevan dan berkualitas secara semantik.
    % Selain itu, ditemukan juga bahwa model kecil 
    % sangat sensitif terhadap variasi konteks,
    % di mana konteks yang sederhana ("Lis Materi")
    % memberikan peningkatan yang minimal, sedangkan konteks yang 
    % lebih terstruktur dan informatif ("LA" dan "ReAct + LA")
    % memberikan peningkatan yang signifikan. Sebaliknya, pada model besar 
    % (Llama 3.3 70B), penambahan konteks yang kompleks
    % tidak selalu menghasilkan peningkatan performa. Justru, konteks yang sederhana 
    % ("Lis Materi") sudah cukup untuk memaksimalkan performa model besar.
    \item \textbf{(Menjawab RQ 3)} Ukuran model \textbf{memiliki 
    pengaruh yang signifikan dan bersifat dua arah} terhadap kualitas 
    umpan balik. Model besar (Llama 3.3 70B) menunjukkan performa 
    \textit{Baseline} yang jauh lebih superior dibandingkan model 8B, 
    terutama dalam hal relevansi (skor F1 \textit{LLM-as-a-Judge} 0,7271 
    vs 0,5371). Namun, model 70B tidak 
    diuntungkan oleh konteks yang kompleks di mana performa terbaiknya 
    justru dicapai dengan konteks sederhana ("Lis Materi"). 
    Penambahan konteks "ReAct + LA" yang kompleks 
    bahkan menurunkan kinerjanya, diduga karena \textit{prompt} 
    tersebut terlalu membatasi dan memaksa model besar meniru gaya 
    deskriptif model kecil, alih-alih mendekati jawaban dari \textit{ground truth}.
    % \item \textbf{(Menjawab RQ 2)} Berdasarkan penilaian ahli 
    % manusia, umpan balik terbaik yang dihasilkan oleh 
    % Llama 3.1 8B dengan ReAct + LA dinilai 
    % \textbf{``Dapat digunakan dengan revisi minor''}. 
    % Pakar melihat potensi \textit{chatbot} sebagai alat 
    % \textit{scaffolding} yang mudah diakses, dan 
    % merekomendasikan revisi minor yang fokus pada tiga 
    % kegagalan eksekusi model, yakni umpan balik yang 
    % kurang spesifik karena tidak menyebut nama kartu yang 
    % dimaksud sehingga relevansi menurun, motivasi yang 
    % terasa monoton dan robotik sehingga keterhubungan 
    % berkurang, serta apresiasi yang tercampur dengan kritik 
    % sehingga penguatan positif menjadi kurang efektif.
    \item \textbf{(Menjawab RQ 2)} Berdasarkan persepsi ahli manusia,
    umpan balik yang dihasilkan oleh model terbaik (Llama 3.1 8B 
    dengan "ReAct + LA") dinyatakan \textbf{"Dapat digunakan 
    dengan revisi minor"}. Pakar mengonfirmasi 
    potensi \textit{chatbot} sebagai sistem \textit{scaffolding} 
    yang aksesibel. Revisi minor yang disarankan 
    berfokus pada penyesuaian tonalitas bahasa, eliminasi diksi negatif,
    tumpang tindihnya kategori respons, kurangnya spesifitas feedback,
    dan minimnya variasi respons pada motivasi.
    % berfokus pada tiga kegagalan eksekusi model, yang ternyata 
    % telah diinstruksikan dalam \textit{prompt} seperti pada 
    % \textit{feedback} kurang spesifik karena gagal 
    % menyebutkan nama kartu yang dibahas sehingga menyebabkan indikator 
    % F3 (Relevansi) 
    % dinilai "Netral". Lalu pada \textit{motivation} yang 
    % terlalu monoton dan robotik yang menyebabkan 
    % indikator M5 (Keterhubungan) dinilai "Netral".  Terakhir pada 
    % \textit{Appreciation} yang terkontaminasi oleh kritik 
    % ("pujian dengan 'tetapi'"), yang membatalkan penguatan positif 
    % sehingga menyebabkan indikator A2 (Pujian Proporsional) dan 
    % A5 (Penguatan) dinilai "Netral". 
    Perlu dicatat bahwa 
    beberapa instruksi dari \textit{prompt} juga turut andil
    dalam beberapa revisi minor yang disarankan. Hal ini juga
    menegaskan pada penilaian kualitas \textit{prompt} yang digunakan
    berdasarkan penelitian dari \cite{jacobsen2025promises} yaitu 
    \textit{theory-driven prompt manual},
    di mana nilai yang baik dari sebuah \textit{prompt}
    belum tentu menjamin keluaran yang optimal dari model.
    Faktor-faktor lain seperti ukuran model,
    kapabilitas bahasa, dan kompleksitas konteks juga berperan 
    penting dalam menentukan kualitas keluaran. Hal tersebut 
    perlu diteliti lebih lanjut pada penelitian selanjutnya.
\end{enumerate}

Kesimpulan yang dihasilkan, penelitian ini menyimpulkan bahwa 
\textit{context engineering} yang diperkaya 
\textit{learning analytics} adalah strategi yang efektif, 
praktis, dan efisien secara komputasi untuk meningkatkan 
umpan balik pedagogis, terutama pada model LLM berukuran kecil.

% Kesimpulan dapat diawali dengan apa yang dilakukan dengan tugas akhir ini lalu 
% dilanjutkan dengan poin-poin yang menjawab tujuan penelitian, apakah tujuan sudah tercapai atau belum, tentunya berdasarkan data ataupun hasil dari Bab pembahasan sebelumnya. Dalam beberapa hal, kesimpulan dapat juga berisi tentang temuan/\textit{findings} yang Anda dapatkan setelah melakukan pengamatan dan atau analisis terhadap hasil penelitian. 

% Kesimpulan menjawab seberapa jauh rumusan masalah tercapai berdasarkan hasil penelitian. Semua rumusan masalah harus disimpulkan berdasarkan data penelitian.

\section{Saran}
Dengan melihat berbagai kekurangan yang ada serta
kemungkinan potensi pengembangan yang bisa dilakukan
pada penelitian kali ini, penelitian
selanjutnya dapat dipertimbangkan berbagai pengembangan sebagai berikut.
\begin{enumerate}
    \item Meneliti lebih lanjut mengenai pengaruh 
    \textit{context engineering} yang kompleks pada LLM
    berukuran besar, untuk memahami batasan dan
    karakteristiknya dalam merespons konteks yang kaya.
    \item Melakukan eksperimen dengan variasi
    \textit{prompt} yang lebih beragam, termasuk
    variasi yang menyesuaikan tonalitas bahasa
    sesuai karakteristik demografis pengguna.
    \item Meneliti lebih lanjut \textit{prompt}
    yang optimal untuk menghasilkan umpan balik
    yang berkualitas, tidak hanya berdasarkan 
    penilaian dari \textit{theory-driven prompt manual}.
    \item Melakukan uji hipotesis statistik
    untuk memperkuat validitas temuan kuantitatif, serta
    mengembangkan arah penelitian yang sekarang berfokus
    pada eksploratif menjadi hipotesis yang kuat.
    \item Melakukan evaluasi terhadap LLM yang memiliki
    performa bahasa Indonesia yang lebih baik, untuk
    melihat dampaknya terhadap kualitas umpan balik.
    \item Menggunakan data pembelajaran nyata dari
    mahasiswa untuk mengekstraksi \textit{learning analytics},
    serta jumlah ahli yang lebih banyak untuk
    validasi pakar dan pengambilan \textit{ground truth}, guna 
    meningkatkan keakuratan dan relevansi hasil penelitian.
    % \item Membandingkan dengan model komersil terbaru
    % seperti GPT-5 untuk melihat bagaimana
    % teknik \textit{context engineering} dan \textit{learning analytics}
    % berperan pada model dengan kemampuan yang lebih
    % maju.
    
\end{enumerate}
% Saran berisi hal-hal yang bisa dilanjutkan dari penelitian atau skripsi ini, yang belum dilakukan karena batasan permasalahan. Saran bukan berisi saran kepada sistem atau pengguna, tetapi saran diberikan kepada aspek penelitian yang dapat dikembangkan dan ditambahkan di penelitian atau skripsi selanjutnya.
% % Apa yang akan di kerjakan di masa depan, 
% % jangan keluar dari scope penelitian yang sudah dilakukan.
% % Membuat saran untuk penelitian selanjutnya,

% \noindent \textcolor{red}{Catatan: Mahasiswa perlu melihat sinkronisasi antara rumusan masalah, tujuan, metode, hasil penelitian, dan kesimpulan.}
