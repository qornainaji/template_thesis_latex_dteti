Penerapan \textit{Self-Regulated Learning} (SRL) di pendidikan 
tinggi menghadapi tantangan dalam hal skalabilitas 
pemberian umpan balik personal yang berkualitas oleh 
pengajar. Penelitian ini bertujuan untuk meningkatkan 
kualitas umpan balik pedagogis yang dihasilkan oleh 
\textit{Large Language Model} (LLM) pada lingkungan pembelajaran 
berbasis papan Kanban dengan memanfaatkan pendekatan 
\textit{Context Engineering} yang diperkaya oleh \textit{Learning Analytics} (LA).

Penelitian ini menggunakan metode eksperimen dengan data 
sintetis aktivitas mahasiswa pada platform Kanban untuk 
mata kuliah Pemrograman Berorientasi Objek. Dua model LLM, 
yaitu Llama 3.1 8B Instant dan Llama 3.3 70B Versatile, 
diuji menggunakan empat variasi \textit{prompt} yaitu \textit{Baseline}, Lis Materi, 
LA, dan kombinasi ReAct + LA. Kualitas umpan balik dievaluasi 
secara kuantitatif menggunakan metrik BERTScore, BARTScore, dan 
LLM-as-a-Judge, serta secara kualitatif melalui validasi ahli 
psikologi pendidikan.

Hasil penelitian menunjukkan bahwa integrasi \textit{Context Engineering} 
dengan \textit{Learning Analytics} mampu meningkatkan kualitas umpan balik, 
namun efektivitasnya bergantung pada ukuran model. Pada model kecil 
(8B), penggunaan metode kompleks (ReAct + LA) memberikan 
peningkatan performa tertinggi secara konsisten. Sebaliknya, 
model besar (70B) memiliki performa dasar (\textit{Baseline}) yang 
lebih superior namun kinerjanya menurun ketika diberikan 
konteks yang terlalu kompleks, sehingga lebih optimal dengan 
konteks sederhana. Validasi ahli menyatakan bahwa umpan balik 
yang dihasilkan model terbaik dapat digunakan dengan revisi 
minor, terutama terkait penyesuaian tonalitas bahasa dan 
spesifisitas saran. Penelitian ini menyimpulkan bahwa rekayasa 
konteks berbasis data analitik merupakan strategi yang efektif 
dan efisien untuk mendukung SRL mahasiswa.

\noindent{Kata kunci} : Model bahasa besar, \textit{self-regulated learning}, \textit{learning analytics}, \textit{context engineering}, papan \textit{Kanban}.
