\chapter{Metode Penelitian}

Bab ini menjelaskan metode atau cara yang digunakan dalam penelitian ini untuk 
mencapai maksud dan tujuan seperti yang tertulis dalam sub-bab 1.3 yaitu:
\begin{enumerate}
	\item Merancang skema \textit{prompt} serta memberikan metode \textit{learning analytics} 
	yang efektif untuk meningkatkan kualitas umpan balik LLM agar mendekati ahli manusia.
	% \item Mengevaluasi secara empiris apakah adanya peningkatan kualitas umpan balik 
	% LLM atas proses SRL mahasiswa baik secara kuantitatif maupun kualitatif dengan umpan 
	% balik ahli manusia.
	\item Mengevaluasi apakah adanya peningkatan kualitas umpan balik 
	LLM atas proses SRL mahasiswa baik secara kuantitatif maupun kualitatif dengan umpan 
	balik ahli manusia.
	\item Melihat pengaruh besar ukuran model LLM terhadap kualitas umpan balik yang dihasilkan.
\end{enumerate}
% [jika diinginkan, kalian dapat menuliskan Kembali tujuan penelitian yang ingin dicapai di sini].

\section{Alat dan Bahan Tugas Akhir}

\subsection{Alat Tugas Akhir}

Alat-alat yang digunakan pada tugas akhir ini berupa perangkat keras maupun 
perangkat lunak sebagai sarana pendukung. Alat-alat perangkat keras yang digunakan
merupakan milik pribadi. Fungsi dari perangkat keras dan lunak ini adalah untuk 
penulisan kode, pengembangan chatbot, melakukan pengujian baik secara kuantitatif
maupun kualitatif, serta untuk pengumpulan dataset.Berikut adalah daftar 
alat-alat tugas akhir yang digunakan pada penelitian ini.

% \begin{enumerate}
% 	\item \textit{Notebook} Apple MacBook Pro 14 inci (M1 Pro, 2021), SoC Apple M1 Pro 8-core CPU dan 14-core GPU dengan 16-core Neural Engine, memori terpadu 16 GB, SSD 512 GB. 
% 	\item Visual Studio Code Version: 1.105.1.
% 	\item Google Docs
% 	\item Google Sheets
% 	\item Groq API 
	
\begin{enumerate}
  \item \textbf{Macbook Pro 14 inci.}
  Spesifikasi utama:
  \begin{itemize}
    \item Prosesor: M1 Pro 10 Core @ 3{,}2~GHz
    \item GPU internal: M1 Pro GPU 16 Core
    \item Neural Engine: 32 Core
    \item RAM: \textit{Unified Memory} 16~GB 200GB/s \textit{memory bandwidth}
    \item Penyimpanan: \textit{Solid State Drive} internal 1~TB (MacOS 26 Tahoe)
  \end{itemize}

  \item \textbf{Visual Studio Code} versi 1.105.1 sebagai \textit{integrated development 
  environment} (IDE) untuk menulis dan mengelola kode.

  \item \textbf{Python 3.12.7} sebagai bahasa pemrograman utama untuk 
  mengembangkan \textit{chatbot} berbasis LLM, dengan beberapa pustaka inti:
  \begin{itemize}
    \item \texttt{LangChain}: kerangka kerja aplikasi LLM yang menyediakan 
	komponen modular untuk orkestrasi \textit{prompt}, alur data, dan 
	integrasi sumber data eksternal.
    \item \texttt{LangGraph}: ekstensi \texttt{LangChain} untuk membangun 
	alur aplikasi yang lebih kompleks dalam bentuk graf.
	\item \texttt{bert\_score}: sebagai salah satu kerangka kerja evaluasi kuantitatif 
	untuk mengukur kesamaan semantik (Precision, Recall, F1) antara respons yang 
	dihasilkan model dengan data ground truth.
	\item \texttt{BARTScorer} sebagai kerangka kerja evaluasi kuantitatif alternatif 
	(menggunakan mBART) untuk menghitung skor F1, Precision, dan Recall 
	berdasarkan log-probability token.
	\item \texttt{Pandas} sebagai library yang digunakan untuk menyusun, mengagregasi, dan menampilkan tabel ringkasan hasil evaluasi kuantitatif (BARTScore dan GPT-4o Judge)  agar mudah dibaca.


	% \item \texttt{LangSmith}: pemantau dan \textit{tracing} interaksi LLM yang membantu proses \textit{debugging} dan evaluasi.
    % \item \texttt{CyVer}: pustaka untuk validasi \textit{Cypher query} (sintaksis, skema, dan properti) yang dihasilkan LLM saat mengambil data dari Neo4J KG.
    % \item \texttt{Streamlit}: kerangka kerja untuk membangun antarmuka pengguna (\textit{user interface}) \textit{chatbot}.
  \end{itemize}
  	\item \textbf{Google Docs}: sebagai alat untuk menulis kuesioner, dan melakukan 
	revisi serta dokumentasi hasil diskusi.
	\item \textbf{Google Sheets}: digunakan untuk memvisualisasikan dan mengelola 
	data Kanban sehingga dapat dibagikan kepada para ahli yang akan mengisi 
	dan menilai data, memudahkan kolaborasi, pengisian respons, serta 
	pengumpulan metrik penilaian.

  \item \textbf{Anaconda} untuk manajemen pustaka Python dan 
  \textit{virtual environment} selama proses pengembangan.

  \item \textbf{ChatGPT 5 Thinking} untuk membuat data sintetis papan Kanban yang 
  akan digunakan dalam penelitian ini.

%   \item \textbf{Neo4J Knowledge Graph} sebagai basis data untuk 
%   menyimpan data Kurikulum 2021 Program Studi Teknologi Informasi, 
%   Fakultas Teknik, Universitas Gadjah Mada.

  \item \textbf{Groq API} untuk melakukan inferensi LLM 
  (pemanggilan model) yang digunakan dalam penelitian ini.
\end{enumerate}


\subsection{Bahan Tugas Akhir}

Bahan yang digunakan dalam proses pembuatan chatbot dan untuk uji coba pada
penelitian ini adalah sebagai berikut:

\begin{enumerate}
	\item LLM dengan model \textbf{Llama 3.1 8B Instant}, dengan parameter
	delapan miliar, yang diperoleh dari platform Groq API.
	\item LLM dengan model \textbf{Llama 3.3 70B Versatile}, dengan parameter
	lebih besar yaitu 70 miliar yang juga diperoleh dari platform Groq API.
	\item \textbf{GPT-4o} sebagai model yang digunakan untuk evaluasi kuantitatif
	\textit{LLM-as-a-judge}
	\item \textbf{Dataset sintetis} dengan format data \texttt{.json} yang 
	di-\textit{generate} dari ChatGPT, berisikan pasangan data 
	pembelajaran mahasiswa berbasis papan Kanban.
	\item \textbf{Data umpan balik} berupa komentar dari ahli 
	pembelajaran yang berisikan \textit{feedback}, \textit{motivation}, 
	dan \textit{appreciation}.
	\item \textbf{Data hasil kuesioner} dari ahli psikologi pendidikan yang berisikan
	penilaian dan komentar terhadap hasil umpan balik LLM dalam menilai
	performa belajar mahasiswa.
\end{enumerate}



\section{Metode yang Digunakan}
\subsection{Evaluasi Kualitas Umpan Balik LLM}
Penelitian ini menggunakan desain eksperimen untuk 
menilai kualitas \textit{context engineering} yang digabungkan dengan 
\textit{learning-analytics} dalam 
meningkatkan mutu umpan balik LLM berbahasa Indonesia pada 
skenario SRL berbasis papan Kanban. Setiap variasi \textit{context engineering}
diimplementasikan secara terpisah dan dipadukan dengan dua ukuran model 
(Llama~3.1~8B \textit{Instant} dan Llama~3.3~70B 
\textit{Versatile} melalui Groq API), menghasilkan beberapa kombinasi 
umpan balik yang diuji secara kuantitatif mapun kualitatif. Pengujian 
kuantitatif berfokus pada kesamaan semantik (Subbab [2.1.7]) 
menggunakan metrik berbasis \textit{embedding} kontekstual, 
yaitu BERTScore [22] dan BARTScore [23], untuk mengukur keselarasan 
antara respons LLM dengan data umpan balik ahli manusia 
\textit{(ground truth)} [32, 33]. Selain itu, evaluasi relevansi juga 
dilakukan menggunakan pendekatan \textit{LLM-as-a-Judge} 
yang terstruktur [24, 25]. Di sisi lain, pengujian 
kualitatif—yang merupakan standar emas (Subbab[2.1.8])—
melibatkan peninjauan langsung oleh pakar psikologi pendidikan [33]. 
Pakar ini menilai keluaran LLM menggunakan rubrik penilaian 
komprehensif yang didasarkan pada kerangka teoretis mapan 
seperti model Hattie dan Timperley (2007) [27], Shute (2008) [26], 
dan \textit{Self-Determination Theory} [28]. Penilaian ini mengukur 
tiga dimensi utama yakni \textit{feedback}, 
\textit{motivation support}, dan \textit{appreciation support} 
menggunakan skala Likert (1-5) untuk memastikan kualitas 
pedagogis dan psikologis dari umpan balik tersebut (Subbab[2.1.8]). 
Berikut adalah diagram alur metode evaluasinya.
\clearpage
\begin{figure}[H]
  \centering
  \includegraphics[width=\linewidth]{contents/chapter-3/Flowchart-Alur-Evaluasi-Kualitas-Umpan-Balik-LLM.png}
  \caption[Contoh gambar]{Diagram Alir Metode Evaluasi.}
  \label{Fig:diagram-alur-evaluasi}
\end{figure}
\clearpage
Terlihat dari Gambar \ref{Fig:diagram-alur-evaluasi}, alur evaluasi 
penelitian ini dimulai dengan generasi umpan balik LLM yang 
dipadukan dengan berbagai skema \textit{context engineering}. 
Umpan balik ini dihasilkan oleh dua model berbeda untuk 
perbandingan, yaitu Llama 3.1 8B \textit{Instant} dan 
Llama 3.3 70B \textit{Versatile}.

Keluaran dari model-model ini pertama-tama melalui 
tahap Uji Kuantitatif. Sejalan dengan metodologi yang telah 
dijelaskan, tahap ini berfokus pada pengukuran kesamaan semantik 
dan relevansi Subbab \ref{subsection:tinjauan_evaluasi_kuantitatif} menggunakan tiga metrik utama: 
Uji BERTScore, BARTScore, dan GPT-4o as a Judge [22, 23, 24].

Keluaran dari pengujian ini kemudian menjalani analisis 
hasil Uji Kuantitatif. Berdasarkan analisis ini, model 
terbaik dipilih—yaitu, kombinasi model dan teknik 
\textit{context engineering} yang menghasilkan skor keselarasan 
tertinggi terhadap \textit{ground truth} ahli.

Selanjutnya, umpan balik dari model terbaik tersebut dilanjutkan 
ke tahap evaluasi kedua, yaitu Uji Kualitatif. Tahap ini 
merupakan "standar emas" Subbab \ref{subsection:tinjauan_evaluasi_kualitatif} yang melibatkan peninjauan 
langsung oleh Ahli Pendidikan. Pakar ini menilai kualitas pedagogis 
dan psikologis dari umpan balik model terbaik menggunakan 
Kuesioner Skala Likert yang dirancang khusus berdasarkan kerangka 
teoretis [26, 27, 28]. Langkah terakhir adalah Analisis hasil uji 
kualitatif untuk mendapatkan kesimpulan akhir mengenai efektivitas 
umpan balik LLM dalam konteks SRL.
% Tolong bantu saya dalam menulis penjelasan gambar \textit{flowchart} 
% diatas dengan detail, sesuai dengan konteks paragraf nya. 


\section{Alur Tugas Akhir}
Dalam penelitian ini, terdapat sebuah alur kerja yang 
menjabarkan semua tahapan dan proses yang terlibat. 
Secara keseluruhan, proses penelitian terdiri dari 
empat tahap utama: tahap studi pendahuluan dan persiapan data, 
yang mencakup Tinjauan Pustaka dan 
Pembuatan \textit{Dataset} Papan Kanban Mahasiswa (termasuk 
pengumpulan \textit{ground truth} ahli); tahap desain eksperimen, 
yang meliputi Pemilihan Model LLMs dan perancangan Metode 
Peningkatan Hasil Umpan Balik (menggunakan Rekayasa Konteks, 
Integrasi \textit{Learning Analytics}, dan Pembuatan Variasi 
\textit{Prompt}); tahap implementasi, di mana dilakukan 
Pengambilan Data Umpan Balik LLM menggunakan metode yang telah 
dirancang; serta tahap evaluasi akhir, yang terdiri dari 
Pengujian Kuantitatif dan Kualitatif, diikuti oleh Analisis dan 
Evaluasi. Alur kerja penelitian secara keseluruhan diilustrasikan pada
Gambar \ref{Fig:flowchart-alur-penelitian}.
% Dalam penelitian ini, terdapat sebuah alur kerja yang 
% menjabarkan semua tahapan dan proses yang terlibat. Secara keseluruhan,
% proses penelitian....

\begin{figure}[H]
  \centering
  \includegraphics[height=24cm]{contents/chapter-3/Flowchart-Alur-Tugas-Akhir.png}
  \caption[Contoh gambar]{Flowchart Alur Penelitian Tugas Akhir.}
  \label{Fig:flowchart-alur-penelitian}
\end{figure}

% Menguraikan prosedur yang akan digunakan dan jadwal atau alur penyelesaian setiap 
% tahap. Alur penelian ini dapat disajikan dalam bentuk diagram. Diagram dapat disusun dengan aturan yang baik semisal menggunakan \textit{flowchart}. Aturan dan tutorial pembuatan \textit{flowchart} dapat dilihat di \textcolor{blue}{http://ugm.id/flowcharttutorial}. Setelah menggambarkannya, penulis wajib menjelaskan langkah-langkah setiap alur tugas akhir dalam sub bab tersendiri sesuai dengan kebutuhan.

\subsection{Tinjauan Pustaka}
Bab ini menyajikan tinjauan pustaka mendalam yang relevan dengan 
topik tesis. Tinjauan ini berfokus pada penelitian-penelitian 
sebelumnya mengenai peningkatan kualitas umpan balik 
LLM dalam konteks pendidikan, dengan menyelidiki dua arus 
utama yang teridentifikasi dari karya-karya kunci [Tabel 2.1]. 
Arus pertama adalah penyelarasan model (\textit{model alignment}) 
melalui \textit{fine-tuning}, seperti \textit{Direct Preference 
Optimization} (DPO) [13] dan \textit{Reinforcement Learning from 
Human Feedback} (RLHF) [16]. Arus kedua adalah penyelarasan konteks 
(\textit{context alignment}), yang mencakup rekayasa \textit{prompt} 
(\textit{prompt engineering}) berbasis teori [15] dan rekayasa konteks 
yang digabungkan dengan \textit{learning analytics} [17]. Tinjauan 
ini juga menganalisis pemanfaatan dasbor orkestrasi kelas berbasis 
Kanban [18] sebagai sumber data proses, bersama dengan metode 
evaluasi yang digunakan untuk menilai kinerja umpan balik, baik 
secara kuantitatif (misalnya, BERTScore [22], BARTScore [23], dan 
\textit{LLM-as-a-Judge} [24]) maupun secara kualitatif oleh ahli 
manusia sebagai standar emas [26, 27, 28].

Selanjutnya, analisis terhadap metodologi yang diterapkan dan 
temuan utama dalam studi-studi ini dilakukan. Tujuan utama dari 
analisis ini adalah untuk menilai secara kritis keunggulan dan 
keterbatasan dari setiap pendekatan. Evaluasi ini mengidentifikasi 
keunggulan (seperti efisiensi biaya dan kelayakan implementasi 
dari rekayasa konteks [15]) dan kelemahan (seperti tingginya 
kebutuhan sumber daya komputasi dan data untuk \textit{fine-tuning} 
[13, 16]). Temuan ini menjadi pertimbangan dalam pemilihan dan 
adaptasi metodologi yang paling tepat untuk menjawab masalah 
penelitian dalam karya ini: memanfaatkan pendekatan \textit{context 
engineering} yang ringan dan murah [15] namun diperkaya dengan 
data \textit{learning analytics} kaya sinyal yang diekstraksi 
dari papan Kanban [18] untuk mengatasi celah penelitian yang 
teridentifikasi.

\subsection{Pembuatan Dataset Papan Kanban Mahasiswa}
Pada bagian ini, dilakukan proses pembuatan \textit{Dataset} 
Papan Kanban Mahasiswa. Tahap ini mencakup beberapa langkah 
krusial, yaitu Pembuatan \textit{Prompt} , Pembuatan 
Data Sintetis yang akan menjadi data papan Kanban , serta Pengambilan 
data umpan balik pedagogis dari ahli manusia yang akan 
digunakan sebagai \textit{ground truth} untuk perbandingan.

\subsubsection{Pembuatan \textit{Prompt}}
Proses pembuatan \textit{prompt} dimulai dengan analisis 
mendalam terhadap struktur \textit{database} asli dari platform 
pembelajaran berbasis papan Kanban. Platform yang digunakan dan 
dikembangkan bernama Gamatutor yang dikembangkan oleh M. I. Azmi \cite{azmi2025pengembangan}. 
Berikut adalah tampilan dari
platform pembelajaran tersebut yang ditampilkan pada Gambar \ref{Fig:gamatutor_1}.

% Pengembangan kanban board untuk self regulated learning.

\begin{figure}[H]
  \centering
  \includegraphics[width=\linewidth]{contents/chapter-3/gamatutor_1.png}
  \caption[Contoh gambar]{Tampilan Utama Platform Gamatutor.}
  \label{Fig:gamatutor_1}
\end{figure}

Secara umum, Gamatutor adalah platform \textit{learning analytics} berbasis 
metode \textit{Kanban} yang dirancang untuk mendukung 
SRL mahasiswa. Platform ini memfasilitasi pengelolaan 
tugas belajar melalui sistem 
papan visual dengan empat tahapan utama yaitu \textit{Planning (To Do)}, 
\textit{Monitoring (In Progress)}, \textit{Controlling (Review)}, 
dan \textit{Reflection (Done)}. Setiap tugas direpresentasikan 
dalam bentuk \textit{kartu} yang memuat informasi lengkap seperti 
judul, mata kuliah, \textit{priority}, \textit{difficulty}, 
\textit{learning strategy}, nilai \textit{pre-test/post-test}, 
dan \textit{checklist} progres. Fitur utamanya meliputi pembuatan 
dan pengelolaan kartu tugas, pemantauan \textit{progress} melalui 
\textit{learning analytics}, serta manajemen profil pengguna.

Untuk satu kartunya merepresentasikan satu sub-topik dari salah satu mata kuliah.
Jadi mahasiswa bisa membuat banyak kartu untuk mata kuliah yang sama.
Untuk mengakses detail informasi-informasi tersebut, pengguna (mahasiswa)
dapat melihatnya pada tiap-tiap kartu dengan menekan kartunya. Tampilan 
dari informasinya dapat dilihat pada Gambar \ref{Fig:gamatutor_2}

\clearpage
\begin{figure}[H]
  \centering
  \includegraphics[width=\linewidth]{contents/chapter-3/gamatutor_2.png}
  \caption[Contoh gambar]{Tampilan Informasi Tiap Kartu Gamatutor.}
  \label{Fig:gamatutor_2}
\end{figure}
\clearpage

Semua detail informasi dan fitur-fitur tersebut dibuat untuk membantu
mahasiswa dalam menerapkan SRL (seperti: \textit{Start Timer}, 
\textit{Checklists}, dan "Add Link") serta secara bersamaan 
memberikan data proses yang kaya untuk \textit{learning analytics}. 
Data ini mencakup \textit{Total study time} (yang dihasilkan dari fitur 
\textit{Start Timer}), progres penyelesaian tugas melalui Checklists, 
hingga data metakognitif dan performa seperti \textit{Priority, 
Difficulty, Learning Strategy} yang dipilih, serta nilai \textit{Pre-test 
Grade} dan \textit{Post-test Grade}. Kumpulan data proses inilah yang dapat 
menjadi insight bagi agen pedagogis (guru ataupun \textit{AI chatbot}) untuk 
memantau, mendiagnosis hambatan, dan memberikan bimbingan yang 
terpersonalisasi sesuai dengan kondisi aktual mahasiswa tersebut.
Gamatutor juga memiliki \textit{Learning Assistant} dengan berbasiskan chatbot LLM 
yang rencananya akan diterapkan secara bertahap. Penelitian ini adalah 
bagian dari pengembangan \textit{Learning Asistant} tersebut.

Semua informasi mengenai pembelajaran mahasiswa dan kartu-kartu Kanban-nya
secara rapi disimpan ke dalam \textit{database} NoSQL yaitu MongoDB.
Data tersebut disimpan dengan mengikuti format \texttt{JSON}. 
Berikut adalah skema penyimpanan data kartu Kanban dalam MongoDB.


\begin{figure}[H]
  \centering
  \includegraphics[height=18cm]{contents/chapter-3/skema_db_kanban_board.png}
  \caption[Contoh gambar]{Skema Penyimpanan Data Kartu Pembelajaran Kanban Mahasiswa.}
  \label{Fig:schema_kanban_mongodb}
\end{figure}

Dapat dilihat pada Gambar \ref{Fig:schema_kanban_mongodb} terdapat banyak 
atribut, khususnya pada entitas \texttt{Card}, yang menyimpan data-data kunci 
untuk merefleksikan proses dan hasil belajar mahasiswa.

Beberapa atribut yang paling penting untuk mengevaluasi kemampuan belajar 
ini meliputi \texttt{pre\_test\_grade} dan \texttt{post\_test\_grade}. 
Kedua atribut ini secara langsung mengukur pengetahuan mahasiswa sebelum dan 
sesudah mengerjakan tugas, sehingga selisih antara keduanya dapat menjadi 
indikator utama peningkatan pemahaman.
% (\textit{learning gain}).

Selain itu, atribut \texttt{difficulty} mengkategorikan tingkat kesulitan 
tugas, yang memungkinkan analisis performa mahasiswa terhadap tantangan yang 
diberikan. Atribut \texttt{column\_movement\_times} juga sangat kaya data 
dengan \texttt{timestamp} di dalamnya, data ini dapat digunakan untuk 
menghitung total durasi pengerjaan, waktu yang dihabiskan dalam fase belajar 
aktif (di kolom "In Progress"), atau bahkan mendeteksi adanya revisi. Terdapat juga
atribut \texttt{notes} yang berisikan catatan dari mahasiswa ketika kartu
sudah berada di tahap \textit{Controlling (Review)}. Isi catatan ini bisa menjadi
gambaran penting tentang rangkuman pembelajaran yang dilakukan mahasiswa terhadap 
sub-topik tersebut yang menjadikannya data yang cukup penting.

Atribut lain yang relevan adalah \texttt{checklists} untuk melihat progres 
granular pengerjaan sub-tugas, \texttt{learning\_strategy} untuk 
memahami pendekatan belajar yang dipilih mahasiswa, dan \texttt{rating} untuk
memberikan penilaian terhadap pengalaman belajar terhadap sub-topik yang dipelajari.
Kombinasi dari atribut-atribut inilah yang memberikan gambaran komprehensif 
mengenai pola dan kemampuan belajar mahasiswa.

Dari skema MongoDB diatas, 
dilakukan reduksi atribut untuk memfokuskan pada data yang paling 
esensial bagi penelitian, seperti \texttt{created\_at}, 
\texttt{checklists}, dan \texttt{column\_movements}. 
Selain itu, dilakukan reduksi ini untuk mempermudah pembuatan
data sintetis. Berikut adalah hasil skema database MongoDB setelah
dilakukan reduksi data yang ada pada Gambar \ref{Fig:schema_kanban_mongodb_reduksi}.

\begin{figure}[H]
  \centering
  \includegraphics[height=16cm]{contents/chapter-3/skema_db_kanban_board_reduksi.png}
  \caption[Contoh gambar]{Skema Penyimpanan Data Kartu Pembelajaran Kanban Mahasiswa Setelah Reduksi}
  \label{Fig:schema_kanban_mongodb_reduksi}
\end{figure}


% Untuk memastikan data sintetis yang dihasilkan dapat merepresentasikan 
% variasi perilaku mahasiswa di dunia nyata, lima kategori profil 
% (Malas, Sedikit Malas, Cukup, Rajin, dan Sangat Rajin) dirumuskan. 
Untuk mencapai tingkat representasi yang valid, data tersebut harus mampu 
menangkap berbagai pola perilaku dan tingkat keterlibatan mahasiswa yang 
berbeda-beda, mulai dari yang paling tidak aktif hingga yang sangat proaktif.
Sebagai strategi utama untuk mengimplementasikan keragaman ini, sebuah 
studi kasus telah dirumuskan yang berfokus pada pengkategorian mahasiswa ke 
dalam profil-profil perilaku yang spesifik.
Melalui studi kasus ini, telah diidentifikasi dan ditetapkan lima kategori 
profil yang berbeda, di mana setiap kategori mewakili satu titik pada 
spektrum tingkat kerajinan mahasiswa.
Kelima kategori yang komprehensif ini yang diberi label \texttt{"MALAS"}, \texttt{"Sedikit 
Malas"}, \texttt{"Cukup"}, \texttt{"Rajin"}, dan \texttt{"Sangat Rajin"}, akan digunakan sebagai dasar 
untuk menghasilkan set data yang bervariasi.

Untuk menyederhanakan penelitian ini, penggunaan asumsi akan digunakan dalam
memberikan label kategori tingkat performa dan perilaku mahasiswa.
Variabel atribut yang digunakan antara lain adalah jumlah total \texttt{card}
yang dibuat oleh mahasiswa, tingkat penyelesaian tugas (\texttt{checklist}),
aktivitas terkini dari kartu tersebut pada \texttt{column\_movements}, dan
letak keberadaan \texttt{card} tersebut. Hal ini diharapkan dapat menjadi proksi 
yang \textit{memadai dan terukur} untuk membedakan secara jelas antara kelima 
profil perilaku dan performa tersebut, sehingga penyederhanaan ini tetap 
dapat menghasilkan data yang representatif.

Aturan-aturan tersebut dirumuskan dan diimplementasikan pada studi kasus
seperti jumlah total kartu, persentase 
penyelesaian \textit{checklist}, dan stempel waktu pergerakan kartu
yang secara implementasinya dapat dilihat pada Tabel \ref{tab:matriks_studi_kasus_tabularx}.

% Pastikan paket-paket ini ada di preamble Anda:
% \usepackage{booktabs}
% \usepackage{array}
% \usepackage{float}

% \begin{table}[H]
%   \centering
%   \caption{Matriks Aturan Studi Kasus Perilaku Mahasiswa}
%   \label{tab:matriks_studi_kasus}
%   % Menggunakan @{} untuk menghapus spasi ekstra di sisi tabel
%   % Kolom 'l' untuk rata kiri, 'p{width}' untuk paragraf dengan text-wrap
%   \begin{tabular}{@{} l l p{3cm} p{3.5cm} p{4.5cm} @{}}
%     \toprule
%     \textbf{Kategori} & \textbf{Studi Kasus} & \textbf{Jumlah Kartu} & \textbf{Penyelesaian Checklist (per kartu)} & \textbf{Aturan Aktivitas \& Posisi Kartu} \\
%     \midrule

%     \textbf{Malas} & MHS\_MALAS\_SK1 & 1 - 2 kartu & 0 - 2 item selesai (dari 3) & \textbf{Tidak ada aktivitas} (2 jam terakhir). \newline Kartu cenderung di list1 atau list2. \\
%     \midrule

%     \textbf{Sedikit Malas} & MHS\_SEDMALAS\_SK1 & 3 - 4 kartu & 0 - 2 item selesai (dari 3) & \textbf{Tidak ada aktivitas} (2 jam terakhir). \\
%     \addlinespace % Menambah sedikit spasi antar sub-grup
%     & MHS\_SEDMALAS\_SK2 & 3 - 4 kartu & 1 - 2 item selesai (dari 3) & \textbf{Tidak ada aktivitas} (2 jam terakhir). \\
%     \midrule

%     \textbf{Cukup} & MHS\_CUKUP\_SK1 & 4 - 5 kartu & 1 - 2 item selesai (dari 3) & \textbf{Ada aktivitas} (2 jam terakhir) untuk min. 3 kartu. \\
%     \addlinespace
%     & MHS\_CUKUP\_SK2 & 3 - 4 kartu & Semua 3 item selesai & \textbf{Ada aktivitas} (2 jam terakhir) untuk min. 3 kartu. \\
%     \midrule

%     \textbf{Rajin} & MHS\_RAJIN\_SK1 & Tepat 5 kartu & 2 - 3 item selesai (dari 3) & Hanya 1 kartu tersisa di list1 ("Planning (To Do)"). \\
%     \addlinespace
%     & MHS\_RAJIN\_SK2 & 4 - 5 kartu & Semua 3 item selesai & Hanya 1 kartu tersisa di list1 ("Planning (To Do)"). \\
%     \midrule

%     \textbf{Sangat Rajin} & MHS\_SANGRAJIN\_SK1 & Tepat 5 kartu & Semua 3 item selesai & Semua 5 kartu berada di list4 ("Reflection (Done)"). \\
%     \bottomrule
%   \end{tabular}
% \end{table}

\begin{table}[H]
  \centering
  \caption{Tabel Aturan Studi Kasus Perilaku Mahasiswa}
  \label{tab:matriks_studi_kasus_tabularx}

  \begingroup
  \setlength{\tabcolsep}{8pt}        % Mengatur spasi antar kolom
  \renewcommand{\arraystretch}{1.5} % Mengatur tinggi baris
  \scriptsize                        % <<< Menggunakan font \scriptsize
  
  % Menggunakan tabularx agar lebar tabel = \textwidth
  % Kolom: l (rata kiri), l (rata kiri), l (rata kiri), Y (fleksibel), Y (fleksibel)
  \begin{tabularx}{\textwidth}{@{} l l l Y Y @{}}
    \toprule
    \textbf{Kategori} & \textbf{Studi Kasus} & \textbf{Jumlah Kartu} & \textbf{Penyelesaian Checklist (per kartu)} & \textbf{Aturan Aktivitas \& Posisi Kartu} \\
    \midrule

    \textbf{Malas} & MHS\_MALAS\_SK1 & 1 - 2 kartu & 0 - 2 item selesai (dari 3) & \textbf{Tidak ada aktivitas} (2 jam terakhir). Kartu cenderung di list1 atau list2. \\
    \midrule

    \textbf{Sedikit Malas} & MHS\_SEDMALAS\_SK1 & 3 - 4 kartu & 0 - 2 item selesai (dari 3) & \textbf{Tidak ada aktivitas} (2 jam terakhir). \\
    \addlinespace % Menambah sedikit spasi antar sub-grup
    & MHS\_SEDMALAS\_SK2 & 3 - 4 kartu & 1 - 2 item selesai (dari 3) & \textbf{Tidak ada aktivitas} (2 jam terakhir). \\
    \midrule

    \textbf{Cukup} & MHS\_CUKUP\_SK1 & 4 - 5 kartu & 1 - 2 item selesai (dari 3) & \textbf{Ada aktivitas} (2 jam terakhir) untuk min. 3 kartu. \\
    \addlinespace
    & MHS\_CUKUP\_SK2 & 3 - 4 kartu & Semua 3 item selesai & \textbf{Ada aktivitas} (2 jam terakhir) untuk min. 3 kartu. \\
    \midrule

    \textbf{Rajin} & MHS\_RAJIN\_SK1 & Tepat 5 kartu & 2 - 3 item selesai (dari 3) & Hanya 1 kartu tersisa di list1 ("Planning (To Do)"). \\
    \addlinespace
    & MHS\_RAJIN\_SK2 & 4 - 5 kartu & Semua 3 item selesai & Hanya 1 kartu tersisa di list1 ("Planning (To Do)"). \\
    \midrule

    \textbf{Sangat Rajin} & MHS\_SANGRAJIN\_SK1 & Tepat 5 kartu & Semua 3 item selesai & Semua 5 kartu berada di list4 ("Reflection (Done)"). \\
    \bottomrule
  \end{tabularx}
  
  \endgroup
\end{table}



Kategori-kategori ini kemudian dijadikan dasar untuk membuat 
\textit{prompt} generator data sintetis pembelajaran mahasiswa berbasis Kanban.
Kumpulan aturan yang telah dirangkum dalam Tabel 
\ref{tab:matriks_studi_kasus_tabularx} tersebut kemudian ditransformasi 
menjadi sebuah \textit{prompt} instruksi yang terstruktur. 
Proses transformasi ini melibatkan penerjemahan setiap studi 
kasus—mulai dari MHS\_MALAS\_SK1 hingga MHS\_SANGRAJIN\_SK1, dari format 
tabel tabular menjadi format teks naratif. Tujuan utamanya adalah 
untuk menyajikan aturan-aturan tersebut dalam format bahasa alami 
(\textit{natural language}) yang dapat dipahami dan dieksekusi 
secara presisi oleh agen \textit{chatbot} (LLM). \textit{Prompt} 
ini secara eksplisit merinci batasan untuk setiap variabel, 
termasuk jumlah kartu yang harus dibuat, syarat penyelesaian 
\textit{checklist}, dan aturan mengenai aktivitas pergerakan kartu.
Terdapat sedikit penambahan aturan pada pembuatan \textit{prompt}
yaitu adanya waktu kapan dibuatnya kartu sub-topik tersebut.
Contoh \textit{prompt} lengkap yang telah diformulasikan untuk 
agen generator tersebut dapat dilihat secara rinci pada potongan
snipet pada Tabel \ref{tab:prompt_studi_kasus_generator}.
% \ref{Fig:contoh_prompt}.

% \begin{verbatim}
% This text contains \LaTeX commands like \textbf{bold} and special characters like % and $.
% It will be printed exactly as typed, with a monospaced font.
% \end{verbatim}



% Pastikan preamble memuat:
% \usepackage{xcolor}
% \usepackage{listings}

% \lstset{...} (Ini bisa ditaruh di preamble atau di sini)
\lstset{
    language=Python,
    basicstyle={\fontsize{8}{8}\selectfont\color{black}\ttfamily\bfseries},
    keywordstyle=\color{black}\bfseries,
    commentstyle=\color{black},
    stringstyle=\color{black},
    breaklines=true,
    showstringspaces=false
}

%--- MULAI LINGKUNGAN TABEL ---
\begin{table}[H] % Anda bisa ganti [H] dengan [htbp] jika perlu
  \centering
  
  % Caption dan Label sekarang milik "table", bukan "lstlisting"
  \caption{\textit{Prompt} untuk Studi Kasus Data Sintetis Mahasiswa}
  \label{tab:prompt_studi_kasus_generator} % Saya ubah labelnya menjadi "tab:"
  
  % Gunakan minipage untuk "membungkus" listing dengan aman
  \begin{minipage}{\textwidth}
    
    % Mulai listing Anda (tanpa caption/label di sini)
    \begin{lstlisting}
"""

Kategori: Malas

* Aturan untuk Studi Kasus 1 (MHS_MALAS_SK1):

* Buat 1 papan JSON dengan 1 atau 2 kartu.

* Setiap kartu memiliki 3 checklist, dengan total 0 hingga 2 item yang dicentang (completed: true) per kartu.

* Tidak ada kartu yang berpindah tempat dalam kurun waktu 2 jam terakhir (semua timestamp created_at atau column_movements terakhir harus <= 2025-06-07T15:00:00Z). Kartu cenderung berada di list1 atau list2.



Kategori: Sedikit Malas

* Aturan untuk Studi Kasus 1 (MHS_SEDMALAS_SK1):

* Buat 1 papan JSON dengan 3 atau 4 kartu.

* Setiap kartu memiliki 3 checklist, dengan total 0 hingga 2 item yang dicentang per kartu.

* Tidak ada kartu yang berpindah tempat dalam kurun waktu 2 jam terakhir.

* Aturan untuk Studi Kasus 2 (MHS_SEDMALAS_SK2):

* Buat 1 papan JSON dengan 3 atau 4 kartu.

* Setiap kartu memiliki 3 checklist, dengan total 1 hingga 2 item yang dicentang per kartu.

* Tidak ada kartu yang berpindah tempat dalam kurun waktu 2 jam terakhir.


# .... Diatas adalah potongan yang tidak lengkap dari kode yang utuh. ....

"""
\end{lstlisting}
%--- Selesai listing ---

  \end{minipage} 
  %--- Selesai minipage ---

\end{table}
%--- SELESAI LINGKUNGAN TABEL ---

Selanjutnya, untuk memperkecil area 

% Aturan skema dan kategori perilaku ini kemudian disusun menjadi 
% sebuah \textit{role-play prompt} yang komprehensif untuk diumpankan 
% ke model generator (ChatGPT). \textit{Prompt} ini menginstruksikan 
% model untuk berperan sebagai "AI yang ahli dalam pembuatan data 
% sintetis" dan menggabungkan teknik \textit{Chain-of-Thought} 
% (CoT) serta \textit{Few-shot} (contoh) untuk memandu LLM menghasilkan 
% 40 \textit{dataset} papan Kanban dalam format JSON  yang valid dan 
% sesuai dengan profil perilaku yang diminta.
Setelah \textit{prompt} studi kasus mahasiswa telah selesai dibuat,
tahapan selanjutnya adalah pembuatan \textit{prompt} intruksi yang
spesifik mengarah pada bentuk hasil data sintetis yang akan dibuat.
Instruksi \textit{prompt} yang digunakan yaitu \textit{role-play}, 
\textit{Few-shot}, dan \textit{Chain-of-Thought} (CoT).

Pada awal \textit{prompt} digunakan terlebih dahulu \textit{prompt role-play}. 
Strategi \textit{prompting} ini secara spesifik menginstruksikan 
model untuk mengadopsi persona sebagai "AI yang ahli dalam 
pembuatan data sintetis". Penetapan peran (\textit{role-play}) ini 
penting untuk mengatur konteks dan ekspektasi, mendorong 
model agar tidak hanya memberikan jawaban umum, tetapi 
bertindak sebagai pakar domain yang memahami nuansa validasi 
skema JSON dan logika perilaku. Berikut adalah potongan
\textit{prompt} dipaparkan pada Tabel \ref{tab:prompt_role_play}.

\begin{table}[H] % Anda bisa ganti [H] dengan [htbp] jika perlu
  \centering
  
  % Caption dan Label sekarang milik "table", bukan "lstlisting"
  \caption{\textit{Prompt Role-Play} pada Pembuatan Data Sintetis}
  \label{tab:prompt_role_play} % Saya ubah labelnya menjadi "tab:"
  
  % Gunakan minipage untuk "membungkus" listing dengan aman
  \begin{minipage}{\textwidth}
    
    % Mulai listing Anda (tanpa caption/label di sini)
    \begin{lstlisting}
"""
Anda adalah AI yang ahli dalam pembuatan data sintetis. Tugas Anda adalah menghasilkan serangkaian data papan Kanban dalam format JSON. Setiap objek JSON mewakili satu papan Kanban milik seorang mahasiswa, yang mencerminkan profil perilaku tertentu. Anda akan menghasilkan objek JSON papan Kanban yang berbeda, sesuai dengan aturan spesifik untuk 5 kategori mahasiswa.
Konteks Umum (Gunakan untuk semua data yang dihasilkan):
"""
\end{lstlisting}
%--- Selesai listing ---

  \end{minipage} 
  %--- Selesai minipage ---

\end{table}

\textit{Prompt} yang ditunjukkan pada Tabel \ref{tab:prompt_role_play}
mengatur kerangka kerja dan identitas dari model pembuat Data Sintetis. 
Instruksi tersebut dimulai dengan penetapan peran yang spesifik sebagai 
"AI yang ahli dalam pembuatan data sintetis". Penugasan peran 
ini sangat penting untuk mengarahkan model agar bertindak sebagai 
pakar domain alih-alih asisten umum. Tugas utamanya kemudian 
didefinisikan secara eksplisit yaitu menghasilkan data papan Kanban 
dengan format keluaran wajib berupa JSON. \textit{Prompt} ini juga 
secara cerdas membangun koneksi ke konteks penelitian dengan 
menyatakan bahwa setiap objek JSON mewakili satu mahasiswa dan 
harus mencerminkan profil perilaku tertentu. Selain itu instruksi ini 
mempersiapkan model untuk keragaman tugas dengan menyebutkan 
bahwa akan ada variasi data berdasarkan lima kategori mahasiswa 
yang spesifik. Kalimat terakhir "Konteks Umum" berfungsi sebagai 
penjembatan yang menandakan bahwa aturan-aturan berikutnya yang akan 
diberikan bersifat global atau berlaku untuk semua data yang akan 
dihasilkan.

Untuk lebih memandu model dan memastikan kepatuhan terhadap 
aturan yang kompleks, \textit{prompt} tersebut diperkaya dengan 
dua teknik lanjutan. Pertama, teknik \textit{Few-shot} 
(pemberian contoh) digunakan dengan menyertakan beberapa contoh 
keluaran JSON yang sudah jadi. Ini bertujuan untuk memberi model 
pemahaman konkret mengenai format dan struktur data yang diharapkan.

Kedua, \textit{prompt} ini juga mengintegrasikan metodologi 
\textit{Chain-of-Thought} (CoT). Teknik ini mengharuskan model 
untuk "berpikir langkah demi langkah" atau mengartikulasikan 
proses logisnya sebelum menghasilkan keluaran akhir. Hal ini 
sangat penting untuk memastikan model secara sadar memeriksa 
dan menerapkan aturan heuristik untuk setiap profil perilaku—seperti 
menghitung jumlah kartu atau memeriksa stempel waktu—sebelum 
menulis data JSON-nya.

Tujuan akhir dari \textit{prompt} gabungan yang canggih ini 
adalah untuk memandu LLM secara sistematis dalam menghasilkan 
40 \textit{dataset} papan Kanban. Setiap \textit{dataset} yang 
dihasilkan harus memenuhi dua kriteria utama: (1) secara teknis 
valid dan dapat diproses (valid JSON), dan (2) secara fungsional 
akurat, di mana data di dalamnya harus benar-benar merepresentasikan 
profil perilaku mahasiswa yang diminta.

% \begin{enumerate}
% 	\item Perancangan \textit{prompt}
% 	\begin{itemize}
% 		\item Menentukan tujuan setiap \textit{prompt} (mis. menghasilkan tugas, status, komentar reflektif).
% 		\item Menyusun format keluaran yang konsisten (mis. JSON dengan field: id, task, status, timestamp, comment).
% 		\item Menyertakan instruksi konteks agar output sesuai skenario SRL berbasis Kanban.
% 	\end{itemize}

% 	\item Generasi data sintetis
% 	\begin{itemize}
% 		\item Menggunakan ChatGPT untuk membuat contoh papan Kanban berbasis skenario mahasiswa.
% 		\item Membuat variasi kondisi (mis. beban tugas, tingkat prokrastinasi, interaksi tim).
% 		\item Menyimpan keluaran dalam berkas \texttt{.json} terstruktur.
% 	\end{itemize}

% 	\item Pengumpulan \textit{ground truth} ahli
% 	\begin{itemize}
% 		\item Merekrut ahli pembelajaran untuk menghasilkan umpan balik pedagogis pada contoh data.
% 		\item Menstandarisasi format umpan balik (kategori: feedback, motivation, appreciation).
% 		\item Menyimpan anotasi ahli sebagai pasangan input–ground truth untuk evaluasi.
% 	\end{itemize}

% 	\item Validasi dan pembersihan data
% 	\begin{itemize}
% 		\item Memeriksa konsistensi format dan menghapus entri duplikat atau tidak relevan.
% 		\item Melakukan pemeriksaan kualitas anotasi oleh reviewer kedua jika diperlukan.
% 	\end{itemize}

% 	\item Pengelolaan versi dan penyimpanan
% 	\begin{itemize}
% 		\item Menyimpan versi dataset dengan penomoran dan catatan perubahan.
% 		\item Menyimpan salinan aman (lokal dan/atau cloud) serta mencatat metadata (tanggal, pembuat).
% 	\end{itemize}

% 	\item Pertimbangan etis dan privasi
% 	\begin{itemize}
% 		\item Membuat data anonim dan memastikan tidak ada informasi identitas pribadi.
% 		\item Mendokumentasikan izin penggunaan data jika diperlukan.
% 	\end{itemize}
% \end{enumerate}

\section{Etika, Masalah, dan Keterbatasan Penelitian (Opsional))}

Bagian ini membahas pertimbangan etis penelitian dan [potensi] masalah serta
keterbatasannya. Jika menyangkut penelitian dengan makhluk hidup, maka dibutuhkan adanya \textit{ethical clearance}, di bagian ini hal itu akan dibahas. Demikian juga tentang keterbatasan ataupun masalah yang akan timbul.
