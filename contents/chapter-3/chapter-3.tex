\chapter{Metode Penelitian}

Bab ini menjelaskan metode atau cara yang digunakan dalam penelitian ini untuk 
mencapai maksud dan tujuan seperti yang tertulis dalam sub-bab 1.3 yaitu:
\begin{enumerate}
	\item Merancang skema \textit{prompt} serta memberikan metode \textit{learning analytics} 
	yang efektif untuk meningkatkan kualitas umpan balik LLM agar mendekati ahli manusia.
	% \item Mengevaluasi secara empiris apakah adanya peningkatan kualitas umpan balik 
	% LLM atas proses SRL mahasiswa baik secara kuantitatif maupun kualitatif dengan umpan 
	% balik ahli manusia.
	\item Mengevaluasi apakah adanya peningkatan kualitas umpan balik 
	LLM atas proses SRL mahasiswa baik secara kuantitatif maupun kualitatif dengan umpan 
	balik ahli manusia.
	\item Melihat pengaruh besar ukuran LLM terhadap kualitas umpan balik yang dihasilkan.
\end{enumerate}
% [jika diinginkan, kalian dapat menuliskan Kembali tujuan penelitian yang ingin dicapai di sini].

\section{Alat dan Bahan Tugas Akhir}

\subsection{Alat Tugas Akhir}

Alat-alat yang digunakan pada tugas akhir ini berupa perangkat keras maupun 
perangkat lunak sebagai sarana pendukung. Alat-alat perangkat keras yang digunakan
merupakan milik pribadi. Fungsi dari perangkat keras dan lunak ini adalah untuk 
penulisan kode, pengembangan chatbot, melakukan pengujian baik secara kuantitatif
maupun kualitatif, serta untuk pengumpulan dataset.Berikut adalah daftar 
alat-alat tugas akhir yang digunakan pada penelitian ini.

% \begin{enumerate}
% 	\item \textit{Notebook} Apple MacBook Pro 14 inci (M1 Pro, 2021), SoC Apple M1 Pro 8-core CPU dan 14-core GPU dengan 16-core Neural Engine, memori terpadu 16 GB, SSD 512 GB. 
% 	\item Visual Studio Code Version: 1.105.1.
% 	\item Google Docs
% 	\item Google Sheets
% 	\item Groq API 
	
\begin{enumerate}
  \item \textbf{Macbook Pro 14 inci.}
  Spesifikasi utama:
  \begin{itemize}
    \item Prosesor: M1 Pro 10 Core @ 3{,}2~GHz
    \item GPU internal: M1 Pro GPU 16 Core
    \item Neural Engine: 32 Core
    \item RAM: \textit{Unified Memory} 16~GB 200GB/s \textit{memory bandwidth}
    \item Penyimpanan: \textit{Solid State Drive} internal 1~TB (MacOS 26 Tahoe)
  \end{itemize}

  \item \textbf{Visual Studio Code} versi 1.105.1 sebagai \textit{integrated development 
  environment} (IDE) untuk menulis dan mengelola kode.

  \item \textbf{Python 3.12.7} sebagai bahasa pemrograman utama untuk 
  mengembangkan \textit{chatbot} berbasis LLM, dengan beberapa pustaka inti:
  \begin{itemize}
    \item \texttt{LangChain}: kerangka kerja aplikasi LLM yang menyediakan 
	komponen modular untuk orkestrasi \textit{prompt}, alur data, dan 
	integrasi sumber data eksternal.
    \item \texttt{LangGraph}: ekstensi \texttt{LangChain} untuk membangun 
	alur aplikasi yang lebih kompleks dalam bentuk graf.
	\item \texttt{bert\_score}: sebagai salah satu kerangka kerja evaluasi kuantitatif 
	untuk mengukur kesamaan semantik (Precision, Recall, F1) antara respons yang 
	dihasilkan model dengan data ground truth.
	\item \texttt{BARTScorer} sebagai kerangka kerja evaluasi kuantitatif alternatif 
	(menggunakan mBART) untuk menghitung skor F1, Precision, dan Recall 
	berdasarkan log-probability token.
	\item \texttt{Pandas} sebagai library yang digunakan untuk menyusun, mengagregasi, dan menampilkan tabel ringkasan hasil evaluasi kuantitatif (BARTScore dan GPT-4o Judge)  agar mudah dibaca.


	% \item \texttt{LangSmith}: pemantau dan \textit{tracing} interaksi LLM yang membantu proses \textit{debugging} dan evaluasi.
    % \item \texttt{CyVer}: pustaka untuk validasi \textit{Cypher query} (sintaksis, skema, dan properti) yang dihasilkan LLM saat mengambil data dari Neo4J KG.
    % \item \texttt{Streamlit}: kerangka kerja untuk membangun antarmuka pengguna (\textit{user interface}) \textit{chatbot}.
  \end{itemize}
  	\item \textbf{Google Docs}: sebagai alat untuk menulis kuesioner, dan melakukan 
	revisi serta dokumentasi hasil diskusi.
	\item \textbf{Google Sheets}: digunakan untuk memvisualisasikan dan mengelola 
	data Kanban sehingga dapat dibagikan kepada para ahli yang akan mengisi 
	dan menilai data, memudahkan kolaborasi, pengisian respons, serta 
	pengumpulan metrik penilaian.

  \item \textbf{Anaconda} untuk manajemen pustaka Python dan 
  \textit{virtual environment} selama proses pengembangan.

  \item \textbf{ChatGPT 5 Thinking} untuk membuat data sintetis papan Kanban yang 
  akan digunakan dalam penelitian ini.

%   \item \textbf{Neo4J Knowledge Graph} sebagai basis data untuk 
%   menyimpan data Kurikulum 2021 Program Studi Teknologi Informasi, 
%   Fakultas Teknik, Universitas Gadjah Mada.

  \item \textbf{Groq API} untuk melakukan inferensi LLM 
  (pemanggilan model) yang digunakan dalam penelitian ini.
\end{enumerate}


\subsection{Bahan Tugas Akhir}

Bahan yang digunakan dalam proses pembuatan chatbot dan untuk uji coba pada
penelitian ini adalah sebagai berikut:

\begin{enumerate}
	\item LLM dengan model \textbf{Llama 3.1 8B Instant}, dengan parameter
	delapan miliar, yang diperoleh dari platform Groq API yang berfungsi
	sebagai agen pemberi umpan balik (\textit{Feedback Agent}).
	\item LLM dengan model \textbf{Llama 3.3 70B Versatile}, dengan parameter
	lebih besar yaitu 70 miliar yang juga diperoleh dari platform Groq API yang
	berfungsi sebagai agen pemberi umpan balik (\textit{Feedback Agent}).
	\item \textbf{GPT-4o} sebagai model yang digunakan untuk evaluasi kuantitatif
	\textit{LLM-as-a-judge}
	\item \textbf{Dataset sintetis} dengan format data \texttt{.json} yang 
	di-\textit{generate} dari ChatGPT, berisikan pasangan data 
	pembelajaran mahasiswa berbasis papan Kanban.
	\item \textbf{Data umpan balik} berupa komentar dari ahli 
	pembelajaran yang berisikan \textit{feedback}, \textit{motivation}, 
	dan \textit{appreciation}.
	\item \textbf{Data hasil kuesioner} dari ahli psikologi pendidikan yang berisikan
	penilaian dan komentar terhadap hasil umpan balik LLM dalam menilai
	performa belajar mahasiswa.
\end{enumerate}



\section{Metode yang Digunakan}
\subsection{Evaluasi Kualitas Umpan Balik LLM}
Penelitian ini menggunakan desain eksperimen untuk 
menilai kualitas \textit{context engineering} yang digabungkan dengan 
\textit{learning-analytics} dalam 
meningkatkan mutu umpan balik LLM berbahasa Indonesia pada 
skenario SRL berbasis papan Kanban. Setiap variasi \textit{context engineering}
diimplementasikan secara terpisah dan dipadukan dengan dua ukuran model 
(Llama~3.1~8B Instant dan Llama~3.3~70B 
Versatile melalui Groq API), menghasilkan beberapa kombinasi 
umpan balik yang diuji secara kuantitatif mapun kualitatif. Pengujian 
kuantitatif berfokus pada kesamaan semantik (Subbab [2.1.7]) 
menggunakan metrik berbasis \textit{embedding} kontekstual, 
yaitu BERTScore [22] dan BARTScore [23], untuk mengukur keselarasan 
antara respons LLM dengan data umpan balik ahli manusia 
\textit{(ground truth)} [32, 33]. Selain itu, evaluasi relevansi juga 
dilakukan menggunakan pendekatan \textit{LLM-as-a-Judge} 
yang terstruktur [24, 25]. Di sisi lain, pengujian 
kualitatif—yang merupakan standar emas Subbab \ref{subsection:tinjauan_evaluasi_kualitatif} yang
melibatkan peninjauan langsung oleh pakar psikologi pendidikan [33]. 
Pakar ini menilai keluaran LLM menggunakan rubrik penilaian 
komprehensif yang didasarkan pada kerangka teoretis mapan 
seperti model Hattie dan Timperley (2007) \cite{HattieTimperley2007PowerOfFeedback}, 
Shute (2008) \cite{Shute2008FocusFormativeFeedback}, 
dan \textit{Self-Determination Theory} \cite{DeciRyan1985IntrinsicMotivation}. Penilaian ini mengukur 
tiga dimensi utama yakni \textit{feedback}, 
\textit{motivation support}, dan \textit{appreciation support} 
menggunakan skala Likert (1-5) untuk memastikan kualitas 
pedagogis dan psikologis dari umpan balik tersebut (Subbab[2.1.8]). 
Berikut adalah diagram alur metode evaluasinya.
\clearpage
\begin{figure}[H]
  \centering
  \includegraphics[width=\linewidth]{contents/chapter-3/Flowchart-Alur-Evaluasi-Kualitas-Umpan-Balik-LLM.png}
  \caption[Contoh gambar]{Diagram Alir Metode Evaluasi.}
  \label{Fig:diagram-alur-evaluasi}
\end{figure}
\clearpage
Terlihat dari Gambar \ref{Fig:diagram-alur-evaluasi}, alur evaluasi 
penelitian ini dimulai dengan generasi umpan balik LLM yang 
dipadukan dengan berbagai skema \textit{context engineering}. 
Umpan balik ini dihasilkan oleh dua model berbeda untuk 
perbandingan, yaitu Llama 3.1 8B \textit{Instant} dan 
Llama 3.3 70B \textit{Versatile}.

Luaran dari model-model ini pertama-tama melalui 
tahap Uji Kuantitatif. Sejalan dengan metodologi yang telah 
dijelaskan, tahap ini berfokus pada pengukuran kesamaan semantik 
dan relevansi Subbab \ref{subsection:tinjauan_evaluasi_kuantitatif} menggunakan tiga metrik utama: 
Uji BERTScore, BARTScore, dan GPT-4o as a Judge [22, 23, 24].

Luaran dari pengujian ini kemudian menjalani analisis 
hasil Uji Kuantitatif. Berdasarkan analisis ini, model 
terbaik dipilih—yaitu, kombinasi model dan teknik 
\textit{context engineering} yang menghasilkan skor keselarasan 
tertinggi terhadap \textit{ground truth} ahli.

Selanjutnya, umpan balik dari model terbaik tersebut dilanjutkan 
ke tahap evaluasi kedua, yaitu Uji Kualitatif. Tahap ini 
merupakan "standar emas" Subbab \ref{subsection:tinjauan_evaluasi_kualitatif} yang melibatkan peninjauan 
langsung oleh Ahli Pendidikan. Pakar ini menilai kualitas pedagogis 
dan psikologis dari umpan balik model terbaik menggunakan 
Kuesioner Skala Likert yang dirancang khusus berdasarkan kerangka 
teoretis [26, 27, 28]. Langkah terakhir adalah Analisis hasil uji 
kualitatif untuk mendapatkan kesimpulan akhir mengenai efektivitas 
umpan balik LLM dalam konteks SRL.
% Tolong bantu saya dalam menulis penjelasan gambar \textit{flowchart} 
% diatas dengan detail, sesuai dengan konteks paragraf nya. 


\section{Alur Tugas Akhir}
Dalam penelitian ini, terdapat sebuah alur kerja yang 
menjabarkan semua tahapan dan proses yang terlibat. 
Secara keseluruhan, proses penelitian terdiri dari 
empat tahap utama: tahap studi pendahuluan dan persiapan data, 
yang mencakup Tinjauan Pustaka dan 
Pembuatan \textit{Dataset} Papan Kanban Mahasiswa (termasuk 
pengumpulan \textit{ground truth} ahli); tahap desain eksperimen, 
yang meliputi Pemilihan Model LLMs dan perancangan Metode 
Peningkatan Hasil Umpan Balik (menggunakan Rekayasa Konteks, 
Integrasi \textit{Learning Analytics}, dan Pembuatan Variasi 
\textit{Prompt}); tahap implementasi, di mana dilakukan 
Pengambilan Data Umpan Balik LLM menggunakan metode yang telah 
dirancang; serta tahap evaluasi akhir, yang terdiri dari 
Pengujian Kuantitatif dan Kualitatif, diikuti oleh Analisis dan 
Evaluasi. Alur kerja penelitian secara keseluruhan diilustrasikan pada
Gambar \ref{Fig:flowchart-alur-penelitian}.
% Dalam penelitian ini, terdapat sebuah alur kerja yang 
% menjabarkan semua tahapan dan proses yang terlibat. Secara keseluruhan,
% proses penelitian....

\begin{figure}[H]
  \centering
  \includegraphics[height=24cm]{contents/chapter-3/Flowchart-Alur-Tugas-Akhir.png}
  \caption[Contoh gambar]{Flowchart Alur Penelitian Tugas Akhir.}
  \label{Fig:flowchart-alur-penelitian}
\end{figure}

% Menguraikan prosedur yang akan digunakan dan jadwal atau alur penyelesaian setiap 
% tahap. Alur penelian ini dapat disajikan dalam bentuk diagram. Diagram dapat disusun dengan aturan yang baik semisal menggunakan \textit{flowchart}. Aturan dan tutorial pembuatan \textit{flowchart} dapat dilihat di \textcolor{blue}{http://ugm.id/flowcharttutorial}. Setelah menggambarkannya, penulis wajib menjelaskan langkah-langkah setiap alur tugas akhir dalam sub bab tersendiri sesuai dengan kebutuhan.

\subsection{Tinjauan Pustaka}
Bab ini menyajikan tinjauan pustaka mendalam yang relevan dengan 
topik tesis. Tinjauan ini berfokus pada penelitian-penelitian 
sebelumnya mengenai peningkatan kualitas umpan balik 
LLM dalam konteks pendidikan, dengan menyelidiki dua arus 
utama yang teridentifikasi dari karya-karya kunci [Tabel 2.1]. 
Arus pertama adalah penyelarasan model (\textit{model alignment}) 
melalui \textit{fine-tuning}, seperti \textit{Direct Preference 
Optimization} (DPO) [13] dan \textit{Reinforcement Learning from 
Human Feedback} (RLHF) [16]. Arus kedua adalah penyelarasan konteks 
(\textit{context alignment}), yang mencakup rekayasa \textit{prompt} 
(\textit{prompt engineering}) berbasis teori [15] dan rekayasa konteks 
yang digabungkan dengan \textit{learning analytics} [17]. Tinjauan 
ini juga menganalisis pemanfaatan dasbor orkestrasi kelas berbasis 
Kanban [18] sebagai sumber data proses, bersama dengan metode 
evaluasi yang digunakan untuk menilai kinerja umpan balik, baik 
secara kuantitatif (misalnya, BERTScore [22], BARTScore [23], dan 
\textit{LLM-as-a-Judge} [24]) maupun secara kualitatif oleh ahli 
manusia sebagai standar emas [26, 27, 28].

Selanjutnya, analisis terhadap metodologi yang diterapkan dan 
temuan utama dalam studi-studi ini dilakukan. Tujuan utama dari 
analisis ini adalah untuk menilai secara kritis keunggulan dan 
keterbatasan dari setiap pendekatan. Evaluasi ini mengidentifikasi 
keunggulan (seperti efisiensi biaya dan kelayakan implementasi 
dari rekayasa konteks [15]) dan kelemahan (seperti tingginya 
kebutuhan sumber daya komputasi dan data untuk \textit{fine-tuning} 
[13, 16]). Temuan ini menjadi pertimbangan dalam pemilihan dan 
adaptasi metodologi yang paling tepat untuk menjawab masalah 
penelitian dalam karya ini: memanfaatkan pendekatan \textit{context 
engineering} yang ringan dan murah [15] namun diperkaya dengan 
data \textit{learning analytics} kaya sinyal yang diekstraksi 
dari papan Kanban [18] untuk mengatasi celah penelitian yang 
teridentifikasi.

\subsection{Pembuatan Dataset Papan Kanban Mahasiswa}
Pada bagian ini, dilakukan proses pembuatan \textit{Dataset} 
Papan Kanban Mahasiswa. Tahap ini mencakup beberapa langkah 
krusial, yaitu Pembuatan \textit{Prompt} , Pembuatan 
Data Sintetis yang akan menjadi data papan Kanban , serta Pengambilan 
data umpan balik pedagogis dari ahli manusia yang akan 
digunakan sebagai \textit{ground truth} untuk perbandingan.

\subsubsection{Pembuatan \textit{Prompt}}
Proses pembuatan \textit{prompt} dimulai dengan analisis 
mendalam terhadap struktur \textit{database} asli dari platform 
pembelajaran berbasis papan Kanban. Platform yang digunakan dan 
dikembangkan bernama Gamatutor yang dikembangkan oleh M. I. Azmi \cite{azmi2025pengembangan}. 
Berikut adalah tampilan dari
platform pembelajaran tersebut yang ditampilkan pada Gambar \ref{Fig:gamatutor_1}.

% Pengembangan kanban board untuk self regulated learning.

\begin{figure}[H]
  \centering
  \includegraphics[width=\linewidth]{contents/chapter-3/gamatutor_1.png}
  \caption[Contoh gambar]{Tampilan Utama Platform Gamatutor.}
  \label{Fig:gamatutor_1}
\end{figure}

Secara umum, Gamatutor adalah platform \textit{learning analytics} berbasis 
metode \textit{Kanban} yang dirancang untuk mendukung 
SRL mahasiswa. Platform ini memfasilitasi pengelolaan 
tugas belajar melalui sistem 
papan visual dengan empat tahapan utama yaitu \textit{Planning (To Do)}, 
\textit{Monitoring (In Progress)}, \textit{Controlling (Review)}, 
dan \textit{Reflection (Done)}. Setiap tugas direpresentasikan 
dalam bentuk \textit{kartu} yang memuat informasi lengkap seperti 
judul, mata kuliah, \textit{priority}, \textit{difficulty}, 
\textit{learning strategy}, nilai \textit{pre-test/post-test}, 
dan \textit{checklist} progres. Fitur utamanya meliputi pembuatan 
dan pengelolaan kartu tugas, pemantauan \textit{progress} melalui 
\textit{learning analytics}, serta manajemen profil pengguna.

Untuk satu kartunya merepresentasikan satu sub-topik dari salah satu mata kuliah.
Jadi mahasiswa bisa membuat banyak kartu untuk mata kuliah yang sama.
Untuk mengakses detail informasi-informasi tersebut, pengguna (mahasiswa)
dapat melihatnya pada tiap-tiap kartu dengan menekan kartunya. Tampilan 
dari informasinya dapat dilihat pada Gambar \ref{Fig:gamatutor_2}

\clearpage
\begin{figure}[H]
  \centering
  \includegraphics[width=\linewidth]{contents/chapter-3/gamatutor_2.png}
  \caption[Contoh gambar]{Tampilan Informasi Tiap Kartu Gamatutor.}
  \label{Fig:gamatutor_2}
\end{figure}
\clearpage

Semua detail informasi dan fitur-fitur tersebut dibuat untuk membantu
mahasiswa dalam menerapkan SRL (seperti: \textit{Start Timer}, 
\textit{Checklists}, dan "Add Link") serta secara bersamaan 
memberikan data proses yang kaya untuk \textit{learning analytics}. 
Data ini mencakup \textit{Total study time} (yang dihasilkan dari fitur 
\textit{Start Timer}), progres penyelesaian tugas melalui Checklists, 
hingga data metakognitif dan performa seperti \textit{Priority, 
Difficulty, Learning Strategy} yang dipilih, serta nilai \textit{Pre-test 
Grade} dan \textit{Post-test Grade}. Kumpulan data proses inilah yang dapat 
menjadi insight bagi agen pedagogis (guru ataupun \textit{AI chatbot}) untuk 
memantau, mendiagnosis hambatan, dan memberikan bimbingan yang 
terpersonalisasi sesuai dengan kondisi aktual mahasiswa tersebut.
Gamatutor juga memiliki \textit{Learning Assistant} dengan berbasiskan chatbot LLM 
yang rencananya akan diterapkan secara bertahap. Penelitian ini adalah 
bagian dari pengembangan \textit{Learning Asistant} tersebut.

Semua informasi mengenai pembelajaran mahasiswa dan kartu-kartu Kanban-nya
secara rapi disimpan ke dalam \textit{database} NoSQL yaitu MongoDB.
Data tersebut disimpan dengan mengikuti format \texttt{JSON}. 
Berikut adalah skema penyimpanan data kartu Kanban dalam MongoDB.


\begin{figure}[H]
  \centering
  \includegraphics[height=18cm]{contents/chapter-3/skema_db_kanban_board.png}
  \caption[Contoh gambar]{Skema Penyimpanan Data Kartu Pembelajaran Kanban Mahasiswa.}
  \label{Fig:schema_kanban_mongodb}
\end{figure}

Dapat dilihat pada Gambar \ref{Fig:schema_kanban_mongodb} terdapat banyak 
atribut, khususnya pada entitas \texttt{Card}, yang menyimpan data-data kunci 
untuk merefleksikan proses dan hasil belajar mahasiswa. Satu \texttt{Board}, hanya 
bisa dimiliki oleh satu mahasiswa saja di mana \texttt{Board} sendiri adalah
representasi dari keseluruhan papan pembelajaran berbasis Kanban.

Beberapa atribut yang paling penting untuk mengevaluasi kemampuan belajar 
ini meliputi \texttt{pre\_test\_grade} dan \texttt{post\_test\_grade}. 
Kedua atribut ini secara langsung mengukur pengetahuan mahasiswa sebelum dan 
sesudah mengerjakan tugas, sehingga selisih antara keduanya dapat menjadi 
indikator utama peningkatan pemahaman.
% (\textit{learning gain}).

Selain itu, atribut \texttt{difficulty} mengkategorikan tingkat kesulitan 
tugas, yang memungkinkan analisis performa mahasiswa terhadap tantangan yang 
diberikan. Atribut \texttt{column\_movement\_times} juga sangat kaya data 
dengan \texttt{timestamp} di dalamnya, data ini dapat digunakan untuk 
menghitung total durasi pengerjaan, waktu yang dihabiskan dalam fase belajar 
aktif (di kolom "In Progress"), atau bahkan mendeteksi adanya revisi. Terdapat juga
atribut \texttt{notes} yang berisikan catatan dari mahasiswa ketika kartu
sudah berada di tahap \textit{Controlling (Review)}. Isi catatan ini bisa menjadi
gambaran penting tentang rangkuman pembelajaran yang dilakukan mahasiswa terhadap 
sub-topik tersebut yang menjadikannya data yang cukup penting.

Atribut lain yang relevan adalah \texttt{checklists} untuk melihat progres 
granular pengerjaan sub-tugas, \texttt{learning\_strategy} untuk 
memahami pendekatan belajar yang dipilih mahasiswa, dan \texttt{rating} untuk
memberikan penilaian terhadap pengalaman belajar terhadap sub-topik yang dipelajari.
Kombinasi dari atribut-atribut inilah yang memberikan gambaran komprehensif 
mengenai pola dan kemampuan belajar mahasiswa.

Dari skema MongoDB diatas, 
dilakukan reduksi atribut untuk memfokuskan pada data yang paling 
esensial bagi penelitian, seperti \texttt{created\_at}, 
\texttt{checklists}, dan \texttt{column\_movements}. 
Selain itu, dilakukan reduksi ini untuk mempermudah pembuatan
data sintetis. Berikut adalah hasil skema database MongoDB setelah
dilakukan reduksi data yang ada pada Gambar \ref{Fig:schema_kanban_mongodb_reduksi}.

\begin{figure}[H]
  \centering
  \includegraphics[height=16cm]{contents/chapter-3/skema_db_kanban_board_reduksi.png}
  \caption[Contoh gambar]{Skema Penyimpanan Data Kartu Pembelajaran Kanban Mahasiswa Setelah Reduksi}
  \label{Fig:schema_kanban_mongodb_reduksi}
\end{figure}


% Untuk memastikan data sintetis yang dihasilkan dapat merepresentasikan 
% variasi perilaku mahasiswa di dunia nyata, lima kategori profil 
% (Malas, Sedikit Malas, Cukup, Rajin, dan Sangat Rajin) dirumuskan. 
Untuk mencapai tingkat representasi yang valid, data tersebut harus mampu 
menangkap berbagai pola perilaku dan tingkat keterlibatan mahasiswa yang 
berbeda-beda, mulai dari yang paling tidak aktif hingga yang sangat proaktif.
Sebagai strategi utama untuk mengimplementasikan keragaman ini, sebuah 
studi kasus telah dirumuskan yang berfokus pada pengkategorian mahasiswa ke 
dalam profil-profil perilaku yang spesifik.
Melalui studi kasus ini, telah diidentifikasi dan ditetapkan lima kategori 
profil yang berbeda, di mana setiap kategori mewakili satu titik pada 
spektrum tingkat kerajinan mahasiswa.
Kelima kategori yang komprehensif ini yang diberi label \texttt{"MALAS"}, \texttt{"Sedikit 
Malas"}, \texttt{"Cukup"}, \texttt{"Rajin"}, dan \texttt{"Sangat Rajin"}, akan digunakan sebagai dasar 
untuk menghasilkan set data yang bervariasi.

Untuk menyederhanakan penelitian ini, penggunaan asumsi akan digunakan dalam
memberikan label kategori tingkat performa dan perilaku mahasiswa.
Variabel atribut yang digunakan antara lain adalah jumlah total \texttt{card}
yang dibuat oleh mahasiswa, tingkat penyelesaian tugas (\texttt{checklist}),
aktivitas terkini dari kartu tersebut pada \texttt{column\_movements}, dan
letak keberadaan \texttt{card} tersebut. Hal ini diharapkan dapat menjadi proksi 
yang \textit{memadai dan terukur} untuk membedakan secara jelas antara kelima 
profil perilaku dan performa tersebut, sehingga penyederhanaan ini tetap 
dapat menghasilkan data yang representatif.

Aturan-aturan tersebut dirumuskan dan diimplementasikan pada studi kasus
seperti jumlah total kartu, persentase 
penyelesaian \textit{checklist}, dan stempel waktu pergerakan kartu
yang secara implementasinya dapat dilihat pada Tabel \ref{tab:matriks_studi_kasus_tabularx}.

% Pastikan paket-paket ini ada di preamble Anda:
% \usepackage{booktabs}
% \usepackage{array}
% \usepackage{float}

% \begin{table}[H]
%   \centering
%   \caption{Matriks Aturan Studi Kasus Perilaku Mahasiswa}
%   \label{tab:matriks_studi_kasus}
%   % Menggunakan @{} untuk menghapus spasi ekstra di sisi tabel
%   % Kolom 'l' untuk rata kiri, 'p{width}' untuk paragraf dengan text-wrap
%   \begin{tabular}{@{} l l p{3cm} p{3.5cm} p{4.5cm} @{}}
%     \toprule
%     \textbf{Kategori} & \textbf{Studi Kasus} & \textbf{Jumlah Kartu} & \textbf{Penyelesaian Checklist (per kartu)} & \textbf{Aturan Aktivitas \& Posisi Kartu} \\
%     \midrule

%     \textbf{Malas} & MHS\_MALAS\_SK1 & 1 - 2 kartu & 0 - 2 item selesai (dari 3) & \textbf{Tidak ada aktivitas} (2 jam terakhir). \newline Kartu cenderung di list1 atau list2. \\
%     \midrule

%     \textbf{Sedikit Malas} & MHS\_SEDMALAS\_SK1 & 3 - 4 kartu & 0 - 2 item selesai (dari 3) & \textbf{Tidak ada aktivitas} (2 jam terakhir). \\
%     \addlinespace % Menambah sedikit spasi antar sub-grup
%     & MHS\_SEDMALAS\_SK2 & 3 - 4 kartu & 1 - 2 item selesai (dari 3) & \textbf{Tidak ada aktivitas} (2 jam terakhir). \\
%     \midrule

%     \textbf{Cukup} & MHS\_CUKUP\_SK1 & 4 - 5 kartu & 1 - 2 item selesai (dari 3) & \textbf{Ada aktivitas} (2 jam terakhir) untuk min. 3 kartu. \\
%     \addlinespace
%     & MHS\_CUKUP\_SK2 & 3 - 4 kartu & Semua 3 item selesai & \textbf{Ada aktivitas} (2 jam terakhir) untuk min. 3 kartu. \\
%     \midrule

%     \textbf{Rajin} & MHS\_RAJIN\_SK1 & Tepat 5 kartu & 2 - 3 item selesai (dari 3) & Hanya 1 kartu tersisa di list1 ("Planning (To Do)"). \\
%     \addlinespace
%     & MHS\_RAJIN\_SK2 & 4 - 5 kartu & Semua 3 item selesai & Hanya 1 kartu tersisa di list1 ("Planning (To Do)"). \\
%     \midrule

%     \textbf{Sangat Rajin} & MHS\_SANGRAJIN\_SK1 & Tepat 5 kartu & Semua 3 item selesai & Semua 5 kartu berada di list4 ("Reflection (Done)"). \\
%     \bottomrule
%   \end{tabular}
% \end{table}

\begin{table}[H]
  \centering
  \caption{Tabel Aturan Studi Kasus Perilaku Mahasiswa}
  \label{tab:matriks_studi_kasus_tabularx}

  \begingroup
  \setlength{\tabcolsep}{8pt}        % Mengatur spasi antar kolom
  \renewcommand{\arraystretch}{1.5} % Mengatur tinggi baris
  \scriptsize                        % <<< Menggunakan font \scriptsize
  
  % Menggunakan tabularx agar lebar tabel = \textwidth
  % Kolom: l (rata kiri), l (rata kiri), l (rata kiri), Y (fleksibel), Y (fleksibel)
  \begin{tabularx}{\textwidth}{@{} l l l Y Y @{}}
    \toprule
    \textbf{Kategori} & \textbf{Studi Kasus} & \textbf{Jumlah Kartu} & \textbf{Penyelesaian Checklist (per kartu)} & \textbf{Aturan Aktivitas \& Posisi Kartu} \\
    \midrule

    \textbf{Malas} & MHS\_MALAS\_SK1 & 1 - 2 kartu & 0 - 2 item selesai (dari 3) & \textbf{Tidak ada aktivitas} (2 jam terakhir). Kartu cenderung di list1 atau list2. \\
    \midrule

    \textbf{Sedikit Malas} & MHS\_SEDMALAS\_SK1 & 3 - 4 kartu & 0 - 2 item selesai (dari 3) & \textbf{Tidak ada aktivitas} (2 jam terakhir). \\
    \addlinespace % Menambah sedikit spasi antar sub-grup
    & MHS\_SEDMALAS\_SK2 & 3 - 4 kartu & 1 - 2 item selesai (dari 3) & \textbf{Tidak ada aktivitas} (2 jam terakhir). \\
    \midrule

    \textbf{Cukup} & MHS\_CUKUP\_SK1 & 4 - 5 kartu & 1 - 2 item selesai (dari 3) & \textbf{Ada aktivitas} (2 jam terakhir) untuk min. 3 kartu. \\
    \addlinespace
    & MHS\_CUKUP\_SK2 & 3 - 4 kartu & Semua 3 item selesai & \textbf{Ada aktivitas} (2 jam terakhir) untuk min. 3 kartu. \\
    \midrule

    \textbf{Rajin} & MHS\_RAJIN\_SK1 & Tepat 5 kartu & 2 - 3 item selesai (dari 3) & Hanya 1 kartu tersisa di list1 ("Planning (To Do)"). \\
    \addlinespace
    & MHS\_RAJIN\_SK2 & 4 - 5 kartu & Semua 3 item selesai & Hanya 1 kartu tersisa di list1 ("Planning (To Do)"). \\
    \midrule

    \textbf{Sangat Rajin} & MHS\_SANGRAJIN\_SK1 & Tepat 5 kartu & Semua 3 item selesai & Semua 5 kartu berada di list4 ("Reflection (Done)"). \\
    \bottomrule
  \end{tabularx}
  
  \endgroup
\end{table}



Kategori-kategori ini kemudian dijadikan dasar untuk membuat 
\textit{prompt} generator data sintetis pembelajaran mahasiswa berbasis Kanban.
Kumpulan aturan yang telah dirangkum dalam Tabel 
\ref{tab:matriks_studi_kasus_tabularx} tersebut kemudian ditransformasi 
menjadi sebuah \textit{prompt} instruksi yang terstruktur. 
Proses transformasi ini melibatkan penerjemahan setiap studi 
kasus—mulai dari MHS\_MALAS\_SK1 hingga MHS\_SANGRAJIN\_SK1, dari format 
tabel tabular menjadi format teks naratif. Tujuan utamanya adalah 
untuk menyajikan aturan-aturan tersebut dalam format bahasa alami 
(\textit{natural language}) yang dapat dipahami dan dieksekusi 
secara presisi oleh model LLM pembuat data tersebut. \textit{Prompt} 
ini secara eksplisit merinci batasan untuk setiap variabel, 
termasuk jumlah kartu yang harus dibuat, syarat penyelesaian 
\textit{checklist}, dan aturan mengenai aktivitas pergerakan kartu.
Terdapat sedikit penambahan aturan pada pembuatan \textit{prompt}
yaitu adanya waktu kapan dibuatnya kartu sub-topik tersebut.
Contoh \textit{prompt} lengkap yang telah diformulasikan untuk 
agen generator tersebut dapat dilihat secara rinci pada potongan
snipet pada Tabel \ref{tab:prompt_studi_kasus_generator}.
% \ref{Fig:contoh_prompt}.

% \begin{verbatim}
% This text contains \LaTeX commands like \textbf{bold} and special characters like % and $.
% It will be printed exactly as typed, with a monospaced font.
% \end{verbatim}



% Pastikan preamble memuat:
% \usepackage{xcolor}
% \usepackage{listings}

% \lstset{...} (Ini bisa ditaruh di preamble atau di sini)
\lstset{
    language=Python,
    basicstyle={\fontsize{10}{8}\selectfont\color{black}\ttfamily\bfseries},
    keywordstyle=\color{black}\bfseries,
    commentstyle=\color{black},
    stringstyle=\color{black},
    breaklines=true,
    showstringspaces=false
}

%--- MULAI LINGKUNGAN TABEL ---
\begin{table}[H] % Anda bisa ganti [H] dengan [htbp] jika perlu
  \centering
  
  % Caption dan Label sekarang milik "table", bukan "lstlisting"
  \caption{\textit{Prompt} untuk Studi Kasus Data Sintetis Mahasiswa}
  \label{tab:prompt_studi_kasus_generator} % Saya ubah labelnya menjadi "tab:"
  
  % Gunakan minipage untuk "membungkus" listing dengan aman
  \begin{minipage}{\textwidth}
    
    % Mulai listing Anda (tanpa caption/label di sini)
    \begin{lstlisting}
"""

Kategori: Malas

* Aturan untuk Studi Kasus 1 (MHS_MALAS_SK1):

* Buat 1 papan JSON dengan 1 atau 2 kartu.

* Setiap kartu memiliki 3 checklist, dengan total 0 hingga 2 item yang dicentang (completed: true) per kartu.

* Tidak ada kartu yang berpindah tempat dalam kurun waktu 2 jam terakhir (semua timestamp created_at atau column_movements terakhir harus <= 2025-06-07T15:00:00Z). Kartu cenderung berada di list1 atau list2.



Kategori: Sedikit Malas

* Aturan untuk Studi Kasus 1 (MHS_SEDMALAS_SK1):

* Buat 1 papan JSON dengan 3 atau 4 kartu.

* Setiap kartu memiliki 3 checklist, dengan total 0 hingga 2 item yang dicentang per kartu.

* Tidak ada kartu yang berpindah tempat dalam kurun waktu 2 jam terakhir.

* Aturan untuk Studi Kasus 2 (MHS_SEDMALAS_SK2):

* Buat 1 papan JSON dengan 3 atau 4 kartu.

* Setiap kartu memiliki 3 checklist, dengan total 1 hingga 2 item yang dicentang per kartu.

* Tidak ada kartu yang berpindah tempat dalam kurun waktu 2 jam terakhir.


# .... Diatas adalah potongan yang tidak lengkap dari kode yang utuh. ....

"""
\end{lstlisting}
%--- Selesai listing ---

  \end{minipage} 
  %--- Selesai minipage ---

\end{table}
%--- SELESAI LINGKUNGAN TABEL ---
\clearpage
Selanjutnya, untuk memperkecil topik pembelajaran yang akan dibuat pada
data sintetisnya, dibatasi satu mata kuliah saja dengan 5 sub-topik yang
dapat diambil. Pembuatan batasan sub-topik harus memiliki aturan yang
perlu diberikan ke model LLM pembuat data sintetis tersebut. Hal ini agar
model dapat memberikan data luaran yang baik serta variasi yang beragam
dengan batasan yang jelas. Tabel \ref{tab:ringkasan_topik_oop} menunjukkan batasan sub-topik
yang akan diberikan pada data sintetis.

\begin{table}[H]
  \centering
  \caption{Ringkasan Topik OOP, Tujuan, dan Checklist}
  \label{tab:ringkasan_topik_oop}

  \begingroup
  \setlength{\tabcolsep}{8pt}
  \renewcommand{\arraystretch}{1.5}
  \scriptsize

  % Kolom fleksibel Y dan kolom fixed-width P
  \newcolumntype{Y}{>{\raggedright\arraybackslash}X}
  \newcolumntype{P}[1]{>{\raggedright\arraybackslash}p{#1}}

  % Ubah 'l Y Y' -> 'P{0.20\textwidth} Y Y'
  \begin{tabularx}{\textwidth}{@{} P{0.20\textwidth} Y Y @{}}
    \toprule
    \textbf{Topik} & \textbf{Tujuan} & \textbf{Checklist Items} \\
    \midrule

    \textbf{Topik 1: Konsep Dasar dan Pilar OOP}
    &
    Memahami fondasi dari OOP dan 4 pilar utamanya.
    &
    \begin{itemize}
      \item Menjelaskan perbedaan paradigma pemrograman prosedural dan OOP.
      \item Mengidentifikasi serta mendeskripsikan 4 pilar OOP: enkapsulasi, abstraksi, pewarisan, polimorfisme.
      \item Menerapkan konsep dasar objek dan kelas dalam program sederhana.
    \end{itemize}
    \\ \addlinespace
    \midrule

    \textbf{Topik 2: Perancangan Kelas dan Interaksi Objek}
    &
    Mampu merancang struktur kelas dan interaksinya secara logis.
    &
    \begin{itemize}
      \item Mendesain \textit{class diagram} berdasarkan kebutuhan dari studi kasus.
      \item Menentukan relasi antar objek: asosiasi, agregasi, komposisi.
      \item Mengimplementasikan rancangan interaksi objek dalam bentuk kode.
    \end{itemize}
    \\ \addlinespace
    \midrule

    \textbf{Topik 3: Enkapsulasi, Inheritance, dan Polimorfisme}
    &
    Menguasai teknik pewarisan dan fleksibilitas perilaku objek.
    &
    \begin{itemize}
      \item Menerapkan enkapsulasi menggunakan modifier akses (private, public, protected).
      \item Membuat kelas turunan dan meng-\textit{override} metode dari \textit{superclass}.
      \item Menggunakan polimorfisme melalui \textit{overloading} dan \textit{overriding} metode.
    \end{itemize}
    \\ \addlinespace
    \midrule

    \textbf{Topik 4: Prinsip SOLID dan Adaptive Code}
    &
    Menulis kode yang fleksibel, \textit{maintainable}, dan \textit{scalable}.
    &
    \begin{itemize}
      \item Menjelaskan dan memberi contoh penerapan dari masing-masing prinsip SOLID.
      \item Melakukan \textit{refactoring} kode agar sesuai dengan prinsip SOLID.
      \item Menerapkan prinsip desain adaptif dalam proyek mini berbasis OOP.
    \end{itemize}
    \\ \addlinespace
    \midrule

    \textbf{Topik 5: Kolaborasi Objek dan Pengorganisasian Kode}
    &
    Menyusun kode OOP secara terstruktur dan profesional.
    &
    \begin{itemize}
      \item Mengorganisasi struktur folder proyek OOP berdasarkan lapisan fungsional.
      \item Mengimplementasikan interaksi antar objek menggunakan \textit{interface} atau \textit{abstract class}.
      \item Mengembangkan aplikasi kecil dengan arsitektur OOP yang modular.
    \end{itemize}
    \\

    \bottomrule
  \end{tabularx}
  \endgroup
\end{table}

Tabel \ref{tab:ringkasan_topik_oop} merincikan batasan konten 
pembelajaran yang digunakan dalam penelitian ini. Fokus mata kuliah 
dikerucutkan pada Pemrograman Berbasis Objek (\textit{Object Oriented 
Programming} atau OOP). Tabel tersebut menjabarkan lima sub-topik 
utama yang akan digunakan sebagai materi kartu tugas, mulai dari 
"Konsep Dasar dan Pilar OOP" hingga "Kolaborasi Objek". Untuk setiap 
sub-topik, dirinci pula tujuan pembelajaran serta daftar \textit{checklist 
items} yang spesifik. Rincian \textit{checklist} ini berkaitan dengan atribut
\texttt{checklist} berfungsi sebagai 
representasi tugas-tugas mendetail yang harus diselesaikan mahasiswa. 
Seluruh batasan konten terstruktur inilah yang kemudian 
diterjemahkan menjadi bagian dari \textit{prompt} instruksi yang 
diberikan kepada model LLM pembuat data sintetis, guna memastikan 
setiap \textit{dataset} yang dihasilkan tetap relevan dan fokus pada 
materi pembelajaran yang telah ditentukan. \textit{Prompt} dari tabel tersebut 
dapat dilihat pada Tabel \ref{tab:prompt_batasab_sub}.

\begin{table}[H] % Anda bisa ganti [H] dengan [htbp] jika perlu
  \centering
  
  % Caption dan Label sekarang milik "table", bukan "lstlisting"
  \caption{\textit{Prompt} untuk Batasan Sub-topik Materi Pembelajaran Mahasiswa}
  \label{tab:prompt_batasan_sub} % Saya ubah labelnya menjadi "tab:"
  
  % Gunakan minipage untuk "membungkus" listing dengan aman
  \begin{minipage}{\textwidth}
    
    % Mulai listing Anda (tanpa caption/label di sini)
    \begin{lstlisting}
* Topik 1: Konsep Dasar dan Pilar OOP
	* Tujuan: Memahami fondasi dari OOP dan 4 pilar utamanya.
	* Checklist Items:
		1. "Menjelaskan perbedaan paradigma pemrograman prosedural dan OOP."
		2. "Mengidentifikasi serta mendeskripsikan 4 pilar OOP: Enkapsulasi, Abstraksi, Pewarisan, Polimorfisme."
		3. "Menerapkan konsep dasar objek dan kelas dalam program sederhana."
* Topik 2: Perancangan Kelas dan Interaksi Objek
	* Tujuan: Mampu merancang struktur kelas dan interaksinya secara logis.
	* Checklist Items:
		1. "Mendesain class diagram berdasarkan kebutuhan dari studi kasus."
		2. "Menentukan relasi antar objek: asosiasi, agregasi, komposisi."
		3. "Mengimplementasikan rancangan interaksi objek dalam bentuk kode."
* Topik 3: Enkapsulasi, Inheritance, dan Polimorfisme
	* Tujuan: Menguasai teknik pewarisan dan fleksibilitas perilaku objek.
	* Checklist Items:
		1. "Menerapkan enkapsulasi menggunakan modifier akses (private, public, protected)."
		2. "Membuat kelas turunan dan meng-overriding metode dari superclass."
		3. "Menggunakan polimorfisme melalui overloading dan overriding metode."
* Topik 4: Prinsip SOLID dan Adaptive Code
	* Tujuan: Menulis kode yang fleksibel, maintainable, dan scalable.
	* Checklist Items:
		1. "Menjelaskan dan memberi contoh penerapan dari masing-masing prinsip SOLID."
		2. "Melakukan refactoring kode agar sesuai dengan prinsip SOLID."
		3. "Menerapkan prinsip desain adaptif dalam proyek mini berbasis OOP."
* Topik 5: Kolaborasi Objek dan Pengorganisasian Kode
	* Tujuan: Menyusun kode OOP secara terstruktur dan profesional.
	* Checklist Items:
		1. "Mengorganisasi struktur folder proyek OOP berdasarkan lapisan fungsional."
		2. "Mengimplementasikan interaksi antar objek menggunakan interface atau abstract class."
		3. "Mengembangkan aplikasi kecil dengan arsitektur OOP yang modular."
\end{lstlisting}
%--- Selesai listing ---

  \end{minipage} 
  %--- Selesai minipage ---

\end{table}

% Aturan skema dan kategori perilaku ini kemudian disusun menjadi 
% sebuah \textit{role-play prompt} yang komprehensif untuk diumpankan 
% ke model generator (ChatGPT). \textit{Prompt} ini menginstruksikan 
% model untuk berperan sebagai "AI yang ahli dalam pembuatan data 
% sintetis" dan menggabungkan teknik \textit{Chain-of-Thought} 
% (CoT) serta \textit{Few-shot} (contoh) untuk memandu LLM menghasilkan 
% 40 \textit{dataset} papan Kanban dalam format JSON  yang valid dan 
% sesuai dengan profil perilaku yang diminta.
Setelah \textit{prompt} studi kasus mahasiswa telah selesai dibuat,
tahapan selanjutnya adalah pembuatan \textit{prompt} intruksi yang
spesifik mengarah pada bentuk hasil data sintetis yang akan dibuat.
Instruksi \textit{prompt} yang digunakan yaitu \textit{role-play}, 
\textit{Few-shot}, dan \textit{Chain-of-Thought} (CoT).

Pada awal \textit{prompt} digunakan terlebih dahulu \textit{role-play prompt}. 
Strategi \textit{prompting} ini secara spesifik menginstruksikan 
model untuk mengadopsi persona sebagai "AI yang ahli dalam 
pembuatan data sintetis". Penetapan peran (\textit{role-play}) ini 
penting untuk mengatur konteks dan ekspektasi, mendorong 
model agar tidak hanya memberikan jawaban umum, tetapi 
bertindak sebagai pakar domain yang memahami nuansa validasi 
skema JSON dan logika perilaku. Berikut adalah potongan
\textit{prompt} dipaparkan pada Tabel \ref{tab:prompt_role_play}.

\begin{table}[H] % Anda bisa ganti [H] dengan [htbp] jika perlu
	\centering
	
	% Caption dan Label sekarang milik "table", bukan "lstlisting"
	\caption{\textit{Role-Play Prompt} pada Pembuatan Data Sintetis}
	\label{tab:prompt_role_play} % Saya ubah labelnya menjadi "tab:"
  
  % Gunakan minipage untuk "membungkus" listing dengan aman
	\begin{minipage}{\textwidth}
		
		% Mulai listing Anda (tanpa caption/label di sini)
		\begin{lstlisting}
		"""
		Anda adalah AI yang ahli dalam pembuatan data sintetis. Tugas Anda adalah menghasilkan serangkaian data papan Kanban dalam format JSON. Setiap objek JSON mewakili satu papan Kanban milik seorang mahasiswa, yang mencerminkan profil perilaku tertentu. Anda akan menghasilkan objek JSON papan Kanban yang berbeda, sesuai dengan aturan spesifik untuk 5 kategori mahasiswa.

		% Konteks Umum (Gunakan untuk semua data yang dihasilkan):
		"""
		\end{lstlisting}
		%--- Selesai listing ---

	\end{minipage} 
	%--- Selesai minipage ---

\end{table}

\textit{Prompt} yang ditunjukkan pada Tabel \ref{tab:prompt_role_play}
mengatur kerangka kerja dan identitas dari model pembuat Data Sintetis. 
Instruksi tersebut dimulai dengan penetapan peran yang spesifik sebagai 
"AI yang ahli dalam pembuatan data sintetis". Penugasan peran 
ini sangat penting untuk mengarahkan model agar bertindak sebagai 
pakar domain alih-alih asisten umum. Tugas utamanya kemudian 
didefinisikan secara eksplisit yaitu menghasilkan data papan Kanban 
dengan format luaran wajib berupa JSON. \textit{Prompt} ini juga 
secara cerdas membangun koneksi ke konteks penelitian dengan 
menyatakan bahwa setiap objek JSON mewakili satu mahasiswa dan 
harus mencerminkan profil perilaku tertentu. Selain itu instruksi ini 
mempersiapkan model untuk keragaman tugas dengan menyebutkan 
bahwa akan ada variasi data berdasarkan lima kategori mahasiswa 
yang spesifik. Kalimat terakhir "Konteks Umum" berfungsi sebagai 
penjembatan yang menandakan bahwa aturan-aturan berikutnya yang akan 
diberikan bersifat global atau berlaku untuk semua data yang akan 
dihasilkan.

Selanjutnya, untuk lebih memandu model dan memastikan kepatuhan terhadap 
aturan yang kompleks, \textit{prompt}-nya diperkaya dengan 
dua teknik lanjutan yaitu \textit{Few-shot} dan \textit{Chain-of-Thought}. 
Teknik \textit{Few-shot} digunakan dengan 
menyertakan beberapa contoh luaran \texttt{JSON} yang sudah jadi. 
Ini bertujuan untuk memberi model pemahaman konkret mengenai format dan 
struktur data yang diharapkan. Dalam konteks pembuatan data sintetis,
contoh luaran JSON yang diberikan ke model LLM adalah struktur objek papan 
Kanban (\texttt{Board}) dan struktur objek kartu (\texttt{Card}). Dapat dilihat
pada Tabel \ref{tab:prompt_few_shot} bentuk struktur JSON yang 
diberikan ke model.

\lstset{
    language=Java,
    basicstyle={\fontsize{8}{8}\selectfont\ttfamily\bfseries},
    keywordstyle=\color{black}\bfseries,
    commentstyle=\color{black},
    stringstyle=\color{black},
    breaklines=true,
    showstringspaces=false
}

\begin{table}[H] % Anda bisa ganti [H] dengan [htbp] jika perlu
	\centering
	
	% Caption dan Label sekarang milik "table", bukan "lstlisting"
	\caption{\textit{Few-Shot Prompt} pada Pembuatan Data Sintetis}
	\label{tab:prompt_few_shot} % Saya ubah labelnya menjadi "tab:"
  
  % Gunakan minipage untuk "membungkus" listing dengan aman
	\begin{minipage}{\textwidth}
		
		% Mulai listing Anda (tanpa caption/label di sini)
		\begin{lstlisting}
			// Struktur Objek Papan Kanban (Board):
			{
				"_id": { "$oid": "ID_BOARD_UNIK" },
				"user_id": { "$oid": "ID_USER_UNIK" },
				"name": "NAMA_PAPAN_KANBAN", // e.g., "Papan OOP - MHS Malas SK1"
				"lists": [
					{
					"id": "list1",
					"title": "Planning (To Do)",
					"cards": [ /* Kartu yang saat ini ada di list ini */ ]
					},
					{
					"id": "list2",
					"title": "Monitoring (In Progress)",
					"cards": [ /* Kartu yang saat ini ada di list ini */ ]
					},
					{
					"id": "list3",
					"title": "Controlling (Review)",
					"cards": [ /* Kartu yang saat ini ada di list ini */ ]
					},
					{
					"id": "list4",
					"title": "Reflection (Done)",
					"cards": [ /* Kartu yang saat ini ada di list ini */ ]
					}
				]
			}

			// Struktur Objek Kartu (Card):
			{
				"id": "ID_KARTU_UNIK", // e.g., "CS101-Konsep_Dasar_OOP-MALAS_SK1_1"
				"title": "Object Oriented Programming [CS101]", // Tetap
				"sub_title": "SUB_TOPIK_DARI_DAFTAR_DI_ATAS",
				"created_at": "YYYY-MM-DDTHH:MM:SSZ",
				"description": "Tujuan dari topik yang bersangkutan.",
				"deleted": false, "archived": false,
				"checklists": [
					{
					"id": "ID_CHECKLIST_UNIK",
					"title": "Tugas untuk [sub_title]",
					"items": [
						{ "id": "ID_ITEM_UNIK_1", "text": "Teks item checklist 1 dari daftar di atas", "completed": true/false },
						{ "id": "ITEM_UNIK_2", "text": "Teks item checklist 2 dari daftar di atas", "completed": true/false },
						{ "id": "ITEM_UNIK_3", "text": "Teks item checklist 3 dari daftar di atas", "completed": true/false }
					]
					}
				],
				"column_movement_times": { "list_id": "timestamp_terakhir_masuk_list_tsb" },
				"column_movements": [ { "fromColumn": "id_asal", "toColumn": "id_tujuan", "timestamp": "timestamp_gerak" } ]
			}


		\end{lstlisting}
		%--- Selesai listing ---

	\end{minipage} 
	%--- Selesai minipage ---

\end{table}

Struktur-struktur \texttt{JSON} memberikan instruksi 
seperti cetak biru (\textit{blueprint}) yang sangat eksplisit 
bagi model LLM. Struktur \texttt{Board} menetapkan skema level atas 
(\textit{top-level}), yang mendefinisikan kontainer utama dan 
empat kolom statis yang wajib ada, mulai dari \texttt{"Planning (To Do)"} 
hingga \texttt{"Reflection (Done)"}. Sementara itu, struktur \texttt{Card} 
memberikan arahan yang jauh lebih granular dan krusial bagi penelitian. 
Struktur ini secara detail menginstruksikan model tentang 
atribut-atribut data apa saja yang harus ada di dalam setiap unit 
tugas pembelajaran, termasuk penegasan bahwa \texttt{sub\_title} 
dan \texttt{items} pada \texttt{checklists} harus diambil dari 
daftar topik yang telah disediakan sebelumnya. Intinnya adalah, 
contoh ini menunjukkan secara presisi di mana variabel-variabel kunci 
untuk aturan ini harus ditempatkan, seperti status 
\texttt{completed} pada setiap item \textit{checklist} dan data 
\texttt{timestamp} pada atribut \texttt{column\_movements}.

Instruksi \textit{Few-shot prompt} juga ditambahkan informasi mengenai
struktur dari papan Kanban. Informasi ini diberikan agar model dapat 
mengerti struktur Kanban yang sedang dikerjakan serta dapat memberikan
penamaan kolom dengan benar. Berikut adalah \textit{prompt} informasi 
struktur Kanban yang diberikan pada Tabel \ref{}

\lstset{
    language=Python,
    basicstyle={\fontsize{10}{10}\selectfont\ttfamily\bfseries},
    keywordstyle=\color{black}\bfseries,
    commentstyle=\color{black},
    stringstyle=\color{black},
    breaklines=true,
    showstringspaces=false
}

\begin{table}[H] % Anda bisa ganti [H] dengan [htbp] jika perlu
	\centering
	
	% Caption dan Label sekarang milik "table", bukan "lstlisting"
	\caption{\textit{Prompt} Struktur Papan Kanban}
	\label{tab:prompt_few_shot} % Saya ubah labelnya menjadi "tab:"
  
  % Gunakan minipage untuk "membungkus" listing dengan aman
	\begin{minipage}{\textwidth}
		
		% Mulai listing Anda (tanpa caption/label di sini)
		\begin{lstlisting}
		Struktur List Kanban (Kolom pada Papan):
		* list1: "Planning (To Do)"
		* list2: "Monitoring (In Progress)"
		* list3: "Controlling (Review)"
		* list4: "Reflection (Done)"
		\end{lstlisting}
		%--- Selesai listing ---

	\end{minipage} 
	%--- Selesai minipage ---

\end{table}


Kemudian, jenis \textit{prompt} yang digunakan adalah 
\textit{Chain-of-Thought} (CoT). Teknik ini mengharuskan model 
untuk berpikir langkah penalaran proses logisnya 
sebelum menghasilkan luaran akhir. Hal ini 
sangat penting untuk memastikan model secara sadar memeriksa 
dan menerapkan aturan \textit{prompt }untuk setiap profil perilaku.
Potongan \textit{prompt} yang menggunakan jenis CoT ada pada
Tabel \ref{tab:prompt_cot}.

\begin{table}[H] % Anda bisa ganti [H] dengan [htbp] jika perlu
  \centering
  
  % Caption dan Label sekarang milik "table", bukan "lstlisting"
  \caption{CoT \textit{Prompt} pada Pembuatan Data Sintetis}
  \label{tab:prompt_cot} % Saya ubah labelnya menjadi "tab:"
  
  % Gunakan minipage untuk "membungkus" listing dengan aman
  \begin{minipage}{\textwidth}
    
    % Mulai listing Anda (tanpa caption/label di sini)
    \begin{lstlisting}
"""Contoh Logika Penerapan Aturan: Untuk MHS_CUKUP_SK1, Anda bisa membuat 4 kartu. 3 di antaranya harus memiliki column_movements terbaru dengan timestamp misalnya 2025-06-07T16:30:00Z. Kartu-kartu ini bisa jadi pindah dari list1 ke list2 atau list2 ke list3. Kartu ke-4 bisa jadi tidak bergerak dalam 2 jam terakhir. Untuk setiap kartu, jumlah completed: true pada checklistnya adalah 1 atau 2.

Petunjuk Tambahan untuk LLM: 
Pastikan semua id unik. Untuk created_at kartu, gunakan tanggal dan waktu yang memenuhi syarat "tidak berpindah selama 2 jam" sebelum 2025-06-02T17:00:00Z. Misalnya, 2025-06-02T08:00:00Z, 2025-06-02T10:30:00Z, 2025-06-02T14:59:00Z, dll. Tempatkan objek kartu di dalam array cards pada list yang sesuai dengan tahapnya saat ini (untuk kasus "Malas" ini, semua akan ada di list1). Pastikan column_movement_times dan column_movements secara akurat mencerminkan (tidak adanya) pergerakan."""
\end{lstlisting}
%--- Selesai listing ---

  \end{minipage} 
  %--- Selesai minipage ---

\end{table}

Tabel \ref{tab:prompt_cot} memberikan "Contoh 
Logika Penerapan Aturan" yang menelusuri proses 
berpikir untuk satu studi kasus spesifik (MHS\_CUKUP\_SK1). 
Contoh ini menguraikan bagaimana model harus mengalokasikan 
jumlah kartu, bagaimana menginterpretasikan aturan "perpindahan 
dalam 2 jam terakhir" secara konkret, dan bagaimana menerapkan 
status penyelesaian \textit{checklist}.

Lalu, bagian "Petunjuk Tambahan" menetapkan batasan-batasan 
kritis yang sering kali ambigu bagi LLM. \textit{Prompt} ini 
secara eksplisit mengklarifikasi logika penerapan \textit{timestamp}, 
terutama untuk kasus "tidak ada aktivitas" dengan memberikan 
contoh-contoh stempel waktu yang valid (yaitu, sebelum jam batas 
yang ditentukan). Selain itu, instruksi ini juga sangat menekankan 
pada aspek integritas data. Model diperintahkan untuk memastikan 
bahwa lokasi final kartu di dalam sebuah \textit{list} harus 
secara logis konsisten dengan riwayat yang tercatat pada 
atribut \texttt{column\_movements} dan \texttt{column\_movement\_times}, 
sehingga data yang dihasilkan tidak kontradiktif.

Setelah pembuatan \textit{prompt}, dilakukan penggabungan semua instruksi
\textit{prompt} tersebut dengan menambahkan beberapa tambahan instruksi 
untuk merapikan serta memperhalus kesinambungan antar \textit{prompt}.
Hasil akhirnya dapat dilihat pada Gambar \ref{Fig:prompt_pembuatan}.

\begin{figure}[H]
  \centering
  \includegraphics[width=\linewidth]{contents/chapter-3/Prompt_Pembuatan_Data_Sintetis.png}
  \caption[Contoh gambar]{Diagram Alur Urutan \textit{Prompt} Pembuatan Data Sintetis.}
  \label{Fig:prompt_pembuatan}
\end{figure}

\subsubsection{Pembuatan Data Sintetis}
Tujuan akhir dari \textit{prompt} gabungan ini 
adalah untuk memandu LLM secara sistematis dalam menghasilkan 
40 \textit{dataset} sintetis papan Kanban. Pembuatan data sintetis
tersebut akan diserahkan pada model \textbf{ChatGPT 5 Thinking} yang dapat
secara baik mengikuti instruksi dan menghasilkan file \texttt{.json}
di bagian luarannya. Berikut adalah contoh hasil luaran data sintetis
yang telah dibuat.

\lstset{
    language=Python,
    basicstyle={\fontsize{8}{10}\selectfont\ttfamily\bfseries},
    keywordstyle=\color{black}\bfseries,
    commentstyle=\color{black},
    stringstyle=\color{black},
    breaklines=true,
    showstringspaces=false
}
\begin{lstlisting}
[
  {
    "_id": {
      "$oid": "d17a5b11-e3d7-427f-9f67-8b1a10e43507"
    },
    "user_id": {
      "$oid": "0674118b-fd74-472d-8745-7a0d8b120a9c"
    },
    "name": "Papan OOP - MHS_MALAS_SK1",
    "lists": [
      {
        "id": "list1",
        "title": "Planning (To Do)",
        "cards": [
          {
            "id": "CS101-Konsep_Dasar_dan_Pilar_OOP-MHS_MALAS_SK1_1",
            "title": "Object Oriented Programming [CS101]",
            "sub_title": "Konsep Dasar dan Pilar OOP",
            "created_at": "2025-06-07T08:00:00+00:00",
            "description": "Memahami fondasi dari OOP dan 4 pilar utamanya.",
            "deleted": false,
            "archived": false,
            "checklists": [
              {
                "id": "92b43982-ed23-43f2-ad7e-6992212b1e16",
                "title": "Tugas untuk Konsep Dasar dan Pilar OOP",
                "items": [
                  {
                    "id": "c8751c49-555e-4990-8539-198b1775f87f",
                    "text": "Menjelaskan perbedaan paradigma pemrograman prosedural dan OOP.",
                    "completed": true
                  },
                  {
                    "id": "a7c54bf0-957a-49d6-a3d5-0c4de31b9441",
                    "text": "Mengidentifikasi serta mendeskripsikan 4 pilar OOP: Enkapsulasi, Abstraksi, Pewarisan, Polimorfisme.",
                    "completed": false
                  },
                  {
                    "id": "dcef3e0a-1ec3-4a5d-8d2e-9a8b9bc213f5",
                    "text": "Menerapkan konsep dasar objek dan kelas dalam program sederhana.",
                    "completed": false
                  }
                ]
              }
            ],
            "column_movement_times": {
              "list1": "2025-06-07T08:00:00+00:00"
            },
            "column_movements": []
          }
        ]
      },
      {
        "id": "list2",
        "title": "Monitoring (In Progress)",
        "cards": []
      },
      {
        "id": "list3",
        "title": "Controlling (Review)",
        "cards": []
      },
      {
        "id": "list4",
        "title": "Reflection (Done)",
        "cards": []
      }
    ]
  },
  ...
]
\end{lstlisting}

Untuk mempermudah pengambilan data umpan balik dari ahli manusia,
data sintetis tersebut akan dilakukan transformasi ke bentuk tabel
yang mudah dipahami dan secara bersamaan tetap memberikan gambaran
yang utuh mengenai struktur dari papan Kanban pembelajaran mahasiswa.
Transformasi ini dilakukan dengan menggunakan model LLM 
\textbf{ChatGPT 5 Thinking} dengan memerintahkan untuk mengkonversi
data berformat \texttt{JSON} untuk diubah menjadi format tabel yang
mudah dibaca oleh manusia. Setelah melakukan beberapa iterasi dalam
eksekusinya, didapatkan hasil akhir dengan contoh pada Tabel \ref{tab:data_sintetis_mahasiswa_tabel}.

\clearpage
\begin{landscape}
\scriptsize  % Add this line to reduce font size
\begin{table}[h]
\centering
\caption{Ringkasan Aktivitas Kanban per Mahasiswa}
\label{tab:data_sintetis_mahasiswa_tabel}
\begin{tabular}{|p{0.08\linewidth}|p{0.12\linewidth}|p{0.22\linewidth}|p{0.14\linewidth}|p{0.14\linewidth}|p{0.08\linewidth}|}
\hline
\textbf{MHS} & \textbf{Kolom} & \textbf{Sub-judul} & \textbf{Dibuat (UTC)} & \textbf{Checklist lengkap} & \textbf{Riwayat pergerakan} \\
\hline
MHS\_1 & Planning & Prinsip SOLID dan Adaptive Code & 2025-06-05 01:00 &
\ding{51}\;Menjelaskan dan memberi contoh penerapan dari masing-masing prinsip SOLID.
\ding{51}\;Melakukan refactoring kode agar sesuai dengan prinsip SOLID.
\ding{55}\;Menerapkan prinsip desain adaptif dalam proyek mini berbasis OOP. & - \\
\hline
MHS\_2 & Planning & Konsep Dasar dan Pilar OOP & 2025-06-07 08:00 &

\ding{51}\;Menjelaskan perbedaan paradigma pemrograman prosedural dan OOP.

\ding{55}\;Mengidentifikasi serta mendeskripsikan 4 pilar OOP: Enkapsulasi, Abstraksi, Pewarisan, Polimorfisme.

\ding{55}\;Menerapkan konsep dasar objek dan kelas dalam program sederhana. & - \\

\hline
\end{tabular}
\end{table}
\end{landscape}
\clearpage

Tabel tersebut sudah dihilangkan penamaan yang mengindikasikan perilaku 
mahasiswa seperti "MALAS" dan "RAJIN" karena akan digunakan untuk diberikan
pada ahli manusia untuk diberikan penilaian umpan balik secara netral.

\subsubsection{Pengambilan Data Umpan Balik Ahli Manusia}
Data sintetis papan pembelajaran Kanban tersebut setelah 
ditransformasi ke bentuk tabel, akan diberikan ke ahli manusia
untuk diberikan penilaian umpan balik nya. Umpan balik yang
diberikan memiliki tiga jenis, yaitu \textit{feedback}, 
\textit{motivation}, dan \textit{appreciation}. Untuk
mempermudah ahli dalam pengisian, data sintetis tersebut
akan ditampilkan menggunakan platform \textbf{Google Sheets}
dan menambahkan kolom \textit{feedback} \textit{motivation},
dan \textit{appreciation} sebagai tempat untuk memberikan umpan
balik nya. Kuesioner juga dibuatkan agar ahli manusia dapat
memberikan jawaban sesuai dengan instruksi yang diinginkan.
Kuesioner dibuat dengan menggunakan platform \textbf{Google Docs}
agar dapat dengan mudah dibagikan ke ahli manusia. Berikut adalah
contoh hasil umpan balik yang dihasilkan oleh ahli manusia yang
sudah dikonversi ke dalam data berformat \texttt{JSON}.

% \lstset{
%     language=Python,
%     basicstyle={\fontsize{10}{10}\selectfont\color{black}\ttfamily},
%     keywordstyle=\color{black}\bfseries,
%     commentstyle=\color{black},
%     stringstyle=\color{black},
%     breaklines=true,
%     showstringspaces=false,
    
%     inputencoding=utf8,     % <-- TAMBAHKAN INI
%     extendedchars=true      % <-- TAMBAHKAN INI (opsional, tapi disarankan)
% }

\begin{lstlisting}
"""
[
    {
    "feedback": "Rutin cek ulang sebelum berpindah topik. Fokus menuntaskan satu kartu sampai `Done` sebelum membuka yang lain. Tuliskan `next step` singkat di setiap perpindahan kolom. Rapikan urutan kerja agar tidak sering ganti konteks. Prioritaskan checklist inti terlebih dahulu.",
    "motivasi": "Konsistensi akan membuat progresmu terasa. Targetkan minimal satu item selesai per hari. Tingkatkan ritme sedikit demi sedikit. Kamu di jalur yang benar\u2014teruskan. Bangun momentum dengan kemenangan kecil.",
    "apresiasi": "Good job, tinggal dipertahankan. Upaya step-by-step menunjukkan komitmen. Kamu sudah bergerak ke arah yang tepat. Progres mulai terlihat dan patut diapresiasi. Terus jaga usaha ini agar stabil."
  },
  ...

]
"""
\end{lstlisting}

% \begin{enumerate}
% 	\item Perancangan \textit{prompt}
% 	\begin{itemize}
% 		\item Menentukan tujuan setiap \textit{prompt} (mis. menghasilkan tugas, status, komentar reflektif).
% 		\item Menyusun format luaran yang konsisten (mis. JSON dengan field: id, task, status, timestamp, comment).
% 		\item Menyertakan instruksi konteks agar output sesuai skenario SRL berbasis Kanban.
% 	\end{itemize}

% 	\item Generasi data sintetis
% 	\begin{itemize}
% 		\item Menggunakan ChatGPT untuk membuat contoh papan Kanban berbasis skenario mahasiswa.
% 		\item Membuat variasi kondisi (mis. beban tugas, tingkat prokrastinasi, interaksi tim).
% 		\item Menyimpan luaran dalam berkas \texttt{.json} terstruktur.
% 	\end{itemize}

% 	\item Pengumpulan \textit{ground truth} ahli
% 	\begin{itemize}
% 		\item Merekrut ahli pembelajaran untuk menghasilkan umpan balik pedagogis pada contoh data.
% 		\item Menstandarisasi format umpan balik (kategori: feedback, motivation, appreciation).
% 		\item Menyimpan anotasi ahli sebagai pasangan input–ground truth untuk evaluasi.
% 	\end{itemize}

% 	\item Validasi dan pembersihan data
% 	\begin{itemize}
% 		\item Memeriksa konsistensi format dan menghapus entri duplikat atau tidak relevan.
% 		\item Melakukan pemeriksaan kualitas anotasi oleh reviewer kedua jika diperlukan.
% 	\end{itemize}

% 	\item Pengelolaan versi dan penyimpanan
% 	\begin{itemize}
% 		\item Menyimpan versi dataset dengan penomoran dan catatan perubahan.
% 		\item Menyimpan salinan aman (lokal dan/atau cloud) serta mencatat metadata (tanggal, pembuat).
% 	\end{itemize}

% 	\item Pertimbangan etis dan privasi
% 	\begin{itemize}
% 		\item Membuat data anonim dan memastikan tidak ada informasi identitas pribadi.
% 		\item Mendokumentasikan izin penggunaan data jika diperlukan.
% 	\end{itemize}
% \end{enumerate}

\subsection{Pemilihan LLMs}
% Pemilihan LLMs yang akan digunakan dalam penelitian pemberian
% ini bukan menjadi
% fokus utama.

Penelitian ini menggunakan dua model dari keluarga Llama yang 
tersedia melalui Groq API yaitu \textbf{Llama 3.1 8B Instant} dan 
Llama 3.3 70B Versatile untuk menghasilkan feedback pedagogis 
yang mendukung SRL mahasiswa. 
Pemilihan LLMs yang akan digunakan dalam penelitian
ini bukan menjadi fokus utama.
Pemilihan kedua model ini didasarkan pada beberapa pertimbangan.

% Kedua model tersebut diketahui bisa digunakan untuk ....

Keterbatasan sumber daya finansial dan komputasi yang 
dimiliki peneliti. Justifikasi berbasis 
\textit{resource constraints} merupakan pendekatan yang valid 
dalam penelitian akademis pendidikan. Groq API menyediakan 
kedua model dengan biaya yang kompetitif 
($0.05-0.08 per juta token untuk 8B dan $0.59-0.79 per 
juta token untuk 70B), memungkinkan eksperimen dengan volume 
yang memadai untuk menghasilkan \textit{sufficient} pedagogical 
variasi dan \textit{robust} statistical analysis. 
Lebih lanjut, Steinert et al. \cite{steinert2024harnessing} menekankan pentingnya 
platform sumber terbuka agar teknologi LLM terjangkau dan 
inklusif. Hal ini mendukung alasan pemilihan model Llama 
(yang bersifat open-source) jika dibanding model proprietary 
berbiaya tinggi. Efisiensi biaya ini penting karena 
memungkinkan penelitian empiris dalam 
praktik pendidikan yang sering menghadapi \textit{constraint} 
sumber daya. \cite{groq_pricing}


% intinya belum ada yang neliti besar vs kecil

Didapatkan bahwa belum ada studi komparatif yang
membandingkan model LLM berukuran kecil (8B)
dengan model berukuran besar (70B) dalam konteks
pemberian \textit{feedback} pedagogis yang mendukung
\textit{Self-Regulated Learning} (SRL) mahasiswa.
% Pertimbangan ilmiah yang kuat untuk mengeksplorasi
Eksplorasi perbedaan ukuran model adalah penting
dalam konteks ini karena
ukuran model LLM akan mempengaruhi sumber daya komputasi, biaya operasional,
dan kemampuan adaptasi \textit{feedback} yang dihasilkan.
Penulis ingin menyelidiki bagaimana perbedaan kapasitas
model ini mempengaruhi kualitas \textit{feedback} yang dihasilkan
dalam mendukung SRL mahasiswa. Hasil penelitian ini dapat menjawab
research question mengenai pengaruh ukuran model terhadap
kualitas \textit{feedback} pedagogis yang dihasilkan.

% Ketiga, kedua model telah terbukti dalam penelitian sebelumnya 
% mampu menghasilkan pedagogically-sound \textit{feedback} 
% yang \textit{aligned} dengan prinsip-prinsip pembelajaran. 
% Studi terbaru (2025) menunjukkan bahwa Llama 3.1 secara 
% spesifik dapat mengekstrak indikator \textit{feedback} 
% dengan \textit{alignment} yang signifikan terhadap 
% \textit{teacher ratings} dalam \textit{learning assessment}. 
% Penelitian lebih luas menunjukkan bahwa LLM termasuk model 
% Llama dapat diintegrasikan dalam sistem pedagogical yang 
% menyediakan \textit{adaptive scaffolding} berbasis teori 
% kognitif sosial dan \textit{framework} SRL Zimmerman. Dengan 
% demikian, kedua model dipilih karena \textit{track record} 
% mereka dalam menghasilkan \textit{feedback} yang mendukung 
% pembelajaran siswa, khususnya dalam konteks \textit{formative 
% assessment} dan \textit{self-regulated learning} support.

% Keempat, kedua model berasal dari \textit{family} yang sama 
% (Llama 3.x) dengan arsitektur serupa dan \textit{context window} 
% identik (128K tokens), memastikan bahwa ukuran model adalah 
% variabel utama yang diamati, bukan perbedaan arsitektural 
% fundamental. Kontrol metodologis ini penting untuk isolasi 
% variabel, memungkinkan peneliti untuk mengatribusikan perbedaan 
% dalam kualitas \textit{feedback} semata-mata pada perbedaan 
% kapasitas model dan bukan pada faktor-faktor teknis lainnya.

% Kelima, berbagai teknik \textit{prompt engineering} yang 
% akan diterapkan dalam penelitian ini dirancang untuk 
% mengeksplorasi bagaimana instruksi yang berbeda menghasilkan 
% tingkat adaptabilitas \textit{feedback} yang berbeda. 
% Penelitian menunjukkan bahwa kualitas \textit{metacognitive 
% scaffolding} dan \textit{personalized feedback} dipengaruhi 
% tidak hanya oleh ukuran model tetapi juga oleh 
% \textit{design of prompting instructions}. Dengan 
% membandingkan \textit{prompt engineering} yang berbeda 
% \textit{across model sizes}, penelitian ini dapat 
% memberikan \textit{insights} tentang strategi optimal 
% untuk menghasilkan \textit{feedback} yang lebih efektif 
% dalam mendukung SRL pada model dengan ukuran berbeda.

% Dengan demikian, pilihan kedua model ini tidak hanya 
% \textit{feasible} secara praktis dan finansial, tetapi 
% juga memenuhi persyaratan ilmiah untuk studi komparatif 
% yang valid mengenai pengaruh ukuran model dan teknik 
% \textit{prompt engineering} terhadap kualitas 
% \textit{feedback} pedagogis yang mendukung 
% \textit{Self-Regulated Learning} mahasiswa. 
% Penelitian ini berkontribusi pada pemahaman tentang 
% bagaimana LLM dapat dioptimalkan untuk memberikan 
% \textit{adaptive scaffolding} dan \textit{metacognitive 
% support} yang efektif dalam konteks pendidikan.



% Pemilihan LLMs yang akan digunakan dalam penelitian pemberian
% ini bukan menjadi
% fokus utama, namun penulis memilih model LLMs yang memiliki kemampuan
% untuk mengikuti instruksi \textit{prompt} dengan baik serta dapat
% m

% Penelitian ini bertujuan untuk mengevaluasi dampak dari berbagai
% strategi rekayasa \textit{prompt} terhadap kualitas penilaian mahasiswa
% yang diotomatisasi, dan bagaimana dampak ini dimoderasi oleh ukuran
% \textit{Large Language Model} (LLM). Untuk menyelidiki pertanyaan
% ini, dua model dari seri Llama Meta yang di-\textit{hosting} di
% \textit{Groq API} dipilih untuk analisis komparatif: Llama 3.1 8B Instant
% dan Llama 3.3 70B Versatile[1].
% % [1] https://console.groq.com/docs/models 

% Pemilihan ini bersifat metodologis dan disengaja, 
% didasarkan pada tiga pilar justifikasi.

% Pertama, kedua model ini mewakili titik-titik yang berbeda secara 
% signifikan pada spektrum biaya-kinerja. Model 8B menawarkan efisiensi 
% biaya yang tinggi (\$0.05/juta token input) dan kecepatan inferensi 
% yang superior ($\sim 560-840$ TPS), mewakili skenario yang terbatas 
% sumber daya[2]. Sebaliknya, model 70B mewakili kinerja canggih (SOTA) dengan 
% biaya operasional yang $\sim 10-12$ kali lebih tinggi 
% (\$0.59/juta token input) dan kecepatan yang lebih lambat 
% ($\sim 280-394$ TPS)[2]. Pilihan ini memungkinkan dilakukannya 
% analisis biaya-manfaat yang ketat, sebuah pertimbangan strategis 
% utama dalam penerapan AI praktis[3].
% % [2] https://console.groq.com/docs/models
% % [3] https://groq.com/pricing

% Kedua, dan yang paling penting secara teoretis, pemilihan 
% model 8B (kecil) \textit{versus} 70B (besar) menyediakan kerangka 
% kerja yang ideal untuk menguji hipotesis inti penelitian ini. 
% Literatur telah menetapkan bahwa model yang lebih kecil secara 
% signifikan lebih sensitif terhadap variasi dalam format dan kualitas 
% \textit{prompt}, sedangkan model yang lebih besar cenderung lebih kuat 
% (\textit{robust}) dan konsisten terhadap variasi tersebut. Secara paralel, 
% penelitian menunjukkan bahwa rekayasa \textit{prompt} yang cermat dapat 
% secara signifikan meningkatkan kinerja model yang lebih kecil hingga 
% mendekati tingkat model yang lebih besar. Oleh karena itu, penelitian 
% ini secara eksplisit akan menguji efek interaksi ini: hipotesisnya 
% adalah bahwa dampak dari intervensi rekayasa \textit{prompt} akan jauh 
% lebih besar pada Llama 3.1 8B daripada pada Llama 3.3 70B yang sudah kuat.

% Ketiga, pilihan model dan domain tugas ini didukung kuat oleh preseden 
% akademis. Penelitian terbaru (2024-2025) telah menggunakan model 
% Llama 3 (termasuk varian 70B) secara ekstensif untuk tugas-tugas 
% penilaian pendidikan, seperti penilaian pekerjaan rumah dan esai. 
% Studi-studi ini mengonfirmasi kelayakan penggunaan Llama 70B untuk 
% tugas-tugas penalaran dan menyoroti peran penting dari '\textit{prompt} 
% yang dirancang dengan baik' untuk mencapai akurasi yang sebanding 
% dengan manusia. Penelitian ini berupaya untuk memperluas 
% temuan-temuan ini dengan secara langsung membandingkan efektivitas 
% \textit{prompt} antara model SOTA (70B) dan model yang efisien 
% secara komputasi (8B).


\subsection{Metode Peningkatan Hasil Umpan Balik}
Tahap ini menjelaskan metode-metode yang digunakan untuk
meningkatkan hasil umpan balik pedagogis yang dihasilkan oleh
LLMs. Metode-metode ini bertujuan untuk mengoptimalkan
kualitas umpan balik yang diberikan kepada siswa.
Peningkatan mutu umpan balik pedagogis dilakukan
melalui tiga pendekatan utama, yaitu rekayasa konteks
(\textit{context engineering}), integrasi metode \textit{learning analytics},
dan pembuatan variasi \textit{prompt}.

\subsubsection{Rekayasa Konteks (\textit{Context Engineering})}
Pembuatan \textit{prompt} yang kaya akan konteks sangat dilakukan
dengan mengikuti beberapa tujuan pembuatan. Agar penelitian ini 
dapat berjalan dengan baik, dibuat beberapa tujuan pembuatan 
\textit{prompt} yaitu dapat memberikan umpan balik yang jelas, 
terstruktur, dan bersifat adaptif secara pedagogis, mampu 
memberikan dorongan untuk meningkatkan efikasi diri (motivasi), 
serta memandu penetapan tujuan (bimbingan) yang selaras dengan 
tahapan belajar individual mahasiswa.

Beberapa strategi rekayasa konteks yang diterapkan meliputi:
\begin{itemize}
  	\item Penjelasan peran model sebagai \textit{pedagogical feedback
  	generator} yang mendukung SRL mahasiswa. 
  	\item Penyertaan data pembelajaran Kanban mahasiswa untuk menyesuaikan
  	umpan balik berdasarkan kemajuan sebelumnya.
  	\item Mengikuti ketentuan \textit{theory-driven prompt manual} Subbab \ref{subsection:tinjauan_perbandingan_metode} 
	untuk mengoptimalkan \textit{prompt}.
	\item Mengikuti prinsip-prinsip yang diuraikan dalam literatur \cite{Shute2008FocusFormativeFeedback, HattieTimperley2007PowerOfFeedback,
	DeciRyan1985IntrinsicMotivation, Fredrickson2001BroadenBuild} agar dihasilkan
	umpan balik feedback, motivation, dan appreciation yang sesuai.
	\item Format luaran yang terstruktur dalam format \texttt{JSON} untuk
	memudahkan dilakukan evaluasi.
	\item \textit{Prompt} tidak boleh melebihi batas konteks (\textit{context length}) model LLM yang digunakan.
\end{itemize}
% ...existing code...
Struktur prompt yang diusulkan terdiri dari beberapa komponen utama 
yang dirancang untuk memastikan keluaran LLM bersifat pedagogis dan mudah 
dievaluasi. Komponen tersebut harus memberikan informasi mengenai. 
Pembagian komponen ini dimaksudkan agar prompt menjadi 
sistematis sehingga model dapat menghasilkan respons yang relevan, 
konsisten, dan siap dianalisis.
% ...existing code...
\begin{enumerate}
  	\item \textbf{Definisi Peran dan Persona Inti}: instruksi dasar 
	yang mengatur identitas, nada, dan batasan utama dari agen penilai.
  	\item \textbf{Logika Internal dan Rubrik Penilaian}: Rubrik penilaian  internal 
	yang digunakan agen untuk menganalisis data mentah.
  	\item \textbf{Informasi Kontekstual}: Pengetahuan domain yang spesifik untuk 
	tugas pembelajaran
  	\item \textbf{\textit{Prompt} Penilaian}: Instruksi eksekusi 
	utama yang menggabungkan semua elemen untuk menghasilkan 
	respons dengan variasi instruksi tertentu.
  	\item \textbf{Format Input dan Output}: Templat yang digunakan untuk 
	memasukkan data dan mendefinisikan format keluaran akhir.
\end{enumerate}

Untuk lebih detailnya, pembagian struktur \textit{Prompt} tersebut 
dibahas lebih rinci seperti berikut ini.
\subsubsection*{Definisi Peran dan Persona Inti}
% tolong buatkan setidaknya 5 kalimat yang menjelaskan bagian ini.
Pembuatan \textit{prompt} menggunakan jenis \textit{Role-play}
untuk mendefinisikan peran dan persona inti dari \textit{Feedback Agent}.
Bagian ini menetapkan identitas dan batasan utama dari \textit{Feedback Agent}.
Instruksi ini mengarahkan model untuk berperan sebagai Agen Pedagogis
yang bertugas membantu dan membimbing mahasiswa dalam proses belajar
khususnya pada mata kuliah Teknologi Informasi (IT).
Instruksi ini menekankan pentingnya penggunaan nada bicara yang
ramah, personal, dan mudah didekati, serta konsisten menggunakan sapaan 'Anda'.
Selain itu, instruksi ini juga menetapkan batasan-batasan penting
agar \textit{Feedback Agent} tidak menyimpang dari topik pendidikan atau
tujuan pembelajaran mahasiswa. Tabel \ref{tab:prompt_definisi} menunjukkan
contoh potongan \textit{prompt} yang digunakan untuk mendefinisikan
peran dan batasan \textit{Feedback Agent}.

\begin{table}[H] % Anda bisa ganti [H] dengan [htbp] jika perlu
  \centering
  
  % Caption dan Label sekarang milik "table", bukan "lstlisting"
  \caption{\textit{Prompt} Definisi Peran dan Batasan Agen Penilai}
  \label{tab:prompt_definisi} % Saya ubah labelnya menjadi "tab:"
  
  % Gunakan minipage untuk "membungkus" listing dengan aman
  \begin{minipage}{\textwidth}
    
    % Mulai listing Anda (tanpa caption/label di sini)
    \begin{lstlisting}
		INITIAL_PROMPT = (

			"Anda adalah Agen Pedagogis yang bertugas untuk membantu dan membimbing mahasiswa dalam proses belajar kuliah IT mereka. "

			"Tugas Anda adalah memantau, menyesuaikan strategi jika diperlukan, dan memastikan bahwa mahasiswa tetap termotivasi untuk menyelesaikan pembelajaran mereka. "

			"Gunakan nada bicara yang ramah, personal, mudah didekati, dan konsisten menggunakan sapaan 'Anda'. "

			"Anda harus menanyakan tentang kemajuan belajar mereka untuk memastikan keterlibatan. "

			"Selalu rujuk data papan (nama kartu, isi checklist, perpindahan kolom/waktu) saat memberi saran. "

			"Jangan menyimpang dari topik pendidikan atau tujuan pembelajaran mahasiswa. "

			"Jangan membocorkan jawaban/solusi akhir; gunakan pertanyaan pemandu dan hint bertahap (scaffolding). "

			"Jangan menampilkan label kategori internal kepada mahasiswa."

		)
	\end{lstlisting}
%--- Selesai listing ---

  \end{minipage} 
  %--- Selesai minipage ---

\end{table}

% ki


\subsubsection*{Logika Internal dan Rubrik Penilaian}
% tolong buatkan setidaknya 5 kalimat yang menjelaskan bagian ini.
Bagian ini menyajikan logika internal dan rubrik penilaian
yang digunakan oleh \textit{Feedback Agent} untuk menganalisis data
pembelajaran mahasiswa. Rubrik ini mendefinisikan kriteria
yang digunakan untuk mengkategorikan perilaku belajar mahasiswa
berdasarkan aktivitas mereka pada papan Kanban. Rubrik ini mencakup lima kategori
perilaku belajar, mulai dari "Malas" hingga "Sangat Rajin",
dengan kriteria spesifik yang mengacu pada jumlah kartu,
jumlah item checklist yang diselesaikan, dan frekuensi perpindahan
kolom dalam jangka waktu tertentu. Tabel \ref{tab:prompt_kriteria} menunjukkan
contoh potongan \textit{prompt} yang digunakan untuk menyajikan
logika internal dan rubrik penilaian \textit{Feedback Agent}.


\begin{table}[H] % Anda bisa ganti [H] dengan [htbp] jika perlu
  \centering
  
  % Caption dan Label sekarang milik "table", bukan "lstlisting"
  \caption{\textit{Prompt} Kriteria Mahasiswa}
  \label{tab:prompt_kriteria} % Saya ubah labelnya menjadi "tab:"
  % Gunakan minipage untuk "membungkus" listing dengan aman
  \begin{minipage}{\textwidth}
    
    % Mulai listing Anda (tanpa caption/label di sini)
    \begin{lstlisting}
		KRITERIA_MAHASISWA= (

			"Catatan: Kategori ini HANYA untuk penalaran internal model, JANGAN disebutkan ke mahasiswa.\n\n"

			"Kategori: Malas\n"

			"- Papan JSON 1-2 kartu; tiap kartu 3 checklist; total 0-2 item dicentang/kartu; tidak ada perpindahan $\\leq$2 jam terakhir.\n\n"

			"Kategori: Sedikit Malas\n"

			"- Papan JSON 3-4 kartu; tiap kartu 3 checklist; total 0-2 item dicentang/kartu; tidak ada perpindahan $\\leq$2 jam terakhir.\n\n"

			"Kategori: Cukup\n"

			"- Papan JSON 4-5 kartu; tiap kartu 3 checklist; total 1-2 item dicentang/kartu; ada \\geq 3 perpindahan dalam \\leq 2 jam terakhir.\n\n"

			"Kategori: Rajin\n"

			"- Tepat 5 kartu; tiap kartu 3 checklist; total 2-3 dicentang/kartu; hanya 1 kartu di Planning, sisanya di kolom lain.\n\n"

			"Kategori: Sangat Rajin\n"

			"- Tepat 5 kartu; tiap kartu 3 checklist dan semua dicentang; semua 5 kartu di 'Reflection (Done)'."

		)
	\end{lstlisting}
%--- Selesai listing ---

  \end{minipage} 
  %--- Selesai minipage ---

\end{table}

Selain itu, bagian ini juga menyajikan logika internal
yang menghubungkan kolom pada papan Kanban dengan fase
pembelajaran mahasiswa. Setiap kolom pada papan Kanban
dikaitkan dengan fase pembelajaran tertentu, yaitu
Perencanaan, Pemantauan, Pengendalian, dan Refleksi. 
Tabel \ref{tab:prompt_fase} menunjukkan
contoh potongan \textit{prompt} yang digunakan untuk menyajikan
logika internal penghubung kolom dengan fase pembelajaran.


\begin{table}[H] % Anda bisa ganti [H] dengan [htbp] jika perlu
  \centering
  
  % Caption dan Label sekarang milik "table", bukan "lstlisting"
  \caption{\textit{Prompt} Fase Pembelajaran Mahasiswa}
  \label{tab:prompt_fase} % Saya ubah labelnya menjadi "tab:"
  % Gunakan minipage untuk "membungkus" listing dengan aman
  \begin{minipage}{\textwidth}
    
    % Mulai listing Anda (tanpa caption/label di sini)
    \begin{lstlisting}
		FASE_SRL_KOLOM= (

			"Hubungan kolom dengan fase pembelajaran:\n"

			"* List1: 'Planning (To Do)' - fase perencanaan.\n"

			"* List2: 'Monitoring (In Progress)' - fase pemantauan progres.\n"

			"* List3: 'Controlling (Review)' - fase kontrol & penyesuaian tugas.\n"

			"* List4: 'Reflection (Done)' - fase refleksi & penandaan selesai."

		)
	\end{lstlisting}
%--- Selesai listing ---

  \end{minipage} 
  %--- Selesai minipage ---

\end{table}

\subsubsection*{Informasi Kontekstual}
% tolong buatkan setidaknya 5 kalimat yang menjelaskan bagian ini.
Bagian ini menyajikan informasi kontekstual yang relevan
untuk mendukung proses penilaian oleh \textit{Feedback Agent}.
Informasi ini mencakup daftar materi pembelajaran
yang dapat diambil oleh mahasiswa, termasuk mata kuliah
dan sub-topik yang tersedia. Daftar materi ini
memberikan konteks tambahan bagi \textit{Feedback Agent}
untuk memahami ruang lingkup pembelajaran mahasiswa.
Tabel \ref{tab:prompt_listmateri} menunjukkan
contoh potongan \textit{prompt} yang digunakan untuk menyajikan
daftar materi pembelajaran mahasiswa.

\begin{table}[H] % Anda bisa ganti [H] dengan [htbp] jika perlu
  \centering
  
  % Caption dan Label sekarang milik "table", bukan "lstlisting"
  \caption{\textit{Prompt} Daftar Materi Mahasiswa}
  \label{tab:prompt_listmateri} % Saya ubah labelnya menjadi "tab:"
  % Gunakan minipage untuk "membungkus" listing dengan aman
  \begin{minipage}{\textwidth}
    
    % Mulai listing Anda (tanpa caption/label di sini)
    \begin{lstlisting}
		LIST_MATERI_MAHASISWA = """

			List Materi yang dapat diambil oleh mahasiswa:

			a. Mata Kuliah: Object Oriented Programming [CS101]

			b. Sub Topik ada 5 (boleh ambil kurang tp maksimal 5):

			- Konsep Dasar dan Pilar OOP

			- Perancangan Kelas dan Interaksi Objek

			- Enkapsulasi, Inheritance, dan Polimorfisme

			- Prinsip SOLID dan Adaptive Code

			- Kolaborasi Objek dan Pengorganisasian Kode



		"""
	\end{lstlisting}
%--- Selesai listing ---

  \end{minipage} 
  %--- Selesai minipage ---

\end{table}

List materi ini diharapkan dapat membantu \textit{Feedback Agent}
dalam memberikan umpan balik yang lebih relevan dan kontekstual
terhadap aktivitas pembelajaran mahasiswa. Namun, bagian ini juga
menambahkan \textit{context length} yang lebih besar untuk
memungkinkan pemrosesan informasi yang lebih kompleks. Hal ini
dapat mempengaruhi performa model LLM, terutama untuk model
dengan kapasitas konteks yang lebih kecil. Oleh karena itu,
bagian ini menjadi variasi dalam rekayasa konteks yang
akan dieksplorasi dalam penelitian ini.

\subsubsection*{\textit{Prompt} Penilaian}
% tolong buatkan setidaknya 5 kalimat yang menjelaskan bagian ini.
Bagian ini menyajikan \textit{prompt} penilaian utama
yang menggabungkan semua komponen sebelumnya untuk
menghasilkan respons umpan balik pedagogis. \textit{Prompt} ini memberikan instruksi
terperinci kepada \textit{Feedback Agent} tentang bagaimana
menggunakan data pembelajaran mahasiswa, rubrik penilaian, dan informasi kontekstual
untuk menghasilkan umpan balik yang relevan dan konstruktif. \textit{Prompt} ini menekankan pentingnya
penggunaan bahasa yang jelas, lugas, dan mudah dipahami, serta
menyediakan pedoman spesifik untuk format keluaran
yang diharapkan. Bagian ini juga menegaskan kembali
pentingnya memberikan umpan balik yang bersifat
analitis, berbasis bukti, dan konstruktif, serta
menyertakan elemen motivasi yang sesuai dengan
prinsip Self-Determination Theory.
Bagian ini akan memiliki variasi dalam rekayasa konteks
dengan menyertakan atau menghilangkan informasi
kontekstual untuk mengevaluasi dampaknya terhadap
kualitas umpan balik yang dihasilkan.
Tabel \ref{tab:prompt_penilaian} menunjukkan
contoh potongan \textit{prompt} yang digunakan untuk menyajikan
\textit{prompt} penilaian \textit{Feedback Agent}.

\begin{table}[H] % Anda bisa ganti [H] dengan [htbp] jika perlu
  \centering
  
  % Caption dan Label sekarang milik "table", bukan "lstlisting"
  \caption{\textit{Prompt} Penilaian Umpan Balik Pedagogis}
  \label{tab:prompt_penilaian} % Saya ubah labelnya menjadi "tab:"
  % Gunakan minipage untuk "membungkus" listing dengan aman
  \begin{minipage}{\textwidth}
    
    % Mulai listing Anda (tanpa caption/label di sini)
    \begin{lstlisting}
		ASSESSMENT_PROMPT= (

			"Anda adalah seorang dosen yang mengevaluasi mahasiswa dalam proses pembelajaran. "

			"Tugas Anda: beri komentar objektif dan konstruktif berdasarkan data.\n\n"

			"Fase-kolom:\n{FASE_SRL_KOLOM}\n\n"

			"Informasi kategori internal (untuk penalaran saja, jangan disebutkan ke mahasiswa):\n{KRITERIA_MAHASISWA}\n\n"

			"Kriteria Output (WAJIB):\n"

			"- Pastikan semua respons menggunakan bahasa yang jelas, lugas, dan mudah dipahami.\n"

			"- Berikan 'feedback' yang berisi analisis singkat, bukti dari data, dan langkah berikutnya yang konstruktif.\n"

			"- Berikan 'motivasi' yang singkat, empatik, dan berdasarkan prinsip Self-Determination Theory:\n"

			" - Dukung Otonomi: Tawarkan pilihan atau ajukan pertanyaan reflektif, jangan hanya memerintah. Contoh: 'Ada dua kartu prioritas, A atau B. Menurut Anda, mana yang ingin diselesaikan dulu?'\n"

			" - Tingkatkan Kompetensi: Kaitkan progres dengan kemampuan spesifik yang ditunjukkan. Contoh: 'Cara Anda memecah tugas di kartu X menunjukkan kemampuan perencanaan yang baik.'\n"

			" - Jaga Keterhubungan: Tunjukkan pemahaman jika ada tanda-tanda kesulitan dari data.\n"

			"- Jangan berikan solusi/jawaban langsung.\n"

			"- Keluarkan HANYA JSON dengan kunci 'feedback','motivasi','apresiasi', tanpa teks lain, dalam Bahasa Indonesia baku ('Anda')."

		)
	\end{lstlisting}
%--- Selesai listing ---

  \end{minipage} 
  %--- Selesai minipage ---

\end{table}

\subsubsection*{Format Input dan Output}
% tolong buatkan setidaknya 5 kalimat yang menjelaskan bagian ini.
Bagian ini menyajikan format input dan output
yang digunakan dalam \textit{prompt} penilaian. Format input
menyediakan struktur yang jelas untuk memasukkan data
pembelajaran mahasiswa, termasuk indikator ringkas,
analitik mahasiswa dalam format JSON, dan data papan
Kanban dalam format JSON mentah. Format output
menyediakan struktur yang jelas untuk menyajikan umpan
balik yang dihasilkan oleh sistem.
Bagian ini menegaskan pentingnya konsistensi
dalam format input dan output untuk memastikan
integritas data dan kemudahan evaluasi.
Terdapat variasi dalam rekayasa konteks
dengan menyesuaikan format input untuk
mengakomodasi data tambahan atau pengurangan.
Tabel \ref{tab:prompt_format_input_output}
menunjukkan contoh format input dan output yang digunakan
dalam \textit{prompt} penilaian.

\begin{table}[H] % Anda bisa ganti [H] dengan [htbp] jika perlu
  \centering
  
  % Caption dan Label sekarang milik "table", bukan "lstlisting"
  \caption{\textit{Prompt} Format Input dan Output}
  \label{tab:prompt_format_input_output} % Saya ubah labelnya menjadi "tab:"
  % Gunakan minipage untuk "membungkus" listing dengan aman
  \begin{minipage}{\textwidth}
    
    % Mulai listing Anda (tanpa caption/label di sini)
    \begin{lstlisting}
		FORMAT_RESPONSE = (

			"{\n"

			" \"feedback\": \"...\",\n"

			" \"motivasi\": \"...\",\n"

			" \"apresiasi\": \"...\"\n"

			"}\n"

		)

		USER_PROMPT_LA_DATA = (

			"Gunakan indikator ringkas berikut sebagai dasar alasan (jangan salin mentah-mentah). "

			"Tetap keluarkan hasil dalam Bahasa Indonesia berupa JSON sesuai format.\n\n"

			"Indikator ringkas mahasiswa (LA): {INDIKATOR_RINGKAS}\n"

			"Analitik mahasiswa (JSON):\n{ANALISIS_JSON}\n\n"

			"Data papan mahasiswa (JSON mentah):\n{DATA}\n\n"

			"Format keluaran (persis):\n"

			"{FORMAT_RESPONSE}"

		)
	\end{lstlisting}
%--- Selesai listing ---

  \end{minipage} 
  %--- Selesai minipage ---

\end{table}


\subsubsection{Integrasi Metode \textit{Learning Analytics}}
Integrasi metode \textit{learning analytics} memungkinkan pengumpulan dan analisis data tentang interaksi siswa dengan materi pembelajaran. Dengan memanfaatkan data ini, umpan balik dapat disesuaikan secara dinamis untuk memenuhi kebutuhan individu siswa, meningkatkan relevansi dan dampaknya.



\subsubsection{Pembuatan Variasi \textit{Prompt}}
Pembuatan variasi \textit{prompt} melibatkan pengembangan berbagai versi \textit{prompt} yang digunakan untuk menghasilkan umpan balik. Dengan menciptakan variasi ini, diharapkan dapat meningkatkan keberagaman dan kedalaman umpan balik yang diberikan kepada siswa, serta mendorong eksplorasi yang lebih luas terhadap materi pembelajaran.

% \section{Etika, Masalah, dan Keterbatasan Penelitian (Opsional))}

% Bagian ini membahas pertimbangan etis penelitian dan [potensi] masalah serta
% keterbatasannya. Jika menyangkut penelitian dengan makhluk hidup, maka dibutuhkan adanya \textit{ethical clearance}, di bagian ini hal itu akan dibahas. Demikian juga tentang keterbatasan ataupun masalah yang akan timbul.
