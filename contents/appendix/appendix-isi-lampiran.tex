\chapter*{LAMPIRAN}
% \addcontentsline{toc}{chapter}{LAMPIRAN}

% 1. Tulis Judul Section di sini (bukan di dalam command pdf)
% \section{Kuesioner Evaluasi Efektivitas Chatbot dalam Memberikan Feedback}
% % Jika ingin masuk daftar isi, uncomment baris bawah:
% % \addcontentsline{toc}{section}{Kuesioner Evaluasi...}

% % 2. Masukkan HALAMAN 1 menggunakan \includegraphics
% % Gunakan 'page=1' untuk mengambil halaman pertama saja.
% % Kita perlu mengatur 'width' atau 'scale' agar muat di sisa ruang halaman (karena sudah terpakai judul).
% \begin{center}
%     \includegraphics[page=1, width=1.5\textwidth]{contents/appendix/Kuesioner Evaluasi Efektivitas Chatbot sebagai Asisten Pembelajaran dalam Memberikan Feedback.pdf}
% \end{center}

% % 3. Masukkan HALAMAN 2 Seterusnya menggunakan \includepdf biasa
% \includepdf[pages=2-, pagecommand={}]{contents/appendix/Kuesioner Evaluasi Efektivitas Chatbot sebagai Asisten Pembelajaran dalam Memberikan Feedback.pdf}

% \chapter*{LAMPIRAN}
%\addcontentsline{toc}{chapter}{LAMPIRAN}	
% saya ingin melampirkan file pdf hasil survei kuesioner
% \includepdf[
%     pages=1,
%     pagecommand=\section{Kuesioner Evaluasi Efektivitas Chatbot dalam Memberikan Feedback}
% ]{contents/appendix/Kuesioner Evaluasi Efektivitas Chatbot sebagai Asisten Pembelajaran dalam Memberikan Feedback.pdf}

\section{Kuesioner Evaluasi Efektivitas Chatbot sebagai Asisten Pembelajaran dalam Memberikan Feedback}
\includepdf[pages=-, pagecommand={}]{contents/appendix/Kuesioner Evaluasi Efektivitas Chatbot sebagai Asisten Pembelajaran dalam Memberikan Feedback.pdf}

% \section{Kuesioner Evaluasi Efektivitas Chatbot sebagai Asisten Pembelajaran dalam Memberikan Feedback}
% \includepdf[pages=-]{contents/appendix/Kuesioner Evaluasi Efektivitas Chatbot sebagai Asisten Pembelajaran dalam Memberikan Feedback.pdf}

\section{Wawancara Penilaian Ahli Manusia terhadap Umpan Balik yang Dihasilkan oleh LLM}
\includepdf[pages=-, pagecommand={}]{contents/appendix/Recording Wawancara.pdf}

\section{Data Respons \textit{Feedback}, \textit{Motivation}, dan \textit{Appreciation} dari Ahli Manusia}
\includepdf[pages=-, pagecommand={}]{contents/appendix/Data Respon Ahli.pdf}
% \section{Isi Lampiran}

% Lampiran bersifat opsional bergantung hasil kesepakatan dengan pembimbing 
% dapat berupa:

% \begin{enumerate}
% \item Bukti pelaksanaan Kuesioner seperti pertanyaan kuesioner, resume jawaban 
% responden, dan dokumentasi kuesioner. 
% \item Spesifikasi Aplikasi atau Sistem yang dikembangkan meliputi spesifikasi 
% teknis aplikasi, tautan unduh aplikasi, manual penggunaan aplikasi, hingga 
% screenshot aplikasi. 
% \item Cuplikan kode yang sekiranya penting dan ditambahkan. 
% \item Tabel yang terlalu panjang yang masih diperlukan tetapi tidak 
% memungkinkan untuk ditayangkan di bagian utama skripsi.
% \item Gambar-gambar pendukung yang tidak terlalu penting untuk ditampilkan di 
% bagian utama. Akan tetapi, mendukung argumentasi/pengamatan/analisis.
% \item Penurunan rumus-rumus atau pembuktian suatu teorema yang terlalu 
% panjang dan terlalu teknis sehingga Anda berasumsi bahwa pembaca biasa 
% tidak akan menelaah lebih lanjut. Hal ini digunakan untuk memberikan 
% kesempatan bagi pembaca tingkat lanjut untuk melihat proses penurunan 
% rumus-rumus ini.
% \end{enumerate}