\chapter{Hasil dan Pembahasan}


% Berikut ini adalah yang perlu diperhatikan untuk mengisi bab hasil dan pembahasan:

% \begin{enumerate}
% 	\item Setiap rumusan masalah boleh memiliki lebih dari 1 tujuan.
% 	\item Setiap subbab harus spesifik menjawab setiap tujuan yang dituliskan.
% 	\item Setiap rumusan masalah boleh dijawab dengan 1 subbab atau lebih.
% \end{enumerate}

% Berikut ini adalah contoh sub bab untuk menjelaskan tujuan penelitian.

\section{Ringkasan Pelaksanaan Eksperimen}
Bab ini menyajikan dan membahas hasil dari 
metodologi penelitian yang diuraikan di Bab 3. 
Eksperimen dilakukan dengan menjalankan 40 
dataset mahasiswa sintetis  melalui empat 
variasi prompt yang berbeda yaitu:
\begin{itemize}
    \item \textit{Baseline Prompt} : \textit{Prompt} dasar tanpa 
    penambahan konteks apa pun.
    \item Lis Materi : penambahan daftar materi pembelajaran
    yang dipelajari mahasiswa dalam batasan studi kasusnya.
    \item LA : Penambahan analisis \textit{learning analytics}
    berdasarkan data sintetis papan Kanban mahasiswa.
    \item ReAct + LA : Penggabungan metode \textit{Reasoning
    and Acting} (ReAct) dengan analisis \textit{learning analytics}.
\end{itemize}

Eksperimen ini dijalankan pada dua model, 
Llama 3.1 8B Instant dan Llama 3.3 70B Versatile, 
melalui Groq API. 
Hasil respons dari LLM kemudian dievaluasi secara 
kuantitatif (menggunakan BERTScore, BARTScore, dan 
LLM-as-a-Judge)  dan kualitatif 
(melalui penilaian ahli manusia)  untuk menjawab 
tiga pertanyaan penelitian.
% Sub bab pertama adalah membahas tujuan penelitian pertama dengan hasil penelitian ke-1. 
% Dapat ditambahkan beberapa sub bab jika diperlukan.

\section{Analisis Perbandingan Metode Peningkatan Umpan Balik (RQ 1)}
Pada bagian ini, disajikan hasil eksperimen yang dilakukan
dengan menggunakan empat variasi prompt yang berbeda
untuk menghasilkan umpan balik pedagogis. Hasil dari
umpan balik yang dihasilkan oleh masing-masing variasi prompt
dianalisis menggunakan metrik kuantitatif seperti BERTScore,
BARTScore, dan LLM-as-a-Judge.
Hasil analisis akan dibandingkan untuk menentukan
apakah ada peningkatan umpan balik
dalam konteks pembelajaran daring.

\subsection{Perbandingan Kualitas Semantik}
% Pendahuluan dan tabel ringkasan skor BERTScore dan BARTScore
Pada bagian ini disajikan ringkasan nilai BERTScore 
dan BARTScore yang diperoleh dari 40 dataset untuk 
setiap variasi prompt dan model. Tabel yang akan disajikan 
menampilkan rata-rata keseluruhan untuk masing-masing metrik. 

Berikut adalah tabel hasil evaluasi dengan metode 
BERTScore untuk setiap variasi prompt dan model
yang tertera pada tabel berikut.

% Requires in preamble: \usepackage{tabularx,array}
\begin{table}[H]
\centering
\caption{Ringkasan Hasil BERTScore per Variasi \textit{Context-Engineering}}
\label{tab:BERTScore_summary}
\begin{tabularx}{\textwidth}{l *{3}{>{\centering\arraybackslash}X}|*{3}{>{\centering\arraybackslash}X}}
\hline
Variant & \multicolumn{3}{c}{8B (Llama 3.1 8B)} & \multicolumn{3}{c}{70B (Llama 3.3 70B)} \\ \cline{2-7}
 & {\fontsize{10}{8}\selectfont \textit{Precision}} & {\fontsize{10}{8}\selectfont \textit{Recall}} & {\fontsize{10}{8}\selectfont F1} & {\fontsize{10}{8}\selectfont \textit{Precision}} & {\fontsize{10}{8}\selectfont \textit{Recall}} & {\fontsize{10}{8}\selectfont F1} \\
\hline
\textit{Baseline}        & 0.6505 & 0.6719 & 0.6608 & 0.6567 & 0.6815 & 0.6687 \\
Lis Materi      & 0.6500 & 0.6700 & 0.6597 & 0.6620 & 0.6805 & \textbf{0.6710}  \\
LA              & 0.6586 & 0.6733 & 0.6657 & 0.6529 & 0.6774 & 0.6648  \\
ReAct + LA      & 0.6613 & 0.6716 & \textbf{0.6663} & 0.6561 & 0.6760 & 0.6658  \\
\hline
\end{tabularx}
\end{table}

Dari Tabel \ref{tab:BERTScore_summary}, nilai F1 BERTScore untuk 
model 8B berkisar antara 0,6597 hingga 0,6663, 
dengan variasi prompt "ReAct + LA" memberikan 
nilai tertinggi (0,6663). Sedangkan pada 
model 70B, nilai F1 BERTScore berkisar 
antara 0,6648 hingga 0,6710, di mana variasi 
"Lis Materi" memperoleh skor tertinggi 
(0,6710). Hal ini mengindikasikan bahwa 
untuk model kecil seperti 8B,
penambahan konteks berupa daftar materi 
pembelajaran dan analisis \textit{learning analytics} 
memberikan peningkatan kualitas semantik 
umpan balik yang dihasilkan oleh LLM. 
Namun, untuk model besar seperti 70B,
penambahan konteks yang lebih sederhana
(dalam hal ini hanya daftar materi pembelajaran)
justru memberikan hasil yang lebih baik
dibandingkan dengan penambahan konteks yang
lebih kompleks seperti analisis
\textit{learning analytics} dan metode ReAct.
% \begin{table}[H]
% \centering
% \caption{Ringkasan Precision per variasi prompt (8B vs 70B).}
% \label{tab:precision_scores}
% \begin{tabularx}{\textwidth}{l *{8}{>{\centering\arraybackslash}X}}
% \hline
% Variant & \multicolumn{4}{c}{8B (Llama 3.1 8B)} & \multicolumn{4}{c}{70B (Llama 3.3 70B)} \\ \cline{2-9}
%  & {\fontsize{10}{8}\selectfont Feedback} & {\fontsize{10}{8}\selectfont Motivasi} & {\fontsize{10}{8}\selectfont Apresiasi} & {\fontsize{10}{8}\selectfont \textbf{Overall}} & {\fontsize{10}{8}\selectfont Feedback} & {\fontsize{10}{8}\selectfont Motivasi} & {\fontsize{10}{8}\selectfont Apresiasi} & {\fontsize{10}{8}\selectfont \textbf{Overall}} \\
% \hline
% Baseline        & 0.6349 & 0.6575 & 0.6590 & 0.6505 & 0.6411 & 0.6530 & 0.6759 & 0.6567 \\
% List Materi     & 0.6368 & 0.6564 & 0.6568 & 0.6500 & 0.6487 & 0.6600 & 0.6772 & 0.6620 \\
% LA              & 0.6393 & 0.6675 & 0.6690 & 0.6586 & 0.6416 & 0.6458 & 0.6712 & 0.6529 \\
% LA + ReAct      & 0.6565 & 0.6570 & 0.6705 & 0.6613 & 0.6531 & 0.6451 & 0.6702 & 0.6561 \\
% \hline
% \end{tabularx}
% \end{table}

% \begin{table}[H]
% \centering
% \caption{Ringkasan Recall per variasi prompt (8B vs 70B).}
% \label{tab:recall_scores}
% \begin{tabularx}{\textwidth}{l *{8}{>{\centering\arraybackslash}X}}
% \hline
% Variant & \multicolumn{4}{c}{8B (Llama 3.1 8B)} & \multicolumn{4}{c}{70B (Llama 3.3 70B)} \\ \cline{2-9}
%  & {\fontsize{10}{8}\selectfont Feedback} & {\fontsize{10}{8}\selectfont Motivasi} & {\fontsize{10}{8}\selectfont Apresiasi} & {\fontsize{10}{8}\selectfont \textbf{Overall}} & {\fontsize{10}{8}\selectfont Feedback} & {\fontsize{10}{8}\selectfont Motivasi} & {\fontsize{10}{8}\selectfont Apresiasi} & {\fontsize{10}{8}\selectfont \textbf{Overall}} \\
% \hline
% Baseline        & 0.6753 & 0.6712 & 0.6691 & 0.6719 & 0.6809 & 0.6736 & 0.6900 & 0.6815 \\
% List Materi     & 0.6720 & 0.6680 & 0.6701 & 0.6700 & 0.6775 & 0.6743 & 0.6897 & 0.6805 \\
% LA              & 0.6715 & 0.6692 & 0.6791 & 0.6733 & 0.6747 & 0.6698 & 0.6876 & 0.6774 \\
% LA + ReAct      & 0.6729 & 0.6640 & 0.6780 & 0.6716 & 0.6742 & 0.6666 & 0.6872 & 0.6760 \\
% \hline
% \end{tabularx}
% \end{table}

% Catatan: ganti nilai contoh di atas dengan hasil pengolahan skor aktual Anda.
Selain hasil dari BERTScore, berikut adalah tabel hasil evaluasi dengan metode 
BARTScore untuk setiap variasi prompt dan model
yang tertera pada tabel berikut.

% buat tabel BARTScore di sini
\begin{table}[H]
\centering
\caption{Ringkasan Hasil BARTScore per Variasi \textit{Context-Engineering}}
\label{tab:BARTScore_summary}
\begin{tabularx}{\textwidth}{l *{3}{>{\centering\arraybackslash}X}|*{3}{>{\centering\arraybackslash}X}}
\hline
Variant & \multicolumn{3}{c}{8B (Llama 3.1 8B)} & \multicolumn{3}{c}{70B (Llama 3.3 70B)} \\ \cline{2-7}
 & {\fontsize{9}{10}\selectfont \textit{Precision}} & {\fontsize{9}{10}\selectfont \textit{Recall}} & {\fontsize{9}{10}\selectfont F1-Score} & {\fontsize{9}{10}\selectfont \textit{Precision}} & {\fontsize{9}{10}\selectfont \textit{Recall}} & {\fontsize{9}{10}\selectfont F1-Score} \\
\hline
\textit{Baseline}        & {\fontsize{9}{10}\selectfont -11.0907} & {\fontsize{9}{10}\selectfont -12.3656} & {\fontsize{9}{10}\selectfont -11.7282} & {\fontsize{9}{10}\selectfont -11.1677} & {\fontsize{9}{10}\selectfont -12.2132} & {\fontsize{9}{10}\selectfont -11.6905} \\
Lis Materi      & {\fontsize{9}{10}\selectfont -11.1788} & {\fontsize{9}{10}\selectfont -12.5383} & {\fontsize{9}{10}\selectfont -11.8586} & {\fontsize{9}{10}\selectfont -11.0399} & {\fontsize{9}{10}\selectfont -12.1851} & {\fontsize{9}{10}\selectfont \textbf{-11.6125}} \\
LA              & {\fontsize{9}{10}\selectfont -10.9045} & {\fontsize{9}{10}\selectfont -12.1800} & {\fontsize{9}{10}\selectfont -11.5422} & {\fontsize{9}{10}\selectfont -11.2543} & {\fontsize{9}{10}\selectfont -12.1306} & {\fontsize{9}{10}\selectfont -11.6924} \\
ReAct + LA      & {\fontsize{9}{10}\selectfont -10.5800} & {\fontsize{9}{10}\selectfont -12.2345} & {\fontsize{9}{10}\selectfont \textbf{-11.4073}} & {\fontsize{9}{10}\selectfont -10.8933} & {\fontsize{9}{10}\selectfont -12.4587} & {\fontsize{9}{10}\selectfont -11.6760} \\
\hline
\end{tabularx}
\end{table}

Pada metrik BARTScore (Tabel \ref{tab:BARTScore_summary}), 
semua variasi 
prompt menunjukkan nilai negatif, yang merupakan 
karakteristik dari metrik ini dalam pengukuran 
kualitas teks. Model 8B menunjukkan peningkatan 
performa dari \textit{Baseline} (-11,7282) ke variasi "\textit{ReAct + LA}" 
(-11,4073), yang berarti kualitas umpan balik secara 
keseluruhan membaik dengan penambahan konteks dan metode 
\textit{reasoning}. Model 70B juga menunjukkan pola serupa, 
meskipun variasi "Lis Materi" dan "\textit{ReAct + LA}" 
memiliki nilai yang cukup berdekatan, dengan "Lis Materi" 
sedikit lebih baik. Untuk model kecil, metrik ini mendukung 
temuan bahwa penambahan konteks dan metode \textit{reasoning} 
memberikan kontribusi positif terhadap kualitas semantik 
umpan balik. Namun, untuk model besar, semakin diberikan
konteks yang kompleks tidak selalu menghasilkan
peningkatan yang signifikan dalam kualitas semantik.


\subsection{Perbandingan Kualitas Relevansi}
% Pendahuluan dan tabel ringkasan skor LLM-as-a-Judge
Pada bagian ini disajikan ringkasan nilai LLM-as-a-Judge
yang diperoleh dari 40 dataset untuk
setiap variasi prompt dan model. Tabel yang disajikan
menampilkan rata-rata keseluruhan untuk masing-masing metrik.
Berikut adalah tabel hasil evaluasi dengan metode
LLM-as-a-Judge untuk setiap variasi prompt dan model:
% buat tabel LLM-as-a-Judge di sini
\begin{table}[H]
\centering
\caption{Ringkasan Overall LLM-as-a-Judge per Metode}
\label{tab:llm_judge_overall}
\small
\begin{tabularx}{\textwidth}{l *{3}{>{\centering\arraybackslash}X}|*{3}{>{\centering\arraybackslash}X}}
\hline
Metode & \multicolumn{3}{c|}{8B (Llama 3.1 8B)} & \multicolumn{3}{c}{70B (Llama 3.3 70B)} \\ \cline{2-7}
 & \textit{Precision} & \textit{Recall} & F1-Score & \textit{Precision} & \textit{Recall} & F1-Score \\
\hline
\textit{Baseline}        & 1.0000 & 0.4567 & 0.5371 & 1.0000 & 0.6417 & 0.7271 \\
Lis Materi      & 0.9333 & 0.4767 & 0.5627 & 1.0000 & 0.6600 & \textbf{0.7571} \\
LA              & 1.0000 & 0.5017 & 0.5896 & 0.9333 & 0.5717 & 0.6399 \\
ReAct + LA  & 1.0000 & 0.5217 & \textbf{0.6121} & 1.0000 & 0.5883 & 0.6715 \\
\hline
\end{tabularx}
\end{table}

Analisis kualitas relevansi menggunakan metode LLM-as-a-Judge 
memberikan gambaran yang lebih detail terkait aspek relevansi 
umpan balik yang dihasilkan. Dari Tabel \ref{tab:llm_judge_overall}, 
untuk model 8B, variasi "ReAct + LA" memberikan nilai 
F1-Score tertinggi sebesar 0,6121, yang menunjukkan 
keseimbangan terbaik antara presisi dan recall dalam 
menilai relevansi umpan balik. Variasi baseline 
memiliki nilai F1-Score terendah (0,5371), menandakan 
bahwa tanpa konteks tambahan, relevansi umpan balik 
lebih rendah. Pada model 70B, variasi "Lis Materi" 
memberikan nilai F1-Score tertinggi sebesar 0,7571, 
diikuti oleh baseline (0,7271) dan "ReAct + LA" 
(0,6715). Hal ini menunjukkan bahwa pada model yang 
lebih besar, penambahan daftar materi pembelajaran 
secara khusus meningkatkan relevansi umpan balik 
secara signifikan.

\subsection{Analisis Evaluasi Kuantitatif}
Analisis evaluasi kuantitatif pada eksperimen ini 
menunjukkan pola yang konsisten terkait pengaruh variasi 
\textit{prompt} dan ukuran model terhadap kualitas umpan balik 
pedagogis yang dihasilkan. Berdasarkan metrik BERTScore dan 
BARTScore, model 8B (Llama 3.1 8B) menunjukkan peningkatan 
kualitas semantik yang paling signifikan ketika metode 
\textit{reasoning} dan konteks \textit{learning analytics} (\textit{ReAct + LA}) 
diterapkan, dengan nilai F1 BERTScore tertinggi sebesar 
0,6663 dan peningkatan BARTScore dari -11,7282 (\textit{Baseline}) 
menjadi -11,4073 (ReAct + LA). Hal ini mengindikasikan 
bahwa penambahan konteks yang kompleks dan metode \textit{reasoning} 
mampu memperbaiki akurasi dan kualitas semantik umpan 
balik pada model yang lebih kecil.

Sebaliknya, pada model 70B (Llama 3.3 70B), peningkatan 
kualitas semantik lebih optimal dicapai dengan penambahan 
konteks yang lebih sederhana, yakni daftar materi 
pembelajaran (Lis Materi), yang menghasilkan nilai F1 
BERTScore tertinggi sebesar 0,6710 dan BARTScore yang 
lebih baik dibandingkan variasi lain. Penambahan
konteks kompleks seperti LA dan ReAct + LA tidak 
memberikan peningkatan signifikan, bahkan cenderung 
stagnan atau sedikit menurun. Ini menunjukkan bahwa 
model besar mungkin sudah memiliki kapasitas internal 
yang cukup untuk memahami konteks pembelajaran, sehingga 
konteks tambahan yang terlalu kompleks tidak selalu 
memperbaiki hasil.

Dari sisi relevansi umpan balik yang diukur menggunakan 
metode \textit{LLM-as-a-Judge}, model 8B kembali menunjukkan 
performa terbaik dengan variasi ReAct + LA, yang 
menghasilkan F1-Score tertinggi 0,6121. Ini menegaskan 
bahwa \textit{reasoning} dan analisis \textit{learning analytics} 
memberikan kontribusi positif dalam meningkatkan relevansi 
umpan balik pada model berukuran kecil. Sedangkan pada 
model 70B, variasi Lis Materi mendominasi dengan F1-Score 
0,7571, menandakan bahwa penambahan konteks berupa daftar 
materi pembelajaran secara khusus meningkatkan relevansi 
umpan balik secara signifikan, lebih baik dibandingkan 
Baseline dan ReAct + LA.

Secara keseluruhan, evaluasi kuantitatif ini mengindikasikan 
bahwa strategi peningkatan umpan balik pedagogis perlu 
disesuaikan dengan kapasitas model LLM yang digunakan. 
Model kecil memperoleh manfaat lebih besar dari penambahan 
konteks yang kompleks dan metode \textit{reasoning}, sedangkan 
model besar lebih efektif dengan konteks yang sederhana 
dan fokus pada materi pembelajaran. Pendekatan ini 
penting untuk mengoptimalkan kualitas semantik dan 
relevansi umpan balik dalam aplikasi pembelajaran daring.

Dari hasil evaluasi kuantitatif menggunakan 
BERTScore, BARTScore, dan \textit{LLM-as-a-Judge}, diputuskan untuk
mengambil Llama 3.1 8B dengan variasi "ReAct + LA" 
untuk di analisis kualitatif lebih lanjut pada bagian berikutnya.
Hal ini karena pada eksperimen kuantitatif, model memberikan
hasil pengamatan eksperimen yang sejalan dengan seiring 
bertambahnya konteks yang diberikan, semakin baik hasilnya.
Alasan untuk tidak diambilnya Llama 3.3 70B adalah karena pada
model tersebut, semakin banyak konteks yang diberikan, hasil 
yang didapatkan semakin menurun sehingga perlu diketahui terlebih
dahulu penyebabnya terjadinya hal tersebut. Pemilihan model yang
lebih kecil juga mendukung efisiensi komputasi dan kecepatan 
respons sehingga lebih praktis untuk diterapkan dalam skala luas.
Dengan begitu, analisis kualitatif dapat difokuskan pada model yang
menunjukkan potensi terbaik dalam efisiensinya.

% Dari hasil BERTScore, terlihat bahwa model 8B 
% memperoleh peningkatan kualitas semantik umpan 
% balik tertinggi pada variasi "ReAct + LA" dengan 
% nilai F1 sebesar 0,6663. Hal ini menunjukkan bahwa 
% penggabungan metode reasoning dan analisis learning 
% analytics secara sinergis memperbaiki akurasi dan 
% keutuhan konteks semantik dalam umpan balik 
% yang dihasilkan oleh model kecil. Sebaliknya, 
% pada model 70B, variasi "Lis Materi" memberikan 
% hasil terbaik dengan F1 sebesar 0,6710, menandakan 
% bahwa penambahan konteks berupa daftar materi p
% embelajaran sudah cukup untuk meningkatkan kualitas 
% semantik pada model yang lebih besar, tanpa perlu 
% tambahan kompleksitas seperti reasoning atau analisis 
% learning analytics.

% Hasil BARTScore mendukung temuan tersebut dengan pola 
% yang serupa, di mana model 8B menunjukkan peningkatan 
% performa dari baseline ke "ReAct + LA", mengindikasikan 
% bahwa konteks dan metode reasoning berkontribusi positif 
% terhadap kualitas teks. Namun, model 70B menunjukkan 
% perbedaan yang lebih kecil antar variasi, dengan "Lis Materi" 
% sedikit unggul, yang mengindikasikan bahwa kompleksitas 
% konteks tambahan tidak selalu berbanding lurus dengan 
% peningkatan kualitas pada model besar.

% Evaluasi menggunakan LLM-as-a-Judge yang mengukur 
% relevansi umpan balik juga memperlihatkan pola konsisten. 
% Pada model 8B, variasi "ReAct + LA" menghasilkan nilai 
% MacroF1 tertinggi (0,6121), menunjukkan keseimbangan 
% terbaik antara presisi dan recall dalam relevansi umpan 
% balik. Sementara itu, model 70B menunjukkan nilai 
% tertinggi pada variasi "Lis Materi" (MacroF1 0,7571), 
% mengindikasikan bahwa penambahan konteks sederhana 
% berupa daftar materi pembelajaran cukup efektif untuk 
% meningkatkan relevansi umpan balik pada model yang 
% lebih besar.

% Secara keseluruhan, analisis kuantitatif ini menggarisbawahi 
% bahwa peningkatan kualitas umpan balik tidak hanya bergantung 
% pada kompleksitas konteks yang diberikan, tetapi juga sangat 
% dipengaruhi oleh kapasitas model yang digunakan. Model kecil 
% (8B) mendapatkan manfaat yang lebih signifikan dari kombinasi 
% konteks dan metode reasoning yang kompleks, sedangkan model 
% besar (70B) cenderung merespon lebih baik terhadap konteks 
% yang sederhana dan fokus. Temuan ini memberikan arahan penting 
% dalam perancangan prompt context-engineering yang efisien dan 
% efektif sesuai dengan kapasitas model LLM yang digunakan.

% Secara keseluruhan, hasil analisis pada kedua 
% metrik tersebut mengindikasikan bahwa penambahan 
% konteks pembelajaran dan integrasi metode \textit{reasoning} 
% seperti ReAct dengan \textit{learning analytics} dapat 
% meningkatkan kualitas semantik dan relevansi umpan 
% balik pedagogis yang dihasilkan oleh \textit{LLM}. Namun, 
% peningkatan ini bervariasi tergantung pada model 
% yang digunakan dan jenis konteks yang ditambahkan. 
% Model besar seperti 70B cenderung memberikan hasil yang lebih baik 
% secara keseluruhan, terutama dalam aspek relevansi, 
% sementara model 8B menunjukkan peningkatan yang lebih 
% nyata ketika metode ReAct digabungkan dengan 
% \textit{learning analytics}. Temuan ini mendukung hipotesis 
% bahwa pengayaan konteks dan metode \textit{reasoning} dapat 
% memperbaiki kualitas umpan balik dalam pembelajaran 
% daring, meskipun perlu diperhatikan bahwa peningkatan 
% kualitas tidak selalu linier dan bergantung pada 
% kombinasi metode dan model yang digunakan.

\section{Analisis Pengaruh Ukuran Model (RQ 3)}
Tapi kenapa untuk model yang 8B semakin diberikan tugas teknik context engineering yang kompleks, semakin bagus nilai BERTScore dan BARTScore nya? Tidak seperti 70B yang berkebalikan


The differences in how 8B and 70B large language models (LLMs) respond to complex context engineering can be attributed to several factors:

Capacity and Efficiency: The 70B model has a larger parameter size, enabling it to naturally capture more intricate patterns and nuances in the language without relying heavily on complex prompts. Conversely, the 8B model, being smaller, might benefit more from detailed prompts, which can provide additional cues and context it might not capture on its own.

Focus and Disambiguation: For a smaller model like the 8B, complex prompts can help disambiguate instructions and focus the model's attention on the specific task, potentially boosting its performance. Meanwhile, the 70B model might already excel at handling diverse linguistic structures, making complex prompts less necessary and sometimes even counterproductive if they introduce noise or irrelevant complexity.

Prompt Engineering Benefits: Smaller models often gain more from prompt engineering because it can serve as a guide or scaffold, assisting these models in focusing on specific details that they might not inherently prioritize. In contrast, larger models typically have more inherent capacity and don't need as much external guidance through tailored prompts.

Data Dependency and Specialization: The performance improvements in the 8B model with complex prompts might also be a result of how its training data aligns with the prompt strategy used. Models can have optimizations and learnings that align better with certain styles of prompts, leading to benefits when such prompts are used.

Differential Performance across Models: As larger models often require substantial resources and finely tuned prompts may not make as significant a difference in outcome because slight improvements are already achieved through their sheer size, smaller models can exhibit more visible gains when complex prompts are correctly applied, as these guide their more limited capacity toward optimal performance.

Thus, while both models may derive some benefit from engineered prompts, the additional complexity can be more salient and beneficial for the 8B model as a support mechanism for its smaller capacity, enhancing its BERTScore and BARTScore significantly compared to the 70B model, which naturally excels with simpler prompts due to its larger inherent capabilities.

\section{Validasi dan Persepsi Ahli Manusia (RQ 2)}
Untuk menjawab \textit{research question} kedua (\textit{RQ 2}) mengenai 
persepsi ahli manusia terhadap kualitas respon yang 
dihasilkan oleh \textit{Feedback Agent}, dilakukan proses validasi pakar. 
Validasi ini bertujuan untuk menilai efektivitas \textit{Feedback Agent} 
sebagai asisten pembelajaran dalam memberikan \textit{feedback}, 
\textit{motivation}, dan \textit{appreciation}. Proses ini melibatkan seorang ahli 
dalam bidang Psikologi Pembelajaran, Shafira Anissa, dari 
Universitas Gadjah Mada.

Penilaian pakar dilakukan menggunakan instrumen kuesioner 
terstruktur yang mencakup tiga aspek utama: \textit{Feedback Quality} 
(Kualitas Umpan Balik), \textit{Motivational Support} 
(Dukungan Motivasi), dan \textit{Appreciation/Affective Support} 
(Apresiasi dan Penguatan Emosional). 
Pakar memberikan penilaian kuantitatif menggunakan skala \textit{Likert} 1--5  
serta umpan balik kualitatif melalui pertanyaan terbuka.
\subsection{Analisis Penilaian Rubrik Ahli Manusia}
Penilaian kuantitatif oleh pakar dirangkum berdasarkan 15 
butir pertanyaan yang mewakili tiga aspek evaluasi. Skor 1 
mewakili "Sangat Tidak Setuju", 2 "Tidak Setuju", 3 "Netral", 4 
"Setuju", dan 5 "Sangat Setuju".


\subsubsection{Aspek Kualitas Umpan Balik} 
Aspek ini menilai seberapa informatif, 
jelas, relevan, dan konstruktif 
umpan balik yang diberikan chatbot. Hasil penilaian pakar 
adalah sebagai berikut dapat dilihat pada Tabel \ref{tab:expert_feedback_quality}:

\begin{table}[H]
\centering
\caption{Penilaian Aspek \textit{Feedback} Umpan Balik oleh Pakar}
\label{tab:expert_feedback_quality}
\begin{tabularx}{\textwidth}{l X >{\centering\arraybackslash}p{1.2cm}}
\hline
Indikator & Pertanyaan Angket & Skor \\
\hline
F1 (Kejelasan) & Feedback dari chatbot mudah saya pahami. & 4 \\
F2 (Spesifisitas) & Chatbot memberikan saran perbaikan yang spesifik terhadap pekerjaan saya. & 4 \\
F3 (Relevansi) & Feedback dari chatbot sesuai dengan aktivitas belajar yang sedang saya lakukan. & 3 \\
F4 (Konstruktivitas) & Umpan balik dari chatbot membantu saya memperbaiki kesalahan dalam belajar. & 4 \\
F5 (Feed-forward) & Chatbot memberikan arahan atau langkah selanjutnya yang dapat saya lakukan. & 4 \\
\hline
\end{tabularx}
\end{table}


% F1 (Kejelasan): 4 (Setuju)

% F2 (Spesifisitas): 4 (Setuju)

% F3 (Relevansi): 3 (Netral)

% F4 (Konstruktivitas): 4 (Setuju)

% F5 (Feed-forward): 4 (Setuju)

% Mayoritas indikator feedback dinilai positif ("Setuju"). 
% Namun, indikator Relevansi (F3) mendapat nilai "Netral", 
% yang mengindikasikan bahwa kesesuaian umpan balik dengan 
% aktivitas pengguna terkadang belum optimal.

Aspek ini secara umum dinilai positif (Skor 4, "Setuju") 
pada mayoritas indikator, seperti Kejelasan (F1), Spesifisitas 
(F2), Konstruktivitas (F4), dan Feed-forward (F5). Hal ini 
menunjukkan \textit{chatbot} telah berhasil menyampaikan umpan balik 
yang dapat dipahami dan berorientasi pada perbaikan. Namun, 
skor Netral (Skor 3) pada F3 (Relevansi) menjadi catatan kritis. 
Ini mengindikasikan bahwa meskipun umpan balik yang diberikan 
berkualitas baik, namun terkadang belum sepenuhnya relevan 
atau selaras dengan konteks aktivitas spesifik yang sedang 
dilakukan pengguna.

Kekurangan ini terletak pada kurangnya spesifisitas kontekstual. 
Sebagai contoh, respon seperti "Saya melihat bahwa Anda memiliki 
satu kartu di Planning..." atau "kartu yang masih dalam fase Monitoring..." 
tidak secara eksplisit menyebutkan \textit{nama} kartu yang dimaksud. Hal 
ini membuat umpan balik terasa generik dan kurang relevan, terutama 
jika pengguna memiliki beberapa kartu dalam fase yang sama. Sebaliknya, 
respon yang lebih efektif 
% (seperti pada data contoh 10) 
secara 
spesifik menyebutkan nama kartu, seperti 
"'Object Oriented Programming [CS101]'". Inkonsistensi dalam menyebutkan 
konteks spesifik inilah yang diduga kuat menjadi penyebab skor F3 
(Relevansi) dinilai Netral.

Analisis tersebut didukung dengan hasil wawancara
dengan ahli. Ahli secara eksplisit mengkritik bahwa feedback 
"nggak spesifik"  pada \textit{timestamp} [00:02:23,946]. Contohnya adalah \textit{chatbot} mengatakan 
"kamu ada lho kartu yang masih di fase planning... tapi kartu yang mana?" 
(\textit{timestamp} [00:02:23,946]).

\subsubsection{Aspek Dukungan Motivasi} 
Berikutnya, aspek ini menilai kemampuan chatbot dalam meningkatkan motivasi 
belajar. Hasil penilaian pakar dapat dilihat pada Tabel \ref{tab:expert_motivational_support}:
\begin{table}[H]
\centering
\caption{Penilaian Aspek Dukungan \textit{Motivation} oleh Pakar}
\label{tab:expert_motivational_support}
\begin{tabularx}{\textwidth}{l X >{\centering\arraybackslash}p{1.2cm}}
\hline  
Indikator & Pertanyaan Angket & Skor \\
\hline
M1 (Dorongan Intrinsik) & Chatbot membuat saya lebih bersemangat untuk belajar. & 4 \\
M2 (Dorongan Tujuan) & Chatbot membantu saya tetap fokus pada tujuan pembelajaran. & 5 \\
M3 (Otonomi) & Chatbot memberi saya kebebasan dalam memilih cara belajar yang sesuai. & 4 \\
M4 (Kompetensi) & Respon chatbot membuat saya merasa lebih percaya diri dalam belajar. & 4 \\
M5 (Keterhubungan) & Chatbot memberikan dukungan yang terasa empatik dan memahami kondisi saya. & 3 \\
\hline 
\end{tabularx}
\end{table}

% M1 (Dorongan Intrinsik): 4 (Setuju)

% M2 (Dorongan Tujuan): 5 (Sangat Setuju)

% M3 (Otonomi): 4 (Setuju)

% M4 (Kompetensi): 4 (Setuju)

% M5 (Keterhubungan): 3 (Netral)

% Aspek motivasi dinilai kuat, terutama dalam membantu 
% pengguna tetap fokus pada tujuan pembelajaran (M2). Akan 
% tetapi, aspek Keterhubungan (M5), yang menilai dukungan 
% empatik, dinilai "Netral".

Aspek ini menunjukkan kinerja terkuat dari chatbot, 
dibuktikan dengan skor Sangat Setuju (Skor 5) pada M2 
(Dorongan Tujuan). Chatbot dinilai sangat efektif dalam 
menjaga fokus pengguna pada tujuan pembelajaran. Indikator 
lain seperti Dorongan Intrinsik (M1), Otonomi (M3), 
dan Kompetensi (M4) juga dinilai kuat (Skor 4). 
Namun, sama seperti aspek feedback, terdapat skor 
Netral (Skor 3) pada M5 (Keterhubungan). Ini menyiratkan 
bahwa chatbot berhasil secara fungsional dalam memotivasi, 
namun gagal dalam membangun koneksi empatik atau relasional 
dengan pengguna.

Kemungkinan penyebab skor rendah pada 
M5 (Keterhubungan) ini dijelaskan 
melalui analisis pola kalimat pada data respon motivasi. 
Terdapat repetisi formulaik yang sangat tinggi. Sebagian 
besar respon motivasi
% (misalnya pada contoh 4, 5, 9, 12) 
dimulai dengan frasa yang hampir identik: "Saya percaya 
bahwa Anda memiliki kemampuan untuk..." atau "Anda memiliki 
kemampuan untuk...". Meskipun tujuannya positif, 
pengulangan yang konstan ini membuat interaksi terasa 
skriptual, robotik, dan tidak tulus. Kegagalan untuk 
memvariasikan bahasa motivasi inilah yang menyebabkan 
respon gagal terasa "empatik" atau "memahami kondisi saya", 
sehingga kemungkinan besar menjadi alasan 
ahli memberikan penilaian Netral. Hal ini juga disebutkan
oleh ahli dalam wawancara di mana ahli mengeluhkan bahwa 
responnya "agak melaton" (monoton) dan "bentukannya itu-itu aja" 
pada \textit{timestamp} [00:03:44,665].

\subsubsection{Aspek Apresiasi dan Penguatan Emosional} 
Aspek ini menilai kemampuan chatbot dalam memberikan pengakuan 
dan dukungan emosional positif. Hasil penilaian pakar dapat 
dilihat pada Tabel \ref{tab:expert_appreciation_support}:

\begin{table}[H]
\centering 
\caption{Penilaian Aspek \textit{Appreciation} dan Penguatan Emosional oleh Pakar}
\label{tab:expert_appreciation_support}
\begin{tabularx}{\textwidth}{l X >{\centering\arraybackslash}p{1.2cm}}
\hline
Indikator & Pertanyaan Angket & Skor \\
\hline
A1 (Pengakuan Usaha) & Chatbot memberikan apresiasi terhadap usaha saya, bukan hanya hasilnya. & 5 \\
A2 (Pujian Proporsional) & Pujian dari chatbot terasa tulus dan sesuai dengan pencapaian saya. & 3 \\
A3 (Peningkatan Rasa Percaya Diri) & Ucapan apresiasi dari chatbot meningkatkan rasa percaya diri saya. & 4 \\
A4 (Dukungan Emosional) & Chatbot memberikan respon yang menenangkan ketika saya mengalami kesulitan. & 4 \\
A5 (Penguatan Positif) & Bahasa yang digunakan chatbot membuat saya merasa dihargai. & 3 \\
\hline
\end{tabularx}
\end{table}

% A1 (Pengakuan Usaha): 5 (Sangat Setuju)

% A2 (Pujian Proporsional): 3 (Netral)

% A3 (Peningkatan Rasa Percaya Diri): 4 (Setuju)

% A4 (Dukungan Emosional): 4 (Setuju)

% A5 (Penguatan Positif): 3 (Netral)


% Chatbot dinilai sangat baik dalam memberikan apresiasi 
% terhadap usaha (A1). Namun, aspek Pujian Proporsional 
% (A2) dan Penguatan Positif (A5) dinilai "Netral", menunjukkan 
% bahwa ketulusan pujian dan penggunaan bahasa positif masih 
% perlu ditingkatkan.

Aspek ini menunjukkan hasil yang paling beragam. 
Chatbot dinilai superior dalam Pengakuan Usaha 
(A1, Skor 5), yang sejalan dengan prinsip penguatan 
positif. Namun, aspek ini juga memiliki dua skor Netral 
(Skor 3), yaitu pada A2 (Pujian Proporsional) dan A5 
(Penguatan Positif). Ini menunjukkan sebuah paradoks: 
chatbot memberikan apresiasi atas usaha, tetapi cara 
penyampaiannya (A5) dan proporsinya (A2) terasa kurang 
tulus atau tidak pas, sehingga membuatnya terasa 
kurang efektif dalam membangun rasa dihargai secara emosional.

Kemungkinan penyebab utama dari skor rendah ini adalah penggunaan 
"apresiasi yang dinegasikan" atau "pujian dengan 'tetapi'". 
Beberapa contoh respon 
% (misalnya contoh 1 dan 7) 
menggunakan pola yang problematik seperti 
"Saya mengapresiasi usaha Anda..., \textbf{tetapi} saya ingin 
Anda terus maju..." atau "Saya mengapresiasi usaha Anda... 
\textbf{tetapi} masih ada banyak yang belum diselesaikan." 
Penggunaan kata 'tetapi' segera setelah pujian secara efektif 
membatalkan penguatan positif tersebut dan justru mengubah 
apresiasi menjadi \textit{feedback} korektif. Hal ini membuat 
pujian terasa tidak tulus (A2) dan bahasa yang digunakan tidak 
terasa menghargai (A5), yang menjelaskan penilaian Netral dari 
pakar. Terdapat pula tumpang tindih fungsi, di mana pesan 
apresiasi dicampurkan dengan \textit{feedback}, yang 
mengurangi efektivitas keduanya. Dalam sesi wawancara, ahli juga 
secara eksplisit mengkritik penggunaan bahasa yang "negatif" 
seperti "utang" dan "hanya selesai" (\textit{timestamp} [00:00:01,553]). Ahli  
meminta bahasa yang "lebih positif gitu ya, yang 
uplifting" (\textit{timestamp} [00:00:01,553]).


\subsection{Analisis Kualitatif Terbuka}
Analisis kualitatif dilakukan terhadap jawaban pakar 
pada tiga pertanyaan terbuka. Umpan balik ini diuraikan 
berdasarkan struktur pertanyaan kuesioner untuk 
mendapatkan wawasan mendalam mengenai persepsi ahli 
terhadap kualitas, kekurangan, dan potensi kontribusi \textit{chatbot}.

\subsubsection{Aspek Kualitas Paling Penting}
Pada pertanyaan pertama mengenai aspek paling 
penting dalam menilai \textit{chatbot} pembelajaran, pakar 
menekankan pada dukungan substantif terhadap proses 
belajar. Aspek terpenting yang disoroti adalah 
"sejauh mana respon yang diberikan oleh \textit{chatbot} 
mendukung pembelajaran peserta didik".

Kualitas ini, menurut pakar, secara spesifik dapat 
diukur melalui "ketepatan saran yang diberikan". 
Hal ini mengindikasikan bahwa fungsionalitas pedagogis 
dan akurasi konten (seberapa baik saran \textit{chatbot} dalam 
memandu mahasiswa) dinilai sebagai prioritas utama, 
melampaui aspek interaksional semata.

\subsubsection{Saran Pengembangan \textit{Chatbot}}
Pertanyaan kedua menggali saran konkret untuk 
pengembangan \textit{chatbot} agar dapat memberikan dukungan 
yang lebih efektif. Pakar memberikan lima rekomendasi utama:

\begin{enumerate}
    \item \textbf{Penyesuaian Tonalitas Bahasa}: Kritik utama adalah 
pada penggunaan bahasa yang "terlalu kaku". Pakar 
menyarankan agar bahasa diubah menjadi "lebih ringan" 
sehingga "lebih menarik dan engaging" bagi target 
demografi, yaitu mahasiswa. Kritik ini menyoroti adanya 
kontradiksi dalam desain \textit{prompt}. 
Templat \texttt{INITIAL\_PROMPT} menginstruksikan 
model untuk menggunakan nada "ramah, personal, mudah 
didekati". Namun, template \texttt{ASSESSMENT\_PROMPT} 
mengubah persona tersebut menjadi "seorang dosen yang mengevaluasi" 
dan secara eksplisit mewajibkan keluaran dalam "Bahasa Indonesia baku 
('Anda')". Model tampaknya lebih memprioritaskan instruksi persona 
"dosen" yang "baku" dalam \texttt{ASSESSMENT\_PROMPT}, yang secara 
langsung menghasilkan nada "kaku" dan mengabaikan nada "ramah" dari 
\texttt{INITIAL\_PROMPT}. Hal ini menunjukkan gagalnya model dalam 
mematuhi instruksi ini secara konsisten.
    \item \textbf{Eliminasi Diksi Negatif}: Terkait dengan poin pertama, 
pakar mengidentifikasi penggunaan diksi yang terasa 
"negatif", seperti "utang refleksi" dan "hanya x\%". 
Penggunaan kata-kata ini dinilai berpotensi "mengurangi 
motivasi peserta didik". Saran yang diberikan adalah agar 
\textit{chatbot} konsisten berfokus pada "bahasa positif yang \textit{uplifting}".
Celah ini muncul bukan karena \textit{prompt} menyuruh model 
menggunakan bahasa negatif, melainkan karena \textit{prompt} 
tidak memiliki \textit{guardrail} (pembatas) negatif. 
\texttt{ASSESSMENT\_PROMPT} menginstruksikan model untuk memberi 
"komentar objektif dan konstruktif berdasarkan data" dan menggunakan 
"bukti dari data". Variabel di dalam \texttt{<ANALISIS\_JSON>} 
berisi \textit{string} seperti "\texttt{reflection\_debt}" dan kalkulasi 
"\texttt{checklist\_pct}" sehingga model, dalam upayanya untuk "objektif", 
hanya melaporkan data tersebut secara harfiah. \textit{Prompt} 
tidak memiliki instruksi sekunder untuk memparafrasa data negatif 
menjadi kalimat yang \textit{uplifting}. Hal ini menunjukkan perlunya 
penambahan \textit{guardrail} dalam \textit{prompt} untuk menghindari 
penggunaan diksi negatif secara eksplisit di penelitian selanjutnya.
    \item \textbf{Tumpang Tindih Kategori Respon}: Pakar mengobservasi 
adanya tumpang tindih (overlap) konseptual, terutama 
antara feedback dengan motivasi, dan motivasi dengan 
apresiasi. Ditemukan bahwa bagian motivasi terkadang masih 
berisi elemen feedback (misalnya, "anda sebaiknya memeriksa..."). 
Meskipun demikian, pakar menilai ini "bukan masalah besar" 
karena ketiga aspek tersebut secara alami "memang saling 
beririsan" dalam satu kesatuan interaksi \textit{chatbot}.
Ini adalah kegagalan model dalam mematuhi \textit{separation of 
concerns} (pemisahan ranah) yang ketat yang telah dirancang 
dalam \texttt{ASSESSMENT\_PROMPT}. \textit{Prompt} tersebut 
sudah dengan benar mendefinisikan tiga keluaran JSON yang 
berbeda (\texttt{'feedback'}, \texttt{'motivasi'}, 
\texttt{'apresiasi'}) dengan tujuan yang berbeda. Namun, 
model mengalami "kebocoran konten" (\textit{content bleeding}), 
di mana instruksi untuk \texttt{'feedback'} ("langkah 
berikutnya yang konstruktif") "bocor" ke dalam \textit{string} 
\texttt{'motivasi'}. Ini menunjukkan bahwa meskipun struktur 
JSON dipaksakan, model masih kesulitan memisahkan niat 
pedagogis antar-kunci.
    \item \textbf{Kurangnya Spesifisitas Feedback}: Poin kritik penting 
lainnya adalah feedback yang diberikan "belum memberikan 
saran yang spesifik". Pakar memberi contoh respon 
\textit{chatbot} seperti, "terlihat ada kartu yang masih di 
fase x". Respon ini dinilai generik karena tidak 
menjawab "kartu yang mana yang dibicarakan?". 
Direkomendasikan agar \textit{chatbot} "menunjukkan dengan 
lebih spesifik kartu yang sedang dibicarakan".
Ini juga merupakan kegagalan eksekusi model yang paling jelas. 
\textit{Prompt} \texttt{INITIAL\_PROMPT} secara eksplisit 
memerintahkan model: "**Selalu rujuk data papan (nama kartu, 
isi checklist...)** saat memberi saran". Templat 
\texttt{USER\_PROMPT\_LA\_DATA} juga telah menyediakan data 
mentah melalui variabel \texttt{<DATA>}. Fakta bahwa model 
menghasilkan \textit{feedback} generik "ada kartu" alih-alih 
mengambil nama kartu spesifik dari \texttt{<DATA>} menunjukkan 
bahwa model gagal mematuhi instruksi eksplisit untuk 
merujuk \texttt{<DATA>} demi spesifisitas.
    \item \textbf{Minimnya Variasi Respon Motivasi}: Respon motivasi 
dinilai "kurang bervariasi" dan cenderung repetitif. 
Pakar mencontohkan bahwa \textit{chatbot} terlalu sering 
menggunakan kalimat seperti, "saya percaya anda 
memiliki kemampuan untuk menyelesaikan tugas ini".
Hal ini kemungkinan besar disebabkan oleh  
model gagal mematuhi 
terhadap instruksi kompleks dalam \texttt{ASSESSMENT\_PROMPT}. 
\textit{Prompt} tersebut meminta model untuk mendasarkan 
motivasi pada "prinsip Self-Determination Theory" (SDT). Frasa 
"saya percaya anda memiliki kemampuan..." adalah cara paling 
sederhana dan "aman" bagi model untuk memenuhi sub-prinsip 
"Tingkatkan Kompetensi" dari SDT. Daripada berimprovisasi, 
model menemukan satu frasa yang paling sesuai dengan instruksi 
teoretis tersebut dan menggunakannya berulang kali. Perlu diperhatikan
juga bahwa \textit{temperature} model diatur sangat rendah (0) sehingga
kemungkinan besar menjadi faktor kurang bervariasinya keluaran model.
\end{enumerate}

\subsubsection{Kontribusi \textit{Chatbot} terhadap Motivasi dan Kemandirian Belajar}
Pertanyaan terakhir mengevaluasi persepsi 
ahli terhadap kontribusi dan peran \textit{chatbot} 
dalam SRL. Ahli memandang \textit{chatbot} ini memiliki 
potensi signifikan untuk berfungsi sebagai "\textit{scaffolding}
yang aksesibel" bagi peserta didik.

Keunggulan utamanya adalah aksesibilitas: \textit{chatbot} 
dapat "menemani pembelajaran peserta didik kapan saja tanpa 
batasan waktu". Hal ini kontras dengan ketersediaan dosen 
atau guru yang terbatas pada waktu-waktu tertentu, di mana 
ahli menyatakan bahwa pengguna "Nggak usah ditungguin... 
harus tungguin mereka ada di kantor" (\textit{timestamp} [00:03:44,665]). 
\textit{Chatbot} dapat berfungsi "meskipun ga 
ada orang asli yang mendorong-dorongnya" (\textit{timestamp} [00:00:01,553]), 
sehingga memberikan "\textit{advantage}"  atau "Kekuatannya" 
yang cukup besar. Ketersediaan 24/7 ini memungkinkan 
peserta didik untuk "belajar sesuai dengan \textit{pace} 
masing-masing".

Pakar menyimpulkan bahwa kombinasi dari 
"kata-kata apresiasi, motivasi, dan \textit{feedback}" 
yang dihasilkan oleh \textit{chatbot} memiliki potensi 
untuk "mendorong pengguna untuk lebih semangat 
belajar" yang diungkapkan pada \textit{timestamp} .


% Pertanyaan terakhir mengevaluasi persepsi ahli 
% terhadap kontribusi dan peran \textit{chatbot} dalam SRL. 
% Ahli memandang \textit{chatbot} ini memiliki potensi 
% signifikan untuk berfungsi sebagai "\textit{scaffolding} 
% yang aksesibel" bagi peserta didik.

% Keunggulan utamanya adalah aksesibilitas: \textit{chatbot} 
% dapat "menemani pembelajaran peserta didik kapan 
% saja tanpa batasan waktu". Hal ini kontras dengan 
% ketersediaan dosen atau guru yang terbatas pada 
% waktu-waktu tertentu, sehingga memberikan "\textit{advantage} 
% yang cukup besar". Ketersediaan 24/7 ini memungkinkan 
% peserta didik untuk "belajar sesuai dengan \textit{pace} 
% masing-masing".

% Pakar menyimpulkan bahwa kombinasi dari 
% "kata-kata apresiasi, motivasi, dan \textit{feedback}" 
% yang dihasilkan oleh \textit{chatbot} memiliki potensi 
% untuk "mendorong pengguna untuk lebih semangat 
% belajar"

% \subsection{Analisis Tematik Umpan Balik Terbuka}
% Analisis terhadap umpan balik kualitatif pakar  
% mengidentifikasi beberapa tema utama yang relevan 
% untuk pengembangan \textit{Feedback Agent}.

% 1. Tonalitas Bahasa dan Keterlibatan Pengguna 
% (User Engagement) Pakar menyoroti bahwa bahasa yang 
% digunakan \textit{Feedback Agent} "terlalu kaku" untuk target 
% demografi mahasiswa. Disarankan agar bahasa dibuat 
% "lebih ringan" agar lebih menarik dan engaging. 
% Selain itu, pakar mengidentifikasi penggunaan diksi 
% yang terasa "negatif" (contoh: "utang refleksi", 
% "hanya x\%") yang berpotensi mengurangi motivasi. 
% Saran utamanya adalah agar \textit{Feedback Agent} berfokus pada 
% bahasa positif yang uplifting.

% 2. Spesifisitas dan Variasi Respon Tema ini muncul 
% sebagai kritik utama. Pakar menyatakan bahwa feedback 
% yang diberikan \textit{Feedback Agent} belum cukup spesifik. Sebagai 
% contoh, ketika \textit{Feedback Agent} merujuk pada sebuah kartu di 
% Kanban board, ia tidak menyebutkan secara spesifik kartu mana yang 
% dimaksud. Selain itu, respon "motivasi" yang diberikan dinilai 
% kurang bervariasi dan cenderung repetitif (misalnya, 
% penggunaan kalimat "saya percaya anda memiliki kemampuan...").

% 3. Tumpang Tindih Antar Kategori Respon Pakar 
% mengobservasi bahwa konten feedback, motivasi, 
% dan apresiasi terkadang sulit dibedakan. Ditemukan 
% bahwa bagian motivasi juga memberikan saran yang 
% bersifat feedback (misalnya, "anda sebaiknya memeriksa..."). 
% Meskipun demikian, pakar mencatat bahwa ketiga aspek ini 
% memang saling beririsan, sehingga hal ini dianggap bukan 
% masalah besar dalam kesatuan fungsi \textit{Feedback Agent}.

% 4. Potensi Kontribusi Chatbot dalam 
% Pembelajaran Secara keseluruhan, pakar 
% memandang positif peran chatbot. Chatbot 
% ini dinilai dapat menjadi "sistem scaffolding 
% yang aksesibel" bagi peserta didik. Keunggulan 
% utamanya adalah ketersediaan yang tidak 
% terbatas waktu, memungkinkan peserta didik 
% belajar sesuai kecepatan (\textit{pace}) masing-masing, 
% tidak seperti dosen atau guru yang 
% ketersediaannya terbatas. Kombinasi apresiasi, 
% motivasi, dan feedback diyakini dapat 
% mendorong semangat belajar pengguna.

\subsection{Validasi Kualitas oleh Ahli Manusia}
Berdasarkan keseluruhan penilaian, ahli 
memberikan kesimpulan akhir bahwa instrumen 
kuesioner ini dinyatakan "\textbf{Dapat digunakan 
dengan revisi minor}". Revisi minor yang disarankan 
oleh pakar lebih berfokus pada perbaikan 
\textit{chatbot} itu sendiri. Poin revisi utama 
meliputi penggunaan bahasa \textit{chatbot} yang 
dinilai "terlalu kaku" dan kurang sesuai dengan 
target demografi mahasiswa. Selain itu, pakar juga 
menyoroti perlunya menambah variasi dalam kalimat 
motivasi dan apresiasi agar tidak terkesan monoton.

Kritik penting lainnya adalah kurangnya spesifisitas 
\textit{feedback}. Pakar menyarankan agar 
\textit{chatbot} dapat meningkatkan kekhususan 
umpan balik, misalnya dengan secara eksplisit 
"menunjukkan... kartu yang sedang dibicarakan". 
Pakar menyimpulkan bahwa dengan perbaikan pada 
aspek aspek tersebut, \textit{chatbot} diharapkan 
dapat menjadi "lebih \textit{engaging} dan berguna"  bagi 
peserta didik dalam mendukung proses belajar mandiri 
mereka.

% Berdasarkan keseluruhan penilaian, 
% baik kuantitatif maupun kualitatif, pakar 
% memberikan kesimpulan akhir terhadap instrumen 
% yang digunakan untuk mengevaluasi chatbot. 
% Instrumen kuesioner ini dinyatakan "Dapat 
% digunakan dengan revisi minor".

% Revisi minor yang disarankan oleh pakar  
% lebih berfokus pada perbaikan chatbot itu 
% sendiri, yang sejalan dengan temuan pada 
% analisis tematik. Poin-poin revisi utama meliputi:

% Penggunaan Bahasa: Mengubah bahasa chatbot 
% agar tidak terlalu kaku dan lebih sesuai 
% dengan target demografi (mahasiswa).

% Variasi Respon: Menambah variasi dalam 
% kalimat-kalimat motivasi dan apresiasi agar 
% tidak monoton.

% Spesifisitas Feedback: Meningkatkan kekhususan 
% umpan balik, misalnya dengan secara eksplisit 
% menyebutkan kartu (tugas) yang sedang dibahas.

% Pakar menyimpulkan bahwa dengan perbaikan pada 
% aspek-aspek tersebut, chatbot diharapkan dapat 
% menjadi lebih engaging dan berguna bagi peserta 
% didik dalam mendukung proses belajar mandiri 
% mereka


% Sub bab ketiga adalah membahas tujuan penelitian kedua. Dapat ditambahkan beberapa sub bab jika diperlukan.

% \section{Perbandingan Hasil Penelitian dengan Hasil Terdahulu}


% Pembahasan penutup dapat menjelaskan mengenai kelebihan hasil pengembangan / 
% penelitian dan kekurangan dibandingkan dengan skripsi atau penelitian terdahulu atau
% perbandingan terhadap produk lain yang ada di pasaran. Penulis dapat menggunakan tabel untuk membandingkan secara gamblang dan menjelaskannya.


\section{Kelebihan dan Kekurangan Penelitian}