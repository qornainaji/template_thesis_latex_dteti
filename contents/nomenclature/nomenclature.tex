% \begin{center}
%     \textbf{DAFTAR SINGKATAN}
% \end{center}

\begin{tabular}{llp{3.5in}}
	$F1$ & \hspace{1.5cm} = & Skor F1 (rata-rata harmonik dari \textit{Precision} dan \textit{Recall}) \\
	$P$ & \hspace{1.5cm} = & \textit{Precision} (Presisi) \\
	$R$ & \hspace{1.5cm} = & \textit{Recall} \\
	AI & \hspace{1.5cm} = & \textit{Artificial Intelligence} \\
	API & \hspace{1.5cm} = & \textit{Application Programming Interface} \\
	CLI & \hspace{1.5cm} = & \textit{Command Line Interface} \\
	CoT & \hspace{1.5cm} = & \textit{Chain-of-Thought} \\
	DPO & \hspace{1.5cm} = & \textit{Direct Preference Optimization} \\
	EDM & \hspace{1.5cm} = & \textit{Educational Data Mining} \\
	GPU & \hspace{1.5cm} = & \textit{Graphics Processing Unit} \\
	IDE & \hspace{1.5cm} = & \textit{Integrated Development Environment} \\
	JSON & \hspace{1.5cm} = & \textit{JavaScript Object Notation} \\
	KCE & \hspace{1.5cm} = & \textit{Knowledge Context Engine} \\
	LA & \hspace{1.5cm} = & \textit{Learning Analytics} \\
	LACE & \hspace{1.5cm} = & \textit{Learning Analytics Context Engine} \\
	LLM & \hspace{1.5cm} = & \textit{Large Language Model} \\
	LMS & \hspace{1.5cm} = & \textit{Learning Management System} \\
	NLP & \hspace{1.5cm} = & \textit{Natural Language Processing} \\
	OOP & \hspace{1.5cm} = & \textit{Object-Oriented Programming} (Pemrograman Berorientasi Objek) \\
	ReAct & \hspace{1.5cm} = & \textit{Reasoning and Acting} \\
	RLHF & \hspace{1.5cm} = & \textit{Reinforcement Learning from Human Feedback} \\
	RNN & \hspace{1.5cm} = & \textit{Recurrent Neural Networks} \\
	RQ & \hspace{1.5cm} = & \textit{Research Question} \\
	SDT & \hspace{1.5cm} = & \textit{Self-Determination Theory} \\
	SFT & \hspace{1.5cm} = & \textit{Supervised Fine-Tuning} \\
	SRL & \hspace{1.5cm} = & \textit{Self-Regulated Learning} \\
	SUS & \hspace{1.5cm} = & \textit{System Usability Scale} \\
	WIP & \hspace{1.5cm} = & \textit{Work in Progress} \\
\end{tabular}

% \begin{center}
% 	\textbf{[SAMPLE]}
% \end{center}

% \begin{tabular}{llp{3in}}
% 	$b$	& \hspace{1.5cm} = &	bias \\
% 	$K(x_i,x_j)$ & \hspace{1.5cm} = & fungsi kernel \\ 
% 	$y$	& \hspace{1.5cm} = & kelas keluaran \\
% 	$C$	&  \hspace{1.5cm} = & parameter untuk mengendalaikan besarnya pertukaran antara penalti variabel slack dengan ukuran margin \\
% 	$L_D$	& \hspace{1.5cm} = & persamaan Lagrange dual \\
% 	$L_P$	& \hspace{1.5cm}  = &	persamaan Lagrange primal \\
% 	$\textbf{w}$ &  \hspace{1.5cm} = &	vektor bobot \\
% 	$\textbf{x}$ &  \hspace{1.5cm} = &	vektor masukan\\
% 		ANFIS &  \hspace{1.5cm} = &	Adaptive Network Fuzzy Inference System\\
% 		ANSI	&  \hspace{1.5cm} = &	American National Standards Institute\\
% 		DAG	&  \hspace{1.5cm} = & Directed Acyclic Graph\\
% 		DDAG &  \hspace{1.5cm} = & Decision Directed Acyclic Graph\\
% 		HIS	&  \hspace{1.5cm} = & Hue Saturation Intensity\\
% 		QP	&  \hspace{1.5cm} = & Quadratic Programming\\
% 		RBF	&  \hspace{1.5cm} = &	Radial Basis Function\\
% 		RGB	&  \hspace{1.5cm} = & Red Green Blue\\
% 		SV	&  \hspace{1.5cm} = &	Support Vector\\
% 		SVM	&  \hspace{1.5cm} = & Support Vector Machines\\
		
% \end{tabular}
